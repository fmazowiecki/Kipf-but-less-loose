%!TEX root =main.tex


\section{Graph neural networks}
Neural networks are a model to compute functions $f : \Rb^s \to \Rb^k$ obtained as a composition of layers $l^1,\ldots, l^m$. Every layer $l^t$ computes a function $f^{(t)} : \Rb^{s} \to \Rb^{k_t}$ based on the previous layer, where $k_0 = s$ and $k_m =k$. To ease the presentation we assume that $f^{(0)} : \Rb^s \to \Rb^s$ is the identity function. Formally, $f = l^m \circ \ldots \circ l^1 $ and every layer $l^t$ is given by
\begin{equation}\label{eq:NN}
 l^t(f^{(t-1)}) (\vec{x}) = \sigma \left(\bW^{(t)} f^{(t-1)}(\vec{x}) + b^{(t)}  \right),
\end{equation}
where: $\bW^{(t)} \in \Rb^{k_{t} \times k_{t-1}}$ is called the \emph{weight}; $b^{(t)} \in \Rb^{k_t}$ is called the \emph{bias}; and $\sigma$ is an \emph{activation function}. Activation functions are nonlinear transformations that are applied pointwise. We will discuss them in more detail later. We will study several architectures of neural networks that will be variants of~\eqref{eq:NN}, which is a standard presentation of a neural network~\cite{?}.
% The most common example is ReLU defined as $\max(\cdot, 0)$ on every point.

In this paper we are interested in architectures of \emph{graph neural networks} (GNN in short). 
There are many different architectures for GNNs in the literature and the goal of this paper is to compare these architectures in terms of expressiveness. In this paper we will present GNNs as a model computing labellings of a graph. However, a labelling of a graph is essentially a function assigning a real vector to every node and it could be also interpreted as a node classifier (as it often is in the literature). We will start by giving a very general definition of a GNN and afterwards, we will discuss how particular architectures are captured by this definition.

A GNN is defined with respect to a labelled graph $(G,\labl)$, where $G = (V,E)$ and $\labl : V \to \Rb^{k}$. A GNN computes a function $f : L_s \to L_k$, where $L_j$ is the set of all labellings $f : V \to \Rb^j$.
We will identify every labelling $f \in L_j$ with a matrix $\bF \in \Rb^{n \times j}$ by putting the label of every node in a separate row.
A GNN is obtained as composition of layers that compute labellings $l^1, \ldots, l^m$, where every layer computes $l^t(f^{(t-1)}) = f^{(t)} \in L_{k_t}$ and $f^{(0)} = \labl$. For every $f^{(t)}$ let $\bF^{(t)} \in \Rb^{n \times k_{t}}$ be its corresponding matrix. We consider architectures, where each layer $l^t$ is of the form
\begin{equation}\label{eq:GNNs}
l^t(\bF^{(t-1)}) = \sigma \left( \bN\bF^{(t-1)}\bW_1^{(t)} + \bF^{(t)}\bW_2^{(t)} + \bB^{(t)} \right),
\end{equation}
where $\bN \in \Rb^{n \times n}$ is called the \emph{neighbourhood}, $\bW_1^{(t)}, \bW_2^{(t)} \in \Rb^{k_{t-1} \times k_t}$ are called \emph{weights}, and $\bB^{(t)} \in \Rb^{n \times k_t}$ is called the \emph{bias}. The bias is restricted to matrices such that all rows of the matrix are the same. Compared to~\eqref{eq:NN} there are four main differences.
\begin{enumerate}
\item The initial $f^{(0)}$ is given by the labelling $\labl$ instead of being identity. This is because we consider labelled graphs. Unlabelled graphs are represented $\labl$ that is identical for all vertices.
 \item There is a new neighbourhood component. Intuitively, this allows us to gather the information about the neighbourhood of each node. Typically $\bN$ will be the adjacency matrix.
 \item There are two weight matrices and they appear on the right side of the previous labelling $\bF^{(t)}$. Here it is important weight matrices appear on the right because we want to restrict combining information about different rows to the neighbourhood component. The reason there are two matrices is to have separately the information about the neighbourhood and about the previous labelling.
 \item The bias is restricted. Like before, the reason is to forbid distinguishing rows with measures other than the neighbourhood information.
\end{enumerate}

% More specifically, a GNN computes for each vertex $v\in V$ a feature vector $\mathbf{F}^{(t)}_{v\bullet}$  in layer $t$ by combining:
% \begin{itemize}
% 	\item[] (i)~the feature vector $\mathbf{F}^{(t-1)}_{v\bullet}$ of $v$ from the previous layer; and
% 	\item[] (ii)~the multi-set of features from the previous layer  of $v$'s neighbors, i.e., the multi-set $\ldbl \mathbf{F}^{(t-1)}_{u\bullet} \st u \in N_G(v) \rdbl$.
% \end{itemize}
% One typically parametrises GNNs by how this combination is achieved.

We will discuss a more abstract definition of a GNN than~\eqref{eq:GNNs}.
For each layer $t$ we consider two functions:
combination function $f_{\textsl{comb}}^{(t)}$; and aggregation function
$f_{\textsl{agg}}^{(t)}$. The layers are defined as follows. For each vertex $v\in V$ and layer $t$:
\begin{equation}
\mathbf{F}^{(t)}_{v\bullet}:=
f_{\textsl{comb}}^{(t)}\Bigl(
\mathbf{F}_{v\bullet}^{(t-1)},f_{\textsl{aggr}}^{(t)}\bigl(\ldbl \mathbf{F}^{(t-1)}_{u\bullet} \st u \in N_G(v) \rdbl\bigr)
\Bigr), \label{lab:generalGNN}
\end{equation}
where $\mathbf{F}^{(t)} = l^{t}(\mathbf{F}^{(t-1)})$.


Finally, if a GNN consists of $L$ layers then
$\mathbf{F}^{(L)}$ is used to classify the vertices in $G$ different classes. More precisely, vertices $v$ and $w$ will belong to the same class if and only if 
$\mathbf{F}^{(L)}_{v\bullet}=\mathbf{F}^{(L)}_{w\bullet}$, i.e., when their feature vectors are equal.

The way that GNNs update feature vectors is clearly reminiscent of how the WL procedure works. The relationship between vertex classification by GNNs of the form~(\ref{lab:generalGNN}) and vertex classification by WL can be made precise~\cite{grohewl,DBLP:conf/iclr/XuHLJ19}. To state this relation formally, we define the notions of being \textit{bounded by WL}, for upper bounding the classification power of GNNs, and being \textit{WL-strong}, for lower bounding the classification power of GNNs~\cite{grohewl}. 

\begin{definition}[Bounded by WL]\normalfont\label{def:wlupper}
Let $(G,\pmb{\ell})$ be a labeled graph and let $\pmb{\ell}^{(0)}:=\pmb{\ell}$. Furthermore, denote by 
$\pmb{\ell}^{(t)}$ the vertex labeling obtained after $t$ iterations of the WL procedure, starting from $\pmb{\ell}^{(0)}$. Let $\mathbf{F}^{(t)}$ be feature matrices computed using a GNN of the form ~(\ref{lab:generalGNN}), for $t\geq 0$. Then, given $\mathbf{F}^{(0)}$ such that $\pmb{\ell}^{(0)}\sqsubseteq \mathbf{F}^{(0)}$ holds, one says that the GNN  is \textit{bounded by WL} if for all $t\geq 0$, 
$\pmb{\ell}^{(t)}\sqsubseteq \mathbf{F}^{(t)}$.
% That is, the vertex labeling induced by $\mathbf{F}^{(t)}$ is always coarser than the vertex labeling $\pmb{\ell}^{(t)}$  obtained by WL after $t$ iterations.
\end{definition}
In terms of vertex classification this implies that when a GNN is bounded by WL, then 
for every $t$, when WL classifies two vertices as the same in step $t$, then so does the GNN. 

The boundedness property holds for \textit{any} GNN of the form~(\ref{lab:generalGNN}):
\begin{proposition}[~\cite{grohewl,DBLP:conf/iclr/XuHLJ19}]\label{prop:upperboundgeneral}
% Let $(G,\pmb{\ell})$ be a labeled graph and let $\mathbf{F}^{(0)}$ consist of feature vectors such that $\pmb{\ell}^{(0)}\sqsubseteq \mathbf{F}^{(0)}$. Then, $\pmb{\ell}^{(t)}\sqsubseteq \mathbf{F}^{(t)}$ for every $t>0$, where $\mathbf{F}^{(t)}$ is computed by a
Any GNN of the form~(\ref{lab:generalGNN}) is bounded by WL.
\end{proposition}


When the combination and aggregation functions are assumed to be all injective one can further show that GNNs of the form~(\ref{lab:generalGNN}) are also as powerful as WL~\cite{DBLP:conf/iclr/XuHLJ19}. That is, when two vertices are classified as the same by the GNN then they are also classified the same by WL. 
We can phrase this more generally in terms of \textit{classes} of GNNs. Here, a class of GNNs is simply a collection of GNNs of the form~(\ref{lab:generalGNN}) in which the allowed aggregation and combination functions are restricted to belong to some class. For example, one could consider the class of injective functions for the combination and aggregation functions.

\begin{definition}[WL-strong]\normalfont\label{def:wlstrong}
A \textit{class} of GNNs of the form~(\ref{lab:generalGNN}) is \textit{WL-strong} if for any given labeled graph
	$(G,\pmb{\ell})$, there exist instantiations of the combination and aggregation functions in that class such that for all $t\geq 0$, $\mathbf{F}^{(t)}\sqsubseteq \pmb{\ell}{}^{(t)}$.
\end{definition}

Clearly, for classes of GNNs of the form~(\ref{lab:generalGNN}) that are WL-strong, 
Proposition~\ref{prop:upperboundgeneral} implies that for every labeled graph $(G,\pmb{\ell})$ there
exist instantiations of the combination and aggregation functions in that class such that
for all $t\geq 0$, $\mathbf{F}^{(t)}\equiv \pmb{\ell}{}^{(t)}$. That is, there is at least one particular instantiation of the GNNs in the class that is equivalent to WL in terms of vertex classification.

A specific class of GNNs which is known to be WL-strong~\cite{grohewl} is given by:
\begin{equation}
\mathbf{F}^{(t)}:=\sigma\Bigl(\mathbf{F}^{(t-1)}\mathbf{W}_1^{(t)}+ \mathbf{A}\mathbf{F}^{(t-1)}\mathbf{W}_2^{(t)} + \mathbf{B}^{(t)}\Bigr), \label{eq:groheGNN}
\end{equation}
where $\mathbf{W}_1^{(t)}$, $\mathbf{W}_2^{(t)}$  are weight matrices, $\mathbf{B}^{(t)}$ are 
constant\footnote{A constant matrix is matrix which is a multiple of the all-ones matrix $\mathbf{J}$.} bias matrices, and $\sigma$ is the sign function. It is readily verified that GNNs of the form~(\ref{eq:groheGNN}) can be seen as  GNNs of the form~(\ref{lab:generalGNN}). Hence, GNNs of the form~(\ref{eq:groheGNN}) are also bounded by WL by Proposition~\ref{prop:upperboundgeneral}.

More specifically, it was shown in~\cite{grohewl} that even the class of GNNs of the form
\begin{equation}
\mathbf{F}^{(t)}:=\sigma\Bigl(\mathbf{F}^{(t-1)}\mathbf{W}_1^{(t)}+ \mathbf{A}\mathbf{F}^{(t-1)}\mathbf{W}_2^{(t)} - \mathbf{J}\Bigr), \label{eq:groheGNNwithJ}
\end{equation}
is WL-strong. That is, compared to class of GNNs of the form~(\ref{eq:groheGNN}), the bias matrices $\mathbf{B}^{(t)}$ is now chosen uniformly (for all layers) to be $-\mathbf{J}$.


\openprob{Although the result reported in~\cite{grohewl} is correct, it is not entirely clear how their construction translates into an architecture of the form~\ref{eq:groheGNN}. Their construction fits, however, in an architecture of the
form $\mathbf{F}^{(t)}:=\Bigl(\mathbf{F}^{(0)},\sigma\bigl(\mathbf{F}^{(t-1)}\mathbf{W}_1^{(t)}+ \mathbf{A}\mathbf{F}^{(t-1)}\mathbf{W}_2^{(t)} -\mathbf{J}\bigr)\Bigr)$ which is not exactly in the form of ~\ref{eq:groheGNN}. We will recover the WL-strongness result from ~(\ref{eq:groheGNN}) as a side result of our analysis, however. {\bf BUT: is it true that their update rule cannot be expressed as they claim??}}.
\floris{It seems to be possible after all, provided that a slightly more general bias matrix is used. That is, to show strongness we can consider. Please check. An additional assumption is that $\sigma(\mathbf{F}^{(0)})=\mathbf{F}^{(0)}$.}
$$
[\mathbf{F}^{(0)},\mathbf{F}^{(t)}]:=\sigma\left([\mathbf{F}^{(0)},\mathbf{F}^{(t-1)}]\begin{pmatrix}
\mathbf{I}_{n\times n} & \mathbf{O}_{n\times n}\\
\mathbf{O}_{n\times n} & \mathbf{O}_{n\times n}\end{pmatrix}
+\mathbf{A}[\mathbf{F}^{(0)},\mathbf{F}^{(t-1)}]
\begin{pmatrix}
\mathbf{O}_{n\times n} & \mathbf{O}_{n\times n}\\
\mathbf{O}_{n\times n} & \mathbf{W}_{n\times n}^{(t-1)}\end{pmatrix}-
q\begin{pmatrix}
\mathbf{O}_{n\times n} & \mathbf{J}_{n\times n}\\
\mathbf{O}_{n\times n} & \mathbf{J}_{n\times n}\end{pmatrix}
\right),
$$
where $\mathbf{W}^{(t-1)}$ is the matrix constructed in Grohe's lower bound proof.
\floris{Our lower bound shows that it holds also for a simpler GNN architecture, however,
see below, and also we can deal with ReLU.}

In~\cite{grohewl}, also GNNs of the form~\ref{eq:groheGNNwithJ} where considered in which the activation function $\sigma$ is the ReLU function. In this case, the relationship
$\mathbf{F}^{(2t)}\sqsubseteq \pmb{\ell}^{(t)}$ holds, for every $t\geq 0$~\cite{grohewl}.
Strictly speaking this architecture is not WL strong in the sense of Definition~\ref{def:wlstrong}. Later in the paper we show that the factor two can be avoided, however, by allowing a bias matrix of the form $-q\mathbf{J}$ for some parameter $q$.
In fact, we establish WL-strongness for
an even more restricted class of GNNs of the form:
\begin{align}
\mathbf{F}^{(t)}&:=\sigma\Bigl(p\mathbf{F}^{(t-1)}\mathbf{W}^{(t)}+ \mathbf{A}\mathbf{F}^{(t-1)}\mathbf{W}^{(t)} - q\mathbf{J}\Bigr), \label{eq:GNN}\\
&:=\sigma\Bigl((\mathbf{A}+p\mathbf{I})\mathbf{F}^{(t-1)}\mathbf{W}^{(t)} - q\mathbf{J}\Bigr), \label{eq:GNN2}
\end{align}
for parameters $p, q$ such that $0< p,q<1$ is WL-strong, where $\sigma$ is either the sign or ReLU activation function. Here, $\mathbf{I}$ is the identity matrix of appropriate dimension.
Compared to~(\ref{eq:groheGNNwithJ}) we only have one weight matrix $\mathbf{W}^{(t)}$ in each layer and have two additional parameters $p$ and $q$\footnote{A technical condition for obtaining WL-strongness is that  the initial feature matrix $\mathbf{F}^{(0)}$ satisfies $\mathbf{F}^{(t)}\equiv \pmb{\ell}^{(0)}$  and that
the unique rows in $\mathbf{F}^{(0)}$ are linearly independent. It will become clear later why this assumption is needed.}



%
% \todo{I believe that this lower bound indeed holds, but the proof given in ~\cite{grohewl} is not correct. I do not see how their approach for the labeled case can be squeezed in the above form. I emailed Martin about this (I keep you in the loop when I get some response.). Why does it hold, because we show later that
% $$
% \mathbf{F}^{(t)}:=\sigma\Bigl((\mathbf{A}+p\mathbf{I})\mathbf{F}^{(t-1)}\mathbf{W}_2^{(t)} +\mathbf{W}_3^{(t)}\Bigr),
% $$
% is WL-strong for $p<1$, which is an instance of that more general architecture.
% }

% \todo{Argue that the normalized adjacency and augmented adjacency (Kipf) do not fit in this framework, whereas the others do but are not WL strong. This motivates to look at a small unifying architecture and study its expressive power.
% }
% A graph neural network (GNN) model consists of layers, where each layer specifies how to update a vertex labeling $\mathbf{F}$. A GNN with $k$ layers is defined by updates of $\mathbf{F}^{(t)}$ for $t = 0, \ldots,k$, which denotes the labeling obtained after $t$ layers. A new labeling $\mathbf{F}^{(t+1)}$ is obtained inductively by transformations defined on the previous labeling $\mathbf{F}^{(t)}$. An \emph{architecture} specifies what kind of transformations are allowed.


Although GNN architectures of the form~(\ref{eq:groheGNN}) or~(\ref{eq:GNN}) are quite general,  they do not capture some common GNN architectures used in the literature:
%
\floris{We may want to add references to the architectures below. This requires some googling around. Also, not entirely happy with the acronym's.
Better suggestions are welcome!}
% Well-known GNN architectures can be obtained from~(\ref{eq:architecture}) by varying $\mathbf{L}$ and $\mathbf{R}$, $p$ and $q$, as follows:
\begin{description}
% \item[\textit{Adjacency}:]
% % $\mathbf{L}=\mathbf{R}:=\mathbf{I}$, $p=q:=0$. Hence,
% $
% \mathbf{F}^{(t+1)}:=\sigma\left(\mathbf{A}\mathbf{F}^{(t)}\mathbf{W}^{(t)}\right)
% $
\item[\textit{Random walk (RW-GNN)}:] 
% $\mathbf{L}:=\mathbf{D}^{-1}$ with $\mathbf{D}$ the degree matrix of $\mathbf{A}$, $\mathbf{R}:=\mathbf{I}$, $p=q:=0$. Hence,
$$
\mathbf{F}^{(t)}:=\sigma\left(\mathbf{D}^{-1}\mathbf{A}\mathbf{F}^{(t-1)}\mathbf{W}^{(t-1)}\right);$$
\item[\textit{Normalized adjacency (NA-GNN)}:] 
% $\mathbf{L}=\mathbf{R}:=\mathbf{D}^{-1/2}$, $p=q:=0$. Hence,
$$
\mathbf{F}^{(t)}:=\sigma\left(\mathbf{D}^{-1/2}\mathbf{A}\mathbf{D}^{-1/2}\mathbf{F}^{(t-1)}\mathbf{W}^{(t-1)}\right)
;$$
\item[\textit{$1$st Order GCN (1-GCN)}:]
$$
\mathbf{F}^{(t)}:=\sigma\left(\mathbf{F}^{(t-1)}\mathbf{W}_1^{(t-1)}+\mathbf{D}^{-1/2}\mathbf{A}\mathbf{D}^{-1/2}\mathbf{F}^{(t-1)}\mathbf{W}_2^{(t-1)}\right);$$

\item[\textit{Simplified $1$st Order GCN (1-GCNs)}:]
$$
\mathbf{F}^{(t)}:=\sigma\left((\mathbf{D}^{-1/2}\mathbf{A}\mathbf{D}^{-1/2}+\mathbf{I})\mathbf{F}^{(t-1)}\mathbf{W}_2^{(t-1)}\right);$$
\item[\textit{Augmented random walk (RW-GNN+)}:] 
% $\mathbf{L}:=\tilde{\mathbf{D}}^{-1}$ with $\tilde{\mathbf{D}}$ the degree matrix of $\mathbf{A}+\mathbf{I}$, $\mathbf{R}:=\mathbf{I}$, $p:=1$, $q:=0$. Hence,
$$
\mathbf{F}^{(t)}:=\sigma\left(\tilde{\mathbf{D}}^{-1}(\mathbf{A}+\mathbf{I})\mathbf{F}^{(t-1)}\mathbf{W}^{(t-1)}\right),$$
with $\tilde{\mathbf{D}}$ the degree matrix of $\mathbf{A}+\mathbf{I}$; and
\item[\textit{Augmented adjacency (NA-GNN+)}~\cite{kipf-loose}:] % $\mathbf{L}=\mathbf{R}:=\tilde{\mathbf{D}}^{-1/2}$, $p:=1$, $q:=0$. Hence,
$$
\mathbf{F}^{(t)}:=\sigma\left(\tilde{\mathbf{D}}^{-1/2}(\mathbf{A}+\mathbf{I})\tilde{\mathbf{D}}^{-1/2}\mathbf{F}^{(t-1)}\mathbf{W}^{(t-1)}\right);
$$
\item[\textit{Weighted augmented adjacency (NA-GNN++)}~\cite{DBLP:journals/corr/abs-1905-03046}:]
$$\mathbf{F}^{(t)}:=\sigma\left((r\mathbf{I}+(1-r)\tilde{\mathbf{D}})^{-1/2}(\mathbf{A}+p\mathbf{I})(r\mathbf{I}+(1-r)\mathbf{D})^{-1/2}\mathbf{F}^{(t-1)}\mathbf{W}^{(t-1)}\right),
$$
which underlies the PiNet architecture.
\end{description}
% Bias can be added to all previous architectures by setting $q\neq 0$.
This begs the question how these architectures relate to WL. This question is of particular importance for the augmented adjacency architecture which is often regarded as a continuous version of WL~\cite{kipf-loose}
\floris{We may want to sell the connection with Kipf a bit more, especially as it served as our motivation to look in all. this.}
Indeed, it is mentioned in ~\cite{kipf-loose} that the augmented adjacency architecture can be interpreted -- loosely speaking -- as differentiable and parameterized WL process on graphs. This claim has, to our knowledge, not been made formal.


% \todo{I believe that we separate these architectures based on:
% \begin{itemize}
% 	\item Presence/absence of $\mathbf{I}$: This relates to labeled vs unlabeled graphs.
% 	\item Presence/absence of $\mathbf{R}$: This relates to being one-step ahead compared to WL.
% 	\item Presence/absence of $\mathbf{L}$: ??
% 	\item parameter $p<1$: this is crucial for being WL-strong.
% \end{itemize}}
To analyze the expressive power of all these GNN architectures we therefore consider a general GNN architecture of the form:
% We consider a GNN architecture which generalises commonly used GNN architectures. Given a labeled
% graph $(G,\pmb{\ell})$ with $G=(V,E)$, we denote by $\mathbf{F}^{(t)}$ the feature matrix assigning to each vertex $v\in V$ a feature vector $\mathbf{F}_{v\bullet}$. In layer $t$ of the the GNN architecture, $\mathbf{F}^{(t)}$ is updated as follows:
\begin{equation}
\mathbf{F}^{(t)}:=\sigma\left(\mathbf{F}^{(t-1)}\mathbf{W}_1^{(t-1)}+\mathbf{L}(\mathbf{A}+p\mathbf{I})\mathbf{R}\mathbf{F}^{(t-1)}\mathbf{W}_2^{(t-1)} + q\mathbf{J}\right), \label{eq:architecture}
\end{equation}
where $\mathbf{L}$ and $\mathbf{R}$ positive diagonal matrices, $p$ and $q$ are  learnable parameters in $[0,1]$, and $\mathbf{W}_1^{(t-1)}$ and $\mathbf{W}_2^{(t-1)}$  are learnable weight matrices, and finally, $\sigma$ is a non-linear activation function such as sign or ReLU. It should be clear that by choosing 
$\mathbf{L}$, $\mathbf{R}$, $p$ and $q$  in an appropriate way, one can obtain all  GNN architectures mentioned so far.

In the remainder of the paper we investigate the expressive power of GNNs of the form~(\ref{eq:architecture}) in detail. We remark that our proof techniques are similar to those used in~\cite{grohewl} but with some minor twists.

% above. Furthermore, we see that~(\ref{eq:architecture}) also encompasses the architecture~(\ref{eq:GNN2}).
% Our analysis also works for the even more general setting
% \begin{equation}
% \mathbf{F}^{(t+1)}:=\sigma\left(\mathbf{F}^{(t)}\mathbf{W_1}+\mathbf{L}(\mathbf{A}+p\mathbf{I})\mathbf{R}\mathbf{F}^{(t)}\mathbf{W}_2^{(t)} + \mathbf{W_3}\right), \label{eq:architecture}
% \end{equation}
% \todo{Perhaps we should use this general version. It does not make a difference of the upper bound, and we do not need $\mathbf{F}^{(t)}\mathbf{W_1}$ to show it is WL strong in our modified sense.}

% \section{Expressive Power}
% It is common to measure the expressive power of GNN architectures in comparison with the 1-WL algorithm~\cite{grohewl,DBLP:conf/iclr/2019}.  Complementary, if for \textit{any} given labeled graph
% $(G,\pmb{\ell})$, there exists learnable parameters of the GNN architecture such that for all $t\geq 0$, $\pmb{\ell}{}^{(t)}\equiv \mathbf{F}^{(t)}$, then one says that the GNN architecture is \textit{1-WL strong}. We next investigate these properties for GNN architectures of the form~(\ref{eq:architecture}).

