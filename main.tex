\documentclass[a4paper]{article}

\usepackage{arxiv}

\usepackage[utf8]{inputenc} % allow utf-8 input
\usepackage[T1]{fontenc}    % use 8-bit T1 fonts
\usepackage{hyperref}       % hyperlinks
\usepackage{url}            % simple URL typesetting
\usepackage{booktabs}       % professional-quality tables
\usepackage{amsfonts}       % blackboard math symbols
\usepackage{nicefrac}       % compact symbols for 1/2, etc.
\usepackage{microtype}      % microtypography
\usepackage{multirow}       % table tricks
\usepackage{textcomp}       % quotations

% Our own Macros
\usepackage{amssymb,amsthm,amsmath}
\usepackage[textsize=small]{todonotes}
\presetkeys{todonotes}{inline}{}

%comments
\newcommand{\filip}[1]{\todo[inline, color=green!10]{{\bf Filip:} #1}}
\newcommand{\openprob}[1]{\todo[inline, color=red!10]{{\bf Open problem}: #1}}
\newcommand{\floris}[1]{\todo[inline, color=blue!10]{{\bf Floris:} #1}}
\newcommand{\tobdone}[1]{\todo[inline, color=blue!30]{{\bf TODO:} #1}}

\newcommand{\hash}{\textsc{Hash}}
\providecommand{\st}{}
\renewcommand{\st}{\mathrel{\mid}}
\newcommand{\ldbl}{\{\!\!\{}
\newcommand{\rdbl}{\}\!\!\}}
\newcommand{\Rb}{\mathbb{R}}
\newcommand{\Nb}{\mathbb{N}}
\newcommand{\bF}{\mathbf{F}}
\newcommand{\bB}{\mathbf{B}}
\newcommand{\bC}{\mathbf{C}}
\newcommand{\bW}{\mathbf{W}}
\newcommand{\bN}{\mathbf{N}}
\newcommand{\bA}{\mathbf{A}}
\newcommand{\bJ}{\mathbf{J}}
\newcommand{\labl}{\pmb{\ell}}
\newcommand{\labm}{\pmb{m}}

\newcommand{\architecture}{\mathcal{M}}
\newcommand{\architectureWL}{\mathcal{M}_{\textsl{WL}}}
\newcommand{\architectureano}{\mathcal{M}_{\textsl{ano}}}
\newtheorem{lemma}{Lemma}
\newtheorem{theorem}{Theorem}
\newtheorem{conjecture}{Conjecture}
\newtheorem{proposition}{Proposition}
\newtheorem{definition}{Definition}
\newtheorem{corollary}{Corollary}
\newtheorem{example}{Example}
\newtheorem{observation}{Observation}

\title{Loosely Stated, Formally Proved: Kipf and Welling's GCNs
are Almost 1WL Powerful}

\author{
  Floris Geerts\\
  University of Antwerp\\
  \texttt{floris.geerts@uantwerpen.be}
  \And
  Filip Mazowiecki\\
  Max Planck Institute for Software Sciences\\
  \texttt{filipm@mpi-sws.org}
  \And
  Guillermo A. P\'erez\\
  University of Antwerp\\
  \texttt{guillermoalberto.perez@uantwerpen.be}
  \And
  Adri\'an Soto\\
  Pontificia Universidad Cat\'olica de Chile\\
  \texttt{assoto@uc.cl}
}

\begin{document}

\maketitle

\begin{abstract}
  We adapt ideas from~\cite{grohewl} on the equivalent power of graph neural
  networks and the Weisfeiler-Lehman algorithm.  Then, we leverage the latter
  to provide a rigurous proof that Kipf and Welling's graph convolutional
  networks are indeed Weisfeiler-Leman powerful (and not just loosely speaking
  so).
\end{abstract}

\clearpage

\tableofcontents

\clearpage

%!TEX root =main.tex
\section{Introduction}\label{sec:intro}

\floris{Clearly, this needs to be written. Let's wait until we have the complete story.}
%!TEX root =main.tex
\section{Preliminaries}
\floris{This section may need updating when we finish the rest.
Some notations, like $A_{v\bullet}$ etc are missing. So, we check at the end.}
Let $G=(V,E)$ be an undirected graph consisting of $n$ vertices.
Given a vertex $v\in V$, we denote by $N_G(v)$ its set of neighbors, i.e., $N_G(v):=\{u\st \{u,v\}\in E\}$. Furthermore, the degree of a vertex $v$, denoted by $d_{v}$, is the number of vertices in $N_G(v)$. With a labeled graph $(G,\pmb{\ell})$ we mean a graph $G=(V,E)$ whose vertices are labeled using a function $\pmb{\ell}:V\to \Sigma$
for some set $\Sigma$ of labels.

Given $G=(V,E)$, we denote by $\mathbf{A}$ its adjacency matrix of dimension $n \times n$ such that the entry $\mathbf{A}_{vw}=1$ if $\{v,w\}\in E$ and  $\mathbf{A}_{vw}=0$ otherwise. We denote by $\mathbf{D}$ the diagonal matrix such that $\mathbf{D}_{vv}=d_v$ for each $v\in V$. Throughout the paper we will assume that $G$ does not have isolated nodes, which is equivalent to assuming that $\mathbf{D}$ does not have any $0$ entries on the diagonal. We will also assume that there are no self loops, so the diagonal of $\mathbf{A}$ is filled with $0$s.

Given a labeled graph $(G,\pmb{\ell})$, it will be convenient to regard the vertex labeling $\pmb{\ell}$ as a vector in $\Rb^{n\times 1}$ such that $\pmb{\ell}_v:=\pmb{\ell}(v)$. Here, without loss of generality, we silently assume an embedding of  the set $\Sigma$ of labels in $\Rb$. More generally, we also consider vertex labelings in which 
vertices in $G$ are labeled  with vectors from $\Rb^q$, for some dimension $q \in \Nb$. 
Given a matrix $\mathbf{F} \in \Rb^{n \times q}$, we refer to the vertex labeling (induced by)
$\mathbf{F}$ as the labeling which associates vertex $v$ with label the row vector  $\mathbf{F}_{v\bullet}$.

It will be important later on to be able to compare two labelings of $G$.
Given a matrix $\mathbf{F} \in \Rb^{n\times q}$ and a matrix $\mathbf{F}' \in \Rb^{n\times q'}$ we say that the
vertex labeling  $\mathbf{F}'$ is coarser than the vertex labeling $\mathbf{F}$, denoted by $\mathbf{F}\sqsubseteq \mathbf{F}'$, if
for all $v,w\in V$,
$
\mathbf{F}_{v\bullet}=\mathbf{F}_{w\bullet} \Rightarrow \mathbf{F}'_{v\bullet}=\mathbf{F}'_{w\bullet}
$
The vertex labelings $\mathbf{F}$ and $\mathbf{F}'$ are equivalent, denoted by $\mathbf{F}\equiv\mathbf{F}'$, if $\mathbf{F}\sqsubseteq \mathbf{F}'$ and
$\mathbf{F}'\sqsubseteq \mathbf{F}$ hold. In other words, $\mathbf{F}\equiv\mathbf{F}'$ if and only if for all $v,w\in V$,
$
\mathbf{F}_{v\bullet}=\mathbf{F}_{w\bullet} \Leftrightarrow \mathbf{F}'_{v\bullet}=\mathbf{F}'_{w\bullet}
$.

Of particular importance is the labeling obtained by color refinement, also known as Weisfeiler-Lehman (or WL, for short). The WL procedure constructs a labeling, in an incremental fashion, based on neighborhood information. More specifically, consider a labeled graph $(G,\pmb{\ell})$. Initially, 
$\pmb{\ell}^{(0)}:=\pmb{\ell}$. Then, the WL procedure computes a labeling $\pmb{\ell}^{(t)}$, for $t> 0$, as follows: 
$$
\pmb{\ell}^{(t)}_v:=\textsc{Hash}\Bigl(\bigl(\pmb{\ell}^{(t-1)}_v,\ldbl \pmb{\ell}_u^{(t-1)} \st u \in N_G(v) \rdbl\bigr)\Bigr),
$$
where $\textsc{Hash}$ bijectively maps the above pair, consisting of (i)~the previous label 
$\pmb{\ell}^{(t-1)}_v$ of $v$; and (ii)~the multi-set $\ldbl \pmb{\ell}_u^{(t-1)} \st u \in N_G(v) \rdbl$ of labels of $v$'s neighbors, to a unique label in $\Sigma$, which has not been used in previous iterations. When the number of distinct labels in $\pmb{\ell}^{(t)}$ and $\pmb{\ell}^{(t-1)}$ is the same, the 1-WL algorithm terminates.
Termination is guaranteed in at most $n$ steps. We refer to the resulting labeling as the \textit{WL labeling of $(G,\pmb{\ell})$}. 
%!TEX root =main.tex

\section{Message Passing Neural Networks}
We start by describing message passing neural networks (MPNNs) for  deep learning on graphs, introduced by \cite{GilmerSRVD17}. Roughly speaking, in MPNNs, vertex features are propagated through a graph according to its connectivity structure. MPNNs are known to model a variety of graph neural network architectures commonly used in practice.

\subsection{Definition}
Given a labeled graph $\langle G,\pmb{\nu}\rangle$, an MPNN computes a vertex labeling $\pmb{\ell}:V\to \mathbb{Q}^{s}$, for some $s\in\mathbb{N}^+$, by means of a number of rounds of computation, starting from the input labeling $\pmb{\nu}:V\to\mathbb{Q}^s$.

The vertex labeling computed by an MPNN after round $t$ is denoted by $\pmb{\ell}^{(t)}$. We next detail how $\pmb{\ell}^{(t)}$ is computed.
\begin{description}\setlength{\itemsep}{-0.4ex}
\item [Initialisation.]  We let $\pmb{\ell}^{(0)}:=\pmb{\nu}$.
\end{description}
Then, for round $t=1,2,\ldots,d$, we define $\pmb{\ell}^{(t)}:V\to\mathbb{Q}^{s}$, as follows\footnote{We can assume, without loss of generality, that every round assigns labels in $\mathbb{Q}^s$ for the same $s$. If not, one can include ``padding with zeroes'' in the message and update functions.}:
\begin{description}\setlength{\itemsep}{-0.4ex}
\item [Message Passing.] Each vertex $v\in V$ receives messages from its neighbours which are subsequently aggregated:
$$
\mathbf{m}^{(t)}_{v}:=\sum_{u\in N_G(v)}\textsc{Msg}^{(t)}\left(\pmb{\ell}^{(t-1)}_v,\pmb{\ell}^{(t-1)}_u,v,u\right)\in\mathbb{Q}^{s}.
$$
\item [Updating.] Each vertex $v\in V$ further updates $\mathbf{m}^{(t)}_{v}$ possibly based on its current label $\pmb{\ell}^{(t-1)}_v$:
$$
\pmb{\ell}^{(t)}_v:=\textsc{Upd}^{(t)}\left(\pmb{\ell}^{(t-1)}_v,\mathbf{m}^{(t)}_{v}\right)\in\mathbb{Q}^{s}.
$$
\end{description}
Here, the message functions $\textsc{Msg}^{(t)}$ and update functions $\textsc{Upd}^{(t)}$ are general (computable) functions.
After round $d$, we define the final labeling $\pmb{\ell}:V\to\mathbb{Q}^{s}$ as  $\pmb{\ell}_v:=\pmb{\ell}^{(d)}_v$ for every $v\in V$. If further aggregation over the entire graph is needed, e.g., for graph classification, an additional readout function 
$\textsc{ReadOut}(\ldbl\pmb{\ell}_v\mid v\in V\rdbl)$ can be applied. We omit the readout function here since most of the computation happens during the rounds of an MPNN.

\subsection{Examples}
We illustrate MPNNs by a number of examples in which the message functions leverage an increasing amount of information of the vertices involved. 

\paragraph{Anonymous MPNNs.}
We start with two examples of so-called \textit{anonymous}  MPNNs. These are MPNNs whose message functions do not depend on the vertices $v$ and $u$. Phrased otherwise, anonymous MPNNs have message functions only depending on $\pmb{\ell}_v^{(t-1)}$ and $\pmb{\ell}_u^{(t-1)}$.
 % will be referred to as \textit{anonymous MPNNs}, since the message functions do not know which vertices are being considered.
\begin{example}[GNN architectures]\normalfont
We first consider
the graph neural network
architectures~\cite{Hamilton2017a,grohewl} defined by:
\begin{equation}
\mathbf{L}^{(t)}:=\sigma\left(\mathbf{L}^{(t-1)}\mathbf{W}_1^{(t)}+\mathbf{A}_G\mathbf{L}^{(t-1)}\mathbf{W}_2^{(t)}+\mathbf{B}^{(t)}\right), \label{gnn:grohe}
\end{equation}
where $\mathbf{L}^{(t)}$ is the $n\times s$-matrix in $\mathbb{Q}^{n\times s}$, consisting of the $n$ rows $\pmb{\ell}^{(t)}_v$, for $v\in V$, $\mathbf{A}_G\in\mathbb{Q}^{n\times n}$ is the adjacency matrix of $G$, $\mathbf{W}_1^{(t)}$ and $\mathbf{W}_2^{(t)}$ are (learnable) weight matrices in $\mathbb{Q}^{s\times s}$,
$\mathbf{B}^{(t)}$ is a bias matrix in $\mathbb{Q}^{n\times s}$ consisting of $n$ copies of the same row $\mathbf{b}^{(t)}\in \mathbb{Q}^s$, and $\sigma$ is a non-linear activation function. We can regard this architecture as an MPNN. Indeed,~(\ref{gnn:grohe}) can be equivalently phrased as the architecture which computes, in round $t$, for each each vertex $v\in V$ the label defined by:
$$
\pmb{\ell}^{(t)}_v:=\sigma\Bigl(\pmb{\ell}^{(t-1)}_v\mathbf{W}_1^{(t)}+ \sum_{u\in N_G(v)}\pmb{\ell}^{(t-1)}_u\mathbf{W}_2^{(t)}+\mathbf{b}^{(t)} \Bigr),
$$
where we identified the labelings with their images, i.e., a row vector in $\mathbb{Q}^s$. 
To phrase this as an MPNN, it suffices to define for each $\mathbf{x}$ and $\mathbf{y}$ in $\mathbb{Q}^s$, each $v\in V$ and $u\in N_G(v)$, and each $t\geq 1$:
\begin{equation*}
	\textsc{Msg}^{(t)}\bigl(\mathbf{x},\mathbf{y},v,u):=\mathbf{y}\mathbf{W}_2^{(t)}
\text{ and } 
\textsc{Upd}^{(t)}(\mathbf{x},\mathbf{y}):=\sigma\left(\mathbf{x}\mathbf{W}_1^{(t)}+\mathbf{y} + \mathbf{b}^{(t)}\right).
\end{equation*} 
We observe that  message functions indeed do not depend on $v$ and $u$. \qed
\end{example}
Another example of an anonymous MPNN originates from the Weisfeiler-Lehman procedure described in the preliminaries.
\begin{example}[Weisfeiler-Lehman]\normalfont
We recall that WL computes, in round $t$, for each vertex $v\in V$ the label:
$$
\pmb{\ell}^{(t)}_v:=\textsc{Hash}\left(\pmb{\ell}^{(t-1)}_v,\ldbl \pmb{\ell}_u^{(t-1)} \st u \in N_G(v) \rdbl\right).
$$
We can cast this as an anonymous MPNN, as follows. 
\floris{Please complete.}
\qed
 \end{example}

% MPNNs with update functions only depending on $\pmb{\ell}_v^{(t)}$ and $\pmb{\ell}_u^{(t)}$ will be referred to as \textit{anonymous MPNNs}, since the message functions do not know which vertices are being considered.



\paragraph{Degree-aware MPNNs.} In our next example, the message functions use a bit more information. More specifically they use  degree information of the vertices.
MPNNs whose message functions depend on 
 $\pmb{\ell}_v^{(t-1)}$, $\pmb{\ell}_u^{(t-1)}$, $d_v$ and $d_u$ will be referred to as \textit{degree-aware} MPNNs. 

\begin{example}[GCN by Kipf and Welling]\label{ex:KipfasMPNN}\normalfont
We consider the GCN architecture by~\cite{kipf-loose}, which in round $t$ computes for each vertex $v\in V$ the label:
\begin{equation}
\pmb{\ell}^{(t)}_v:=\sigma\Bigl(\bigl(\frac{1}{1+d_v}\bigr)\pmb{\ell}_v^{(t-1)}\mathbf{W}^{(t)} + \sum_{u\in N_G(v)} \bigl(\frac{1}{\sqrt{1+d_v}}\bigr)\bigl(\frac{1}{\sqrt{1+d_u}}\bigr)\pmb{\ell}^{(t-1)}_u\mathbf{W}^{(t)}\Bigr), \label{GNN:Kipf}
\end{equation}
where $\mathbf{W}$ is a learnable weight matrix in $\mathbb{Q}^{s\times s}$ and $\sigma$ is a non-linear activation function.
We can  regarded this architecture again as an MPNN. Indeed, it suffices to define for each $\mathbf{x}$ and $\mathbf{y}$ in $\mathbb{Q}^s$, each $v\in V$ and $u\in N_G(v)$, and each $t\geq 1$:
\begin{align*}
\textsc{Msg}^{(t)}\bigl(\mathbf{x},\mathbf{y},v,u)&:=
\frac{1}{d_v}\bigl(\frac{1}{1+d_v}\bigr)\mathbf{x}\mathbf{W}^{(t)}+
\bigl(\frac{1}{\sqrt{1+d_v}}\bigr)\bigl(\frac{1}{\sqrt{1+d_u}}\bigr)\mathbf{y}\mathbf{W}_2^{(t)}
\intertext{and} \textsc{Upd}^{(t)}(\mathbf{x},\mathbf{y})&:=\sigma(\mathbf{y}).
\end{align*}
We remark that the initial factor $1/d_v$ in the message functions is introduced for renormalisation purposes.
We indeed observe that the message functions depend on $\pmb{\ell}^{(t-1)}_v$, 
$\pmb{\ell}^{(t-1)}_u$ and the degrees $d_v$ and $d_u$ of the vertices $v$ and $u$, respectively.\qed
\end{example}


\paragraph{Turing complete MPNNs.}
 As a final example, we show how an MPNN can compute any computable function when the message functions can use knowledge of which vertices are under consideration. 

\begin{example}[Turing complete MPNN]\normalfont
	Let $\langle G,\pmb{\nu}\rangle$ be a connected labeled graph with $\pmb{\nu}:V\to\mathbb{Q}^t$ and let $\mathbf{f}_G:V\to\mathbb{Q}$ be a labeling 
	computed by a computable function on input 	$\langle G,\pmb{\nu}\rangle$. To compute $\mathbf{f}_G$ by means of an MPNN we first use a number  of $d$ rounds  to ensure that $\pmb{\ell}^{(d)}_v$ represents $\langle G,\pmb{\nu}\rangle$. In other words, after $d$ rounds,  each vertex has the entire input labeled graph as label. An update function simulating $\mathbf{f}_G$ then suffices to computer $\mathbf{f}_G$. We assume that $V=[n]$ and use $i\in[n]$ to denoted the $i$th vertex in $V$.
	
To encode $\langle G,\pmb{\nu}\rangle$ in the labels of vertices we use labels in $\mathbf{Q}^s$ with 
$s=t+1+n(t+2)+n^2$. The aim it to ensure that for each vertex $i$, the final label $\pmb{\ell}_i^{(d)}$ is of the form
\begin{equation}
(\underbrace{\vphantom{f}\pmb{\nu}_i,1}_{\text{initial label}},\underbrace{\vphantom{f}1,\pmb{\nu}_1,1}_{\text{vertex $i$}},\ldots,\underbrace{\vphantom{f}n,\pmb{\nu}_n,1}_{\text{vertex $n$}},\textsc{Vect}(\mathbf{A}_G)), \label{eq:graphinlabel}
\end{equation}
where the first $t$ positions hold the initial label $\pmb{\nu}_i$,
the $(t+1)$st position holds a counter (in order to remember how many times
this information has been passed on), for each $i\in[n]$, position
$i(t+2)$ holds the vertex id ($i$), positions $i(t+2)+1$ up $(i+1)(t+2)-2$
hold label $\pmb{\nu}_i$, and position $(i+1)(t+2)-1$ holds again a counter.
The remaining $n^2$ positions hold a vectorised representation of the adjacency matrix $\mathbf{A}_G$ obtained by concatenating its rows denoted by $\textsc{Vect}(\mathbf{A}_G)$. Clearly, one can extract 
$\langle G,\pmb{\nu}\rangle$ from the label~(\ref{eq:graphinlabel}). We now show how this label can be obtained.

For convenience, we initially extend $\pmb{\nu}$
to a labeling $V:\to\mathbb{Q}^s$ by padding each $\pmb{\nu}_i$ with $s-t$ zeroes. We abuse notation and also refer to this initial labeling by $\pmb{\nu}$.

For each round $t$, vertices $i$ and $j\in N_G(i)$, and $\mathbf{x},\mathbf{y}\in\mathbb{Q}^s$, we define
$
\textsc{Msg}^{(t)}(\mathbf{x},\mathbf{y},i,j)
$
as the function which attaches the initial labels to vertices $i$ and $j$ and adds edges $(i,j)$ and $(j,i)$ to the vectorised encoding of the adjacency matrix.
More specifically, $
\textsc{Msg}^{(t)}(\mathbf{x},\mathbf{y},i,j)
$ returns:
$$
\mathbf{x}+(\mathbf{0}_{1\times t},0,\ldots,\underbrace{i,\mathbf{x}_{1:t},1}_{\text{vertex $i$}},\ldots, \underbrace{j,\mathbf{y}_{1:t},1}_{\text{vertex $j$}},\ldots,\underbrace{1}_{\text{entry $(i,j)$}},\ldots,\underbrace{1}_{\text{entry $(j,i)$}},\ldots),
$$
where entry $(i,j)$ concerns position $t+1+n(t+2)+(i-1)n+j$ and the entry $(j,i)$
concerns position $t+1+n(t+2)+(j-1)n+i$. We then define
$\textsc{Upd}^{(t)}(\mathbf{x},\mathbf{y})$ as the
function which returns on input 
$$
(\mathbf{y}_0,c_0,i_1,\mathbf{y}_1,c_1,\ldots,i_n,\mathbf{y}_n,c_n, \underbrace{c_{11},\ldots,c_{nn}}_{\text{last $n^2$ slots}})
$$
the vector
$$
(\frac{1}{c_0}\mathbf{y}_0,\delta(c_0),i_1,\frac{1}{c_1}\mathbf{y}_1,\delta(c_1),\ldots,
i_n,\frac{1}{c_n}\mathbf{y}_n,\delta(c_n),\delta(c_{11}),\ldots,\delta(c_{nn})),
$$
where we assume that $\frac{1}{c}=0$ when $c=0$ and $\delta(x)=1$ if $x\neq 0$ and $\delta(x)=0$ otherwise. It is easily verified that after $d$ rounds, where $d$ is the diameter of $G$, each vertex will carry
the desired label~(\ref{eq:graphinlabel}). It now suffices incorporate $\mathbf{f}_G$ in the last update function to ensure that $\pmb{\ell}_i^{(t)}:=(\mathbf{f}_G(i),\mathbf{0})\in\mathbb{Q}^s$.\qed
\end{example}
We remark that the Turing-completeness of MPNNs, whose message functions can access  the vertices themselves, was recently shown by~\cite{Loukas2019}  using close connections with the LOCAL model for distributed graph computations of~\cite{Angluin}, which is known to be complete. The previous example provides a direct proof.

In the next section we recall results concerning the distinguishing power of anonymous MPNNs and establish an upper bound on the distinguishing power of degree-aware MPNNs.

\subsection{On the choice of formalism}
We use a different formalisation of MPNNs than given in~\cite{GilmerSRVD17}. More specifically, we explicitly allow a dependency of the message functions on $v$ and $u\in N_G(v)$. The reason is that there is a certain ambiguity in the formalisation in ~\cite{GilmerSRVD17} on what precisely the message functions can depend on. More specifically, only a dependence on  $\pmb{\ell}_v^{(t-1)}$ and $\pmb{\ell}_u^{(t-1)}$
is specified. In contrast, the examples given in~\cite{GilmerSRVD17} use more information, such as the degree of vertices. Another difference is that the MPNNs in ~\cite{GilmerSRVD17} work on graphs that carry both vertex and edge labels. We ignore edge labelings in this paper but all our upper bound results carry over to this more general setting. Indeed, it suffices to use the extension of the Weisfeiler-Lehman algorithm for this more general class of graphs~\cite{Jaume2019}. Our formalisation also differs from the one given by~\cite{Loukas2019} in that we only exchange messages from $u\in N_G(v)$
to $v$. In~\cite{Loukas2019}, every vertex can also send itself a message. We provide this functionality by parametrizing the update functions with the current label of the vertex itself, just as in~\cite{GilmerSRVD17}. One can verify that both formalisations are equivalent.

%!TEX root =main.tex
\section{Comparing the distinguishing power of classes of MPNNs}\label{subsec:compare}
The distinguishing power of MPNNs relates to their ability to distinguish vertices based on the labellings that MPNNs compute. We are interested in comparing the distinguishing power of classes of MPNNs. In this section we formally define how to compare classes of MPNNs.

For a given labelled graph $\langle G,\pmb{\nu}\rangle$ and MPNN $M$, we denote by 
$\pmb{\ell}_M^{(t)}$ the vertex labelling computed by $M$ after $t$ rounds. We will fix the input graph in what follows, so we do not need to include the dependency on the graph in the notation of labellings.
% The \textit{distinguishing power} of an MPNN $M$ relates to its ability to distinguish vertices by means of the labelings $\pmb{\ell}_M^{(t)}$, for $t\geq 0$. 

\begin{definition}\label{def:mpnnweak}\normalfont
Consider two MPNNs $M_1$ and $M_2$ and let $\pmb{\ell}_{M_1}^{(t)}$ and $\pmb{\ell}_{M_2}^{(t)}$  be their corresponding labellings on an input graph $\langle G,\pmb{\nu}\rangle$ obtained after $t$ rounds of computation. Then,
$M_1$ is said to be \textit{weaker} than $M_2$, denoted by $M_1\preceq M_2$, if $M_1$ cannot distinguish more vertices  than $M_2$ in every round of computation. More formally, $M_1\preceq M_2$ if $\pmb{\ell}_{M_2}^{(t)}\sqsubseteq
\pmb{\ell}_{M_1}^{(t)}$ for every $t\geq 0$. We also say that $M_2$ is \textit{stronger} than $M_1$. \qed
\end{definition}
We can lift this notion to classes  $\architecture_1$ and $\architecture_2$ of MPNNs in a standard way. 

\begin{definition}\label{def:classesweak}\normalfont
Consider two classes $\architecture_1$ and $\architecture_2$ of MPNNs.
Then, $\architecture_1$ is said to be \textit{weaker} than $\architecture_2$, denoted by 
$\architecture_1\sqsubseteq \architecture_2$, if for every $M_1\in \architecture_1$
there exists an $M_2\in\architecture_2$ which is stronger than $M_1$. \qed
\end{definition}
% We observe that when $\architecture_1$ and $\architecture_2$ consist of single MPNNs $M_1$ and $M_2$, respectively, then $\architecture_1\sqsubseteq \architecture_2$ if and only $M_1\preceq M_2$. We further remark that whereas $\preceq$ defines a partial order on MPNNs,
% $\sqsubseteq$ define a pre-order on class of MPNNs.
% In particular, $\architecture_1\sqsubseteq \architecture_2$ and $\architecture_2\sqsubseteq \architecture_1$ does not necessarily implies that
% $\architecture_1=\architecture_2$

We will also need a generalisation of the previous definitions in which the MPNNs involved may run a different number of rounds.  More specifically, we consider the case when the distinguishing power of $M_1$ is weaker than that of $M_2$, provided that $M_2$ can run for a possibly different number of rounds. This is formalised as follows.

\begin{definition}\normalfont
Consider two MPNNs $M_1$ and $M_2$ and let $\pmb{\ell}_{M_1}^{(t)}$ and $\pmb{\ell}_{M_2}^{(t)}$  be their corresponding labellings on an input graph $\langle G,\pmb{\nu}\rangle$ obtained after $t$ rounds of computation. Let $g:\mathbb{N}\to \mathbb{N}$ be a function. We say that $M_1$ is \textit{$g$-weaker} than $M_2$, denoted by $M_1\preceq_{g} M_2$, if 
$\pmb{\ell}_{M_2}^{g(t)}\sqsubseteq
\pmb{\ell}_{M_1}^{(t)}$ for every $t\geq 0$.\qed
\end{definition}

Only the following special cases of this definition, depending on extra information on the function $g:\mathbb{N}\to\mathbb{N}$, will be relevant to this paper:
\begin{itemize}
    \item $g(t)=t$, for all $t\geq 0$. This case corresponds to Definition~\ref{def:mpnnweak}. If $M_1\preceq_{g} M_2$, then we simply say that $M_1$ is weaker than $M_2$, as before.
    \item $g(t)\leq t+c$, for all $t\geq 0$ and some constant $c$. If $M_1\preceq_{g} M_2$, then we say that $M_1$ is weaker than $M_2$ \textit{but may be $c$ steps ahead};
    \item $g(t)\leq c't+c$, for all $t\geq 0$ and some constants $c'$ and $c$. If $M_1\preceq_{g} M_2$, then we say that $M_1$ is weaker than $M_2$ \textit{possibly up to a linear factor of $c'$ and $c$ steps ahead}.
\end{itemize}

One can again lift these definitions to classes of MPNNs, just like in Definition~\ref{def:classesweak}.

We finally define when two classes of MPNNs are equally strong.
\begin{definition}\normalfont
Consider two classes $\architecture_1$ and $\architecture_2$ of MPNNs. We say that 
$\architecture_1$ and $\architecture_2$ are \textit{equally strong}, denoted by $\architecture_1\equiv \architecture_2$, if 
both  $\architecture_1\sqsubseteq \architecture_2$ 
and  $\architecture_2\sqsubseteq \architecture_1$ hold.\qed\end{definition}

\paragraph{Remarks} The previous definitions all assume the input labeled graph to be fixed. One could consider an alternative definition where the distinguishing power is compared with regards to all
input graphs. For example, an MPNN $M_1$ is said to be \textit{uniformly weaker} than an MPNN $M_2$ if $M_1\preceq M_2$ for \textit{any} input graph $\langle G,\pmb{\nu}\rangle$. A similar notion can be defined for classes of MPNNs. Being uniformly weaker clearly implies being weaker. The converse may not hold, however. We point out in the paper which results carry over for this stronger notion.


%!TEX root =main.tex
\section{The distinguishing power of anonymous MPNNs}
We recall from Section~\ref{sec:MPNNs} that anonymous MPNNs are MPNNs whose message functions
only depend on the previous labels of the vertices involved. The distinguishing power of anonymous MPNNs (or aMPNNs, for short) is well understood.
Indeed, as we will shortly see, it follows from two independent works~\cite{xhlj19,grohewl} that the distinguishing power of aMPNNs can be linked to the distinguishing power of the WL algorithm. Apart from rephrasing known results in terms of aMPNNs, we provide two simplifications of the results of~\cite{grohewl}.

\subsection{General aMPNNs}
Let $( G,\pmb{\nu})$ be a labelled graph.
We denote by $\architectureWL$ the `class' of aMPNNs consisting of the single aMPNN $M_{\textsl{WL}}$ originating from the WL algorithm (see Example~\ref{ex:WL}). We denote the class of anonymous MPNNs by $\architectureano$.

\begin{theorem}[Based on~\cite{xhlj19,grohewl}]\label{thm:eqstrongWL}
The classes $\architectureano$ and  $\architectureWL$ are equally strong.
\end{theorem}


\begin{proof}
We first argue that $\architectureano$ is weaker than $\architectureWL$. The proof is a trivial adaptation of the proofs of Lemma 2 in~\cite{xhlj19} and Theorem 5 in~\cite{grohewl}. We show, by induction on the number of rounds of computation, that  $\pmb{\ell}_{M_{\textsl{WL}}}^{(t)}\sqsubseteq \pmb{\ell}_M^{(t)}$ for all $M \in \architectureano$ and every $t\geq 0$.

Clearly, this holds for $t=0$ since $\pmb{\ell}_{M_{\textsl{WL}}}^{(0)}=\pmb{\ell}_M^{(0)}:=\pmb{\nu}$, by definition.
We assume next that the induction hypothesis holds up to round $t-1$ and consider round $t$.
Let $v$ and $w$ be two vertices such that 
$(\pmb{\ell}_{M_{\textsl{WL}}}^{(t)})_v=(\pmb{\ell}_{M_{\textsl{WL}}}^{(t)})_w$ holds.
This implies, by the definition of $M_{\textsl{WL}}$, that $(\pmb{\ell}_{M_{\textsl{WL}}}^{(t-1)})_v=(\pmb{\ell}_{M_{\textsl{WL}}}^{(t-1)})_w$  and
$$
\ldbl (\pmb{\ell}_{M_{\textsl{WL}}}^{(t-1)})_u\mid u\in N_G(v) \rdbl=
\ldbl (\pmb{\ell}_{M_{\textsl{WL}}}^{(t-1)})_u\mid u\in N_G(w) \rdbl.
$$
By the induction hypothesis, this implies that 
$(\pmb{\ell}_{M}^{(t-1)})_v=(\pmb{\ell}_{M}^{(t-1)})_w$  and
$$
\ldbl (\pmb{\ell}_{M}^{(t-1)})_u\mid u\in N_G(v) \rdbl=
\ldbl (\pmb{\ell}_{M}^{(t-1)})_u\mid u\in N_G(w) \rdbl.
$$
As a consequence, for every vertex $u\in N_G(v)$ there exists a (unique) vertex $u'\in N_G(w)$ such that $(\pmb{\ell}_{M}^{(t-1)})_u=(\pmb{\ell}_{M}^{(t-1)})_{u'}$. Hence,
$$
\textsc{Upd}^{(t)}\left((\pmb{\ell}_{M}^{(t-1)})_v,(\pmb{\ell}_{M}^{(t-1)})_u\right)=
\textsc{Upd}^{(t)}\left((\pmb{\ell}_{M}^{(t-1)})_w,(\pmb{\ell}_{M}^{(t-1)})_{u'}\right).
$$
Furthermore, this mapping between elements of $N_G(v)$ and $N_G(w)$ is a bijection so that we also have:
$$
\mathbf{m}^{(t)}_v=\sum_{u\in N_G(v)}\textsc{Upd}^{(t)}\left((\pmb{\ell}_{M}^{(t-1)})_v,(\pmb{\ell}_{M}^{(t-1)})_u\right)=\sum_{u'\in N_G(w)}\textsc{Upd}^{(t)}\left((\pmb{\ell}_{M}^{(t-1)})_w,(\pmb{\ell}_{M}^{(t-1)})_{u'}\right)=\mathbf{m}^{(t)}_w.
$$
We may thus conclude that $$(\pmb{\ell}_{M}^{(t)})_v=\textsc{Upd}^{(t)}\left((\pmb{\ell}_{M}^{(t-1)})_v,\mathbf{m}^{(t)}_v\right)=\textsc{Upd}^{(t)}\left((\pmb{\ell}_{M}^{(t-1)})_w,\mathbf{m}^{(t)}_w\right)=(\pmb{\ell}_{M}^{(t)})_w,
$$
as desired.
%
% This was shown in \cite{xhlj19} and \cite{grohewl} for  graph neural networks which, in round $t \geq 1$, compute for each vertex $v$ a label $\pmb{\ell}^{(t)}_{v}$, as follows:
% \begin{equation}
% \pmb{\ell}^{(t)}_{v}:=
% f_{\textsl{comb}}^{(t)}\Bigl(
% \pmb{\ell}_{v}^{(t-1)},f_{\textsl{aggr}}^{(t)}\bigl(\ldbl \pmb{\ell}^{(t-1)}_{u} \mid u \in N_G(v) \rdbl\bigr)
% \Bigr), \label{eq:combaggr}
% \end{equation}
% where $f_{\textsl{comb}}^{(t)}$ and  $f_{\textsl{agg}}^{(t)}$ are general (computable) combination and aggregation functions. Furthermore, $\pmb{\ell}^{(0)}:=\pmb{\nu}$, just as before.
% Clearly, any aMPNN can be written in the form ~(\ref{eq:combaggr}). Conversely, every graph neural network of the form~(\ref{eq:combaggr}) is readily cast as an aMPNN.
%
% Indeed,
% it suffices to observe, just as we did in Example~\ref{ex:WL}, that the aggregation functions $f_{\textsl{aggr}}^{(t)}\bigl(\ldbl \pmb{\ell}^{(t-1)}_{u} \mid u \in N_G(v) \rdbl\bigr)$ can be written in the form $g^{(t)}\bigl(\sum_{u\in N_G(v)} h^{(t)}(\pmb{\ell}^{(t-1)}_{u})\bigr)$. This was already observed in Lemma 5 in~\cite{xhlj19}, based on Theorem 2 in~\cite{ZaheerKRPSS17}.
%
% Suppose that
% $\pmb{\nu}:V\to\mathbb{A}^s$. It now suffices to define for every $\mathbf{x}$ and $\mathbf{y}$ in $\mathbb{A}^s$, every $v\in V$ and $u\in N_G(u)$, and every $t\geq 1$:
% $$
% \textsc{Msg}^{(t)}(\mathbf{x},\mathbf{y},v,u):=h^{(t)}(\mathbf{y}) \text{ and } \textsc{Upd}^{(t)}(\mathbf{x},\mathbf{y}):=f_{\textsl{comb}}^{(t)}\left(\mathbf{x},g^{(t)}\left(\mathbf{y}\right)\right).
% $$
% This is clearly an aMPNN which computes the same labelling as~(\ref{eq:combaggr}). We remark that
% Lemma 5 in~\cite{xhlj19} crucially relies on the assumption that labels come from a countable domain
% and that the size of multisets is bounded. These conditions are satisfied because $\mathbb{A}^s$ is a countable set and there are at most $|V|$ neighbours (elements in the multiset) for every vertex in $V$.

It remains to show that $\architectureWL$ is weaker than $\architectureano$. For this, it suffices to recall that $M_{\textsl{WL}}$ is an element of $\architectureano$.
\end{proof}

We remark that the proofs in \cite{xhlj19} and \cite{grohewl} relate to  graph neural networks which, in round $t\geq 1$, compute for each vertex $v$ a label $\pmb{\ell}^{(t)}_{v}$, as follows:
\begin{equation}
\pmb{\ell}^{(t)}_{v}:=
f_{\textsl{comb}}^{(t)}\left(
\pmb{\ell}_{v}^{(t-1)},f_{\textsl{aggr}}^{(t)}\left(\ldbl \pmb{\ell}^{(t-1)}_{u} \mid u \in N_G(v) \rdbl\right)
\right), \label{eq:combaggr}
\end{equation}
where $f_{\textsl{comb}}^{(t)}$ and  $f_{\textsl{aggr}}^{(t)}$ are general (computable) combination and aggregation functions which we assume to assign labels in $\mathbb{A}^s$. Furthermore, $\pmb{\ell}^{(0)}:=\pmb{\nu}$, just as before. Every graph neural network of the form~(\ref{eq:combaggr}) is readily cast as an aMPNN. Indeed,
it suffices to observe, just as we did in Example~\ref{ex:WL}, that the aggregation functions $f_{\textsl{aggr}}^{(t)}\bigl(\ldbl \pmb{\ell}^{(t-1)}_{u} \mid u \in N_G(v) \rdbl\bigr)$ can be written in the form $g^{(t)}\bigl(\sum_{u\in N_G(v)} h^{(t)}(\pmb{\ell}^{(t-1)}_{u})\bigr)$. 
%This was already observed in Lemma 5 in~\cite{xhlj19}, based on Theorem 2 in~\cite{ZaheerKRPSS17}.

Suppose that
$\pmb{\nu}:V\to\mathbb{A}^s$. It now suffices to define for every $\mathbf{x}$ and $\mathbf{y}$ in $\mathbb{A}^s$, every $v\in V$ and $u\in N_G(u)$, and every $t\geq 1$:
\begin{equation}
\textsc{Msg}^{(t)}(\mathbf{x},\mathbf{y},v,u):=h^{(t)}(\mathbf{y}) \text{ and } \textsc{Upd}^{(t)}(\mathbf{x},\mathbf{y}):=f_{\textsl{comb}}^{(t)}\left(\mathbf{x},g^{(t)}\left(\mathbf{y}\right)\right).\label{eq:combaggrtoaMPNN}
\end{equation}
This is clearly an aMPNN which computes the same labelling as~(\ref{eq:combaggr}). We remark that
Lemma 5 in~\cite{xhlj19} crucially relies on the assumption that labels come from a countable domain
and that the size of multisets is bounded. These conditions are satisfied because $\mathbb{A}^s$ is a countable set and there are at most $|V|$ neighbours (elements in the multiset) for every vertex in $V$.

The aMPNNs that we consider in this paper are slightly more general than those defined by
(\ref{eq:combaggrtoaMPNN}). Indeed, we consider message functions that can also depend on the previous labelling $\pmb{\ell}_v^{(t-1)}$. In contrast, the message functions in~(\ref{eq:combaggrtoaMPNN}) only depend on $\mathbf{y}$, which corresponds to the previous labelling of neighbours $u\in N_G(v)$. Let $\architecture{}^{-}_{\textsl{anon}}$ denote the class of aMPNNs whose message functions only
depend on the previous labels of neighbours. It now suffices to observe that
 $M_{\textsl{WL}}\in \architecture{}^{-}_{\textsl{anon}}$ to infer,  combined with Theorem~\ref{thm:eqstrongWL}, that:
 \begin{corollary}
	 The classes $\architecture{}^{-}_{\textsl{anon}}$, $\architectureano$ and $\architectureWL$ are all equally strong.
 \end{corollary}
We observe, however, that this does not imply that for every aMPNN $M$ in $\architectureano$ there exists an aMPNN $M'$ in $\architecture{}^{-}_{\textsl{anon}}$
such that $\pmb{\ell}_{M}^{(t)}\equiv \pmb{\ell}_{M'}^{(t)}$ for all $t\geq 0$.
Indeed, the corollary implies that for every $M$ in $\architectureano$ there exists an aMPNN $M'$ in $\architecture{}^{-}_{\textsl{anon}}$ such that $M'\preceq M$, and there exists an $M''$ in $\architectureano$, possibly different from $M$, such that $M''\preceq M'$. In fact, the aMPNN $M''$ in this case is $M_{\textsl{WL}}$.
% What it does imply, however, is that there exists an aMPNN $M$
% in $\architecture{}^{-}_{\textsl{anon}}$ such that $M\equiv M_{\textsl{WL}}$.
\openprob{I actually don't know what is the precise relationship between the aMPNNs coming from combine/aggregate and our aMPNNs.}
\floris{Optional: say something about universal version of equally strong...}
\todo{G: read up until here}
\subsection{GNN-based aMPNNs}
More practical examples of aMPNNs, related to the graph neural networks, were considered in~\cite{grohewl}.
In that paper, the authors consider graph neural networks of the form
$$
\mathbf{L}^{(t)}:=\sigma\left(\mathbf{L}^{(t-1)}\mathbf{W}_1^{(t)}+\mathbf{A}_G\mathbf{L}^{(t-1)}\mathbf{W}_2^{(t)}+\mathbf{B}^{(t)}\right), $$
which we already described in Example~\ref{ex:GNN}. We also know from that example that such graph neural networks correspond to aMPNNs.
Let us denote by $\architecture_{\textsl{GNN}}$ the class of aMPNNs with message and update functions of the form
\begin{equation}\textsc{Msg}^{(t)}\bigl(\mathbf{x},\mathbf{y},v,u):=\mathbf{y}\mathbf{W}_2^{(t)}
\text{ and } 
\textsc{Upd}^{(t)}(\mathbf{x},\mathbf{y}):=\sigma\left(\mathbf{x}\mathbf{W}_1^{(t)}+\mathbf{y} + \mathbf{b}^{(t)}\right) \label{eq:MPNN-GNN}
\end{equation}
for any $\mathbf{W}_1^{(t)}\in\mathbb{A}^{s\times s}$,$\mathbf{W}_2^{(t)}\in\mathbb{A}^{s\times s}$, and $\mathbf{B}^{(t)}\in\mathbb{A}^{n\times s}$ consisting of $n$ copies of a row $\mathbf{b}^{(t)}\in\mathbb{A}^{s}$. 

We start by stating a direct consequence of Theorem~\ref{thm:eqstrongWL}. It follows by observing
that $\architecture_{\textsl{GNN}}$ is sub-class of 
$\architectureano$.
\begin{corollary}
	The class 
$\architecture_{\textsl{GNN}}$ is weaker than $\architectureano$ and is thus also weaker than $\architectureWL$.
\end{corollary}

More challenging is to show that $\architecture_{\textsl{GNN}}$ and $\architectureWL$, and thus also $\architecture_{\textsl{GNN}}$ and $\architectureano$, are equally strong. The following results are known. We denote
by $\architecture_{\textsl{GNN}}^{\textsl{sign}}$
and $\architecture_{\textsl{GNN}}^{\textsl{ReLU}}$ the classes of  aMPNNs in $\architecture_{\textsl{GNN}}$ whose update functions, as defined in~(\ref{eq:MPNN-GNN}), use the sign and ReLU as activation function $\sigma$, respectively.

\begin{theorem}[\cite{grohewl}] \label{thm:grohe_lower}
(i)~The classes $\architecture_{\textsl{GNN}}^{\textsl{sign}}$ and  $\architectureWL$ are equally strong. (ii)~The class 
$\architecture_{\textsl{GNN}}^{\textsl{ReLU}}$ is weaker than $\architectureWL$, and
$\architectureWL$ is weaker than $\architecture_{\textsl{GNN}}^{\textsl{ReLU}}$, with a factor of two.
\end{theorem}

% In the proof of this Theorem~\cite{grohewl}, an explicit construction is given of an aMPNN $M_1$ in $\architecture_{\textsl{GNN}}^{\textsl{sgn}}$
% and an aMPNN $M_2$ in $\architecture_{\textsl{GNN}}^{\textsl{ReLU}}$
% such that $M_1\preceq M_{\textsl{WL}}$
% and $M_2\preceq_g M_{\textsl{WL}}$ with $g:\mathbb{N}\to \mathbb{N}:t\mapsto 2t$.
The reason for the factor of two in (ii) in this Theorem is due to a simulation of the sign activation function by means of a two-fold application of the ReLu function. We next show that this factor of two can be avoided. As a side effect, we obtain a simpler aMPNN $M$ in $\architecture_{\textsl{GNN}}$, satisfying $M\preceq M_{\textsl{WL}}$, than the one constructed in \cite{grohewl}. The proof strategy is inspired by that of~\cite{grohewl}. Crucial in the proof the notion of \textit{row-independence modulo equality}, which we define next.

\begin{definition}[row-independence modulo equality]\label{def:label2}\normalfont
	A labelling $\pmb{\ell}:V\to\mathbb{A}^s$ is \textit{row-independent modulo equality} if the set of unique labels assigned by $\pmb{\ell}$ are linearly independent. \qed
\end{definition}
In what follows, we always assume that the labelling $\pmb{\nu}$ of $G$ is row-independent modulo equality. One can always ensure this by extending the labels. 
%
% \begin{definition}[a good matrix]\label{def:label3}\normalfont
% A matrix $\mathbf{F}$ is \textit{good relative to an other matrix} $\mathbf{F}'$ if $\mathbf{F}\equiv \mathbf{F}'$ and $\mathbf{F}$ is row-independent modulo equality.\qed
% \end{definition}
\begin{proposition}
The classes $\architecture_{\textsl{GNN}}^{\textsl{ReLU}}$ and  $\architectureWL$ are equally strong.
\end{proposition}
\begin{proof}
We already know that $\architecture_{\textsl{GNN}}^{\textsl{ReLU}}$ is weaker than $\architectureWL$ (Theorem~\ref{thm:grohe_lower}). It remains to show that $\architectureWL$ is weaker than $\architecture_{\textsl{GNN}}^{\textsl{ReLU}}$. That is, given the aMPNN $M_{\textsl{WL}}$, we need to construct an aMPNN $M$ in $\architecture_{\textsl{GNN}}^{\textsl{ReLU}}$ such that 
$\pmb{\ell}_{M}^{(t)}\sqsubseteq \pmb{\ell}_{M_{\textsl{WL}}}^{(t)$, for all $t\geq 0$. We observe that
since $\pmb{\ell}_{M_{\textsl{WL}}}^{(t)}\sqsubseteq \pmb{\ell}_{M}^{(t)}$ for any $M$ in $\architecture_{\textsl{GNN}}^{\textsl{ReLU}}$, this is equivalent to constructing an $M$ such that $\pmb{\ell}_{M}^{(t)}\equiv \pmb{\ell}_{M_{\textsl{WL}}}^{(t)}$. 

The proof is by induction on the number of computation rounds. The MPNN $M$ in 
$\architecture_{\textsl{GNN}}^{\textsl{ReLU}}$ 
that we will construct will use message and update functions of the form:
\begin{equation}\textsc{Msg}^{(t)}\bigl(\mathbf{x},\mathbf{y},v,u):=\mathbf{y}\mathbf{W}^{(t)}
\text{ and } 
\textsc{Upd}^{(t)}(\mathbf{x},\mathbf{y}):=\sigma\left(p\mathbf{x}\mathbf{W}^{(t)}+\mathbf{y} + \mathbf{b}^{(t)}\right) \label{eq:MPNN-GNN-simple}
\end{equation}
for some value $p\in\mathbb{A}$, $0<p<1$, weight matrix $\mathbf{W}^{(t)}\in\mathbb{A}^{s\times s}$, and bias vector $\mathbf{b}^{(t)}\in\mathbb{A}^s$. Note that, in contrast to aMPNNs of the form~(\ref{eq:MPNN-GNN}), we only have one weight matrix per round, instead of two, at the cost of introducing an extra parameter $p\in\mathbb{A}$.  Furthermore, the aMPNN constructed in \cite{grohewl} uses two distinct weight matrices in $\mathbb{A}^{2s\times 2s}$ (we come back to this at the end of this section) whereas our weight matrices are elements of $\mathbb{A}^{s\times s}$.

The induction hypothesis is that $\pmb{\ell}^{(t)}_M\equiv \pmb{\ell}_{M_{\textsl{WL}}}^{(t)}$ and that $\pmb{\ell}^{(t)}_M$ is row-independent modulo equality.

For $t=0$, we have that for any $M\in \architecture_{\textsl{GNN}}^{\textsl{ReLu}}$, $\pmb{\ell}_M^{(0)}=\pmb{\ell}_{M_{\textsl{WL}}}^{(0)}:=\pmb{\nu}$, by definition.
Moreover, $\pmb{\ell}_M^{(0)}$ is row-independent modulo equality because $\pmb{\nu}$ is so, by assumption. 

We next assume that up to round $t-1$, we have found weight matrices and bias vectors for $M$ such that 
$\pmb{\ell}_M^{(t-1)}$ satisfies the induction hypothesis.
We next consider round $t$ and show that we can find a weight matrix $\mathbf{W}^{(t)}\in\mathbb{A}^{s\times s }$ and bias vector $\mathbf{b}^{(t)}\in\mathbb{A}^s$ such that also
$\pmb{\ell}_M^{(t)}$ satisfies the hypothesis.


Let $\mathbf{L}^{(t-1)}\in\mathbb{A}^{n\times s}$ denote the matrix consisting of rows $(\pmb{\ell}_M^{(t-1)})_v$, for $v\in V$.
Moreover, we denote by $\mathsf{uniq}(\mathbf{L}^{(t-1)})$ the $u\times s$-matrix consisting of the $u$ unique rows in $\mathbf{L}^{(t-1)}$. We denote the rows in $\mathsf{uniq}(\mathbf{L}^{(t-1)})$ by $\mathbf{a}_1,\ldots,\mathbf{a}_u\in\mathbb{A}^s$.
By the induction hypothesis, these rows are linearly independent. Following the same argument as in~\cite{grohewl} this implies that there exists an $s\times u$-matrix $\mathbf{U}^{(t)}$ such that $\mathsf{uniq}(\mathbf{L}^{(t-1)})\mathbf{U}^{(t)}=\mathbf{I}_{u\times u}$. 
Let us denote by $\mathbf{e}_1,\ldots,\mathbf{e_u}\in\mathbb{A}^u$ the rows of $\mathbf{I}_{u\times u}$. In other words, in $\mathbf{e}_i$, all entries are zero except for entry $i$ that holds value $1$. 

We consider the following intermediate labelling $\pmb{\mu}^{(t)}:V\to\mathbb{A}^u$ defined by
\begin{equation}
v\mapsto \left((\mathbf{A}+p\mathbf{I})\mathbf{L}^{(t-1)}\mathbf{U}^{(t)}\right)_{v\bullet}.\label{eq:labelmu}
\end{equation}
We know that for every vertex $v$, $(\pmb{\ell}_M^{(t-1)})_v$ corresponds to a unique row $\mathbf{a}_i$ in $\mathsf{uniq}(\mathbf{L}^{(t-1)})$. We denote the index of this row by $\rho(v)$. More specifically, $(\pmb{\ell}_M^{(t-1)})_v=\mathbf{a}_{\rho(v)}$. Let
$N_G(v,i):=\{u \st u\in N_G(v), \rho(v)=i\}$. That is, $N_G(v,i)$ consists of all neighbours $u$ of $v$ which are labelled as $\mathbf{a}_i$ by $\pmb{\ell}_M^{(t-1)}$.
It is now readily verified that the label $\pmb{\mu}^{(t)}_v$ defined in~(\ref{eq:labelmu}) is of the form 
\begin{equation}
\pmb{\mu}^{(t)}_v=\sum_{i=1}^u |N_G(v,i)|\mathbf{e}_i + p\mathbf{e}_{\rho(v)}.  \label{eq:linearcomb}
\end{equation}
We clearly have that $\pmb{\ell}_{M_{\textsl{WL}}}^{(t)}\sqsubseteq\pmb{\mu}^{(t)}$.
\floris{Is this indeed ``clearly''?}
The converse also holds, as is shown in the following lemma.
\begin{lemma}
For any two vertices $v$ and $w$, we have that 
	$\pmb{\mu}^{(t)}_v=\pmb{\mu}^{(t)}_w$ implies 
	$(\pmb{\ell}_{M_{\textsl{WL}}}^{(t)})_v=(\pmb{\ell}_{M_{\textsl{WL}}}^{(t)})_w$.
\end{lemma}
\begin{proof}
We argue by contradiction. Suppose, for the sake of contradiction, that there exists two vertices $v,w\in V$ such that 
	\begin{equation}
		\pmb{\mu}^{(t)}_{v}=\pmb{\mu}^{(t)}_{w} \text{ and } (\pmb{\ell}_{M_{\textsl{WL}}}^{(t)})_v\neq(\pmb{\ell}_{M_{\textsl{WL}}}^{(t)})_w \label{eq:contra}
	\end{equation}
	hold.
We show that this is impossible for any value $p$ satisfying $0<p<1$. (Recall from~(\ref{eq:linearcomb}) that $\pmb{\mu}^{(t)}_v$ depends on $p$.)

We distinguish between the following two cases. If $ (\pmb{\ell}_{M_{\textsl{WL}}}^{(t)})_v\neq(\pmb{\ell}_{M_{\textsl{WL}}}^{(t)})_w$ then either
(i)~$(\pmb{\ell}_{M_{\textsl{WL}}}^{(t-1)})_v\neq(\pmb{\ell}_{M_{\textsl{WL}}}^{(t-1)})_w$; or 
(ii)~$(\pmb{\ell}_{M_{\textsl{WL}}}^{(t-1)})_v=(\pmb{\ell}_{M_{\textsl{WL}}}^{(t-1)})_w$
but
	$$
	\ldbl (\pmb{\ell}_{M_{\textsl{WL}}}^{(t-1)})_u \st u \in N_G(v) \rdbl\neq
	\ldbl(\pmb{\ell}_{M_{\textsl{WL}}}^{(t-1)})_u \st u \in N_G(w) \rdbl.
	$$
% We recall that, by the induction hypothesis, $\pmb{\ell}^{(t-1)}_M$ is row-independent modulo equality. Let us denote by $\mathbf{a}_1,\ldots,\mathbf{a}_d \in \mathbb{A}^s$ the unique labels
% in $\pmb{\ell}^{(t-1)}_M$, which are thus linearly independent.
%
% good for $\pmb{\ell}{}^{(t-1)}$.
% In particular, if we consider the unique row vectors in  $\mathbf{F}^{(t-1)}$, then these are linearly independent. Let us denote the unique row vectors in $\mathbf{F}^{(t-1)}$ by $\mathbf{F}_1,\ldots,\mathbf{F}_s$ for some $s$.

We first consider case (i). In this case, $(\pmb{\ell}_{M_{\textsl{WL}}}^{(t-1)})_v\neq(\pmb{\ell}_{M_{\textsl{WL}}}^{(t-1)})_w$ implies that
	$(\pmb{\ell}^{(t-1)}_M)_{v}\neq (\pmb{\ell}_M^{(t-1)})_w$. After all,
	$\pmb{\ell}^{(t-1)}_M\equiv \pmb{\ell}_{M_{\textsl{WL}}}^{(t-1)}$ by the induction hypothesis. 
	
It now suffices to observe that $\pmb{\mu}^{(t)}_{v}=\pmb{\mu}^{(t)}_{w}$ implies that
the corresponding linear combinations, as described in~(\ref{eq:linearcomb}), satisfy:
$$
\sum_{i=1}^u |N_G(v,i)|\mathbf{e}_i + p\mathbf{e}_{\rho(v)}=
\sum_{i=1}^u |N_G(w,i)|\mathbf{e}_i + p\mathbf{e}_{\rho(w)}.
$$
We can assume, without loss of generality,  that $(\pmb{\ell}^{(t-1)}_M)_{v}=\mathbf{a}_1$ and
	$(\pmb{\ell}^{(t-1)}_M)_{w}=\mathbf{a}_2$. Recall that $\mathbf{a}_1$ and $\mathbf{a}_2$ are two distinct labels.
Then,
\begin{multline*}
\left(|N_G(v,1)|+p-|N_G(w,1)|\right)\mathbf{e}_1+
\left(|N_G(v,2)|-|N_G(w,2)-p|\right)\mathbf{e}_2{}\\+
\sum_{i=3}^u \left(|N_G(v,i)|-|N_G(w,i)|\right)\mathbf{e}_i+ p\mathbf{e}_{\rho(v)}  - p \mathbf{e}_{\rho(w)}=0.
\end{multline*}
Since $\mathbf{e}_1,\ldots,\mathbf{e}_u$ are linearly independent, this implies that $|N_G(v,i)|-|N_G(w,i)|=0$
for all $i=3,\ldots,u$ and $|N_G(v,1)|+p-|N_G(w,1)|=0$
and $|N_G(v,2)|-|N_G(w,2)|-p|=0$. Since $|N_G(v,1)|-|N_G(w,1)|\in\mathbb{Z}$ and $0<p<1$, this is impossible. We may thus conclude that case (i) cannot occur.
	

Suppose next that we are in case (ii). Recall that for case (ii), we have that
$(\pmb{\ell}{}_{M_{\textsl{WL}}}^{(t-1)})_v=(\pmb{\ell}{}_{M_{\textsl{WL}}}^{(t-1)})_w$ and thus also  $(\pmb{\ell}{}_M^{(t-1)})_v=(\pmb{\ell}{}_M^{(t-1)})_w$.
	Using the same notation as above, we may assume that $(\pmb{\ell}{}_M^{(t-1)})_v=(\pmb{\ell}{}_M^{(t-1)})_w=\mathbf{a}_1$. In case (ii), however, we have that
	$
	\ldbl (\pmb{\ell}{}_{M_{\textsl{WL}}}^{(t-1)})_{u} \st u \in N_G(v) \rdbl\neq
	\ldbl (\pmb{\ell}{}_{M_{\textsl{WL}}}^{(t-1)})_{u} \st u \in N_G(w) \rdbl
	$ and thus also 
	$
	\ldbl (\pmb{\ell}{}_{M}^{(t-1)})_{u} \st u \in N_G(v) \rdbl\neq
	\ldbl (\pmb{\ell}{}_{M}^{(t-1)})_{u} \st u \in N_G(w) \rdbl
	$.
	That is, there must exist a label assigned by $\pmb{\ell}{}_M^{(t-1)}$ that does not occur the same number of times in the neighbourhoods of $v$ and $w$, respectively. Suppose that this label is $\mathbf{a}_2$. The case when this label is $\mathbf{a}_1$ can be treated similarly. 
	It now suffices to  observe that $\pmb{\mu}^{(t)}_{v}=\pmb{\mu}^{(t)}_{w}$ implies that
the corresponding linear combinations, as described in~(\ref{eq:linearcomb}), satisfy:
$$
\left(|N_G(v,1)|+p\right)\mathbf{e}_1 +|N_G(v,2)|\mathbf{e}_2+\sum_{i=3}^u |N_G(v,i)|\mathbf{e}_i=
\left(|N_G(w,1)|+p\right)\mathbf{e}_1 +|N_G(w,2)|\mathbf{e}_2+\sum_{i=3}^u |N_G(w,i)|\mathbf{e}_i.
$$
Using a similar argument as before, based on the linear independence of $\mathbf{e}_1,\ldots,\mathbf{e}_u$,
we can infer that $|N_G(v,2)|=|N_G(w,2)|$. We note,
however, that $\mathbf{a}_2$ appeared a different number
of times among the neighbours of $v$ and $w$. Hence, also case (ii) is ruled out and our assumption~(\ref{eq:contra})
is invalid. This implies $\pmb{\mu}^{(t)}\sqsubseteq\pmb{\ell}_{M_{\textsl{WL}}}^{(t)}$, as desired. This concludes the proof of the Lemma. 
\end{proof}


% We show this in two steps.
% \begin{description}
% 	\item [\textit{Step 1.}] First, we show that the intermediate labelling $\pmb{\mu}_M^{(t)}:V\to\mathbb{A}^s$, defined by
% \begin{equation}\pmb{\mu}_v^{(t)}:=p^{(t)}(\pmb{\ell}_M^{(t-1)})_v+\sum_{u\in N_G(v)}(\pmb{\ell}_M^{(t-1)})_u, \label{eq:mu}
% \end{equation}
% for every $v\in V$, satisfies conditions (a), (b) and (c). 
% \item  [\textit{Step 2.}]
% Second, we show the existence of a weight matrix $\mathbf{W}^{(t)}$ and a bias vector
% $\mathbf{b}^{(t)}$ such that the labelling defined by
% $$
% v\mapsto \sigma\left((\pmb{\mu}_M^{(t)})_v\mathbf{W}^{(t)}+ \mathbf{b}^{(t)}\right),
% $$
% which corresponds to $(\pmb{\ell}_M^{(t)})_v$, satisfies the  three desired conditions.
% \end{description}
% It is in the first step that we need that $\pmb{\ell}_M^{(t-1)}$ is row-independent modulo equality. We observe that we only need to show that $\pmb{\mu}_M^{(t)}\sqsubseteq \pmb{\ell}_{M_{\textsl{WL}}}^{(t)}$, to conclude that $\pmb{\mu}_M^{(t)}\equiv \pmb{\ell}_{M_{\textsl{WL}}}^{(t)}$. Of course, we also need to verify conditions (b) and (c).

% \smallskip
% \noindent
% \textbf{Step 1.\,}The labelling \textit{$\pmb{\mu}_M^{(t)}$ satisfies the induction hypotheses.}
% We first verify condition (a), i.e.,
% we show that $\pmb{\mu}_M^{(t)}\sqsubseteq \pmb{\ell}_{M_{\textsl{WL}}}^{(t)}$. 
 
% In fact, we may also conclude that $\pmb{\mu}_M^{(t)}$ is row-independent modulo equality. Indeed, if two distinct labels would be dependent, then one can again infer that $\mathbf{a}_1,\ldots,\mathbf{a}_d$ must be linearly dependent.
 
% \smallskip
% \noindent
% \textbf{Step 2.\, }\textit{Construct weight matrix and bias vector.}
% Let $\mathbf{M}^{(t)}\in\mathbb{A}^{n\times s}$ the matrix consisting of rows $(\pmb{\mu}_M^{(t)})_v$, for $v\in V$.
% Moreover, we denote by $\mathsf{uniq}(\mathbf{M}^{(t)})$ the $u\times s$-matrix consisting of the unique rows in $\mathbf{M}$. We know from step 1 that
% $\mathsf{uniq}(\mathbf{M}^{(t)})$ consist of $u$ linear independent labels in $\mathbb{A}^s$. Following the same argument as in~\cite{grohewl} this implies that there exists a $s\times u$-matrix $\mathbf{U}^{(t)}$ such that $\mathsf{uniq}(\mathbf{M}^{(t)})\mathbf{U}^{(t-1)}=\mathbf{I}_{u\times u}$. 



% By induction $\mathbf{F}^{(t-1)}\equiv\hat{\pmb{\ell}}{}^{(t-1)}$. Let $\Sigma^{(t-1)}$ be the set of  labels assigned by $\hat{\pmb{\ell}}{}^{(t-1)}$ to vertices $v\in V$. 
% For any $v\in V$ and $c\in\Sigma^{(t-1)}$, we denote by $\mathbf{F}^{(t-1)}_{v\bullet}\sim c$ that
% $\hat{\pmb{\ell}}{}^{(t)}_v=c$.
% Then for each $v\in V$ and $c\in \Sigma^{(t-1)}$ we have:
% \begin{align*}
% (\mathbf{A}\mathbf{F}^{(t-1)}\mathbf{M}^{(t-1)})_{vc}&=|\{u\in N_G(v)\mid \mathbf{F}^{(t-1)}_{u\bullet}\sim c\}|.
% \intertext{Furthermore,} 
% p\mathbf{I}(\mathbf{F}^{(t-1)}\mathbf{M}^{(t-1)})_{vc}&=p\delta_{vc},
% \end{align*}
% with $\delta_{vc}=1$ if $\mathbf{F}^{(t-1)}_{v\bullet}\sim c$ and $\delta_{vc}=0$ otherwise.


From here on, we follow again closely the proof strategy
of~\cite{grohewl}. More specifically, we re-establish Lemma 9 from~\cite{grohewl}, which concerned the sign activation, for the ReLU function.
% From the definition of $\pmb{\mu}$ it follows that $\mathbf{M}$ (and thus also $\mathsf{uniq}(\mathbf{M})$) does not contain rows consisting entirely out of zero. Furthermore, 
  
\begin{lemma}\label{lem:ReLUlemma9}
  Let  $\mathbf{C}\in \mathbb{A}^{u\times s}$ be a matrix in which 
  all entries are non-negative, all rows are pairwise disjoint and such that no row consists entirely
  out of zeroes\footnote{Compared to Lemma 9 in from~\cite{grohewl},
 we additionally require non-zero rows.}.
%  \footnote{I believe that this can be
%  guaranteed in 1-WL}).\todo{G: with our extended features we actually guarantee this for free by adding the 1 column; also, t as dimension is a bad choice\ldots}
  Then there exists a matrix $\mathbf{X}\in\mathbb{A}^{s\times s}$ and a constant $m\in\mathbb{A}$
  such that $\text{\normalfont ReLU}(\mathbf{CX}-m\mathbf{J})$ is a
  non-singular matrix in $\mathbb{A}^{u\times u}$.
\end{lemma}
\begin{proof}
Let $C$ be the maximal entry in $\mathbf{C}$ and consider the column vector $\mathbf{z}=(1,C,C^2,\ldots,C^{s-1})^{\textsc{t}}\in\mathbb{A}^{s\times 1}$.
Then each entry in $\mathbf{c}=\mathbf{C}\mathbf{z}\in\mathbb{A}^{u\times 1}$ is positive and they are all pairwise distinct. 
Let $\mathbf{P}$ be a permutation matrix in $\Rb^{u\times u}$ such that $\mathbf{c}'=\mathbf{P}\mathbf{c}$ is such that  $\mathbf{c}'=(c_1',c_2',\ldots,c_u')^{\textsc{	t}}\in\mathbb{A}^{u\times 1}$ with $c_1'> c_2'>\cdots > c_u'>0$. 
Consider $\mathbf{x}=\left(\frac{1}{c_1'},\ldots,\frac{1}{c_u'}\right)\in \mathbb{A}^{1\times u}$. Then, for $\mathbf{D}=\mathbf{c}'\mathbf{x}\in\mathbb{A}^{u\times u}$
$$
\mathbf{D}_{ij}=\frac{c_i'}{c_j'}  \text{ and } \mathbf{D}_{ij}=\begin{cases}  1 & \text{if $i=j$}\\
>1 & \text{if $i<j$}\\
< 1 & \text{if $i>j$}.
\end{cases}
$$
Let $m$ be the greatest value  in $\mathbf{D}$ smaller than $1$.
% G: I think the m instantiated here is not correct
%, i.e., $m=\frac{b_s}{b_1}$.
Consider $\mathbf{E}=\mathbf{D}- m\mathbf{J}$.
Then,
$$
\mathbf{E}_{ij}=\frac{b_i'}{b_j'}- m \text{ and } \mathbf{E}_{ij}=\begin{cases}  1-m & \text{if $i=j$} \\
> 0 & \text{if $i<j$}\\
\leq 0  & \text{if $i>j$}.
\end{cases}
$$
As a consequence,
$$
\text{ReLU}(\mathbf{E})_{ij}=\begin{cases}  1-m & \text{if $i=j$}\\
>0 & \text{if $i<j$}\\
0  & \text{if $i>j$}.
\end{cases}
$$
This is an upper triangular matrix with (nonzero) value $1-m$ on its diagonal. It is therefore non-singular. 
We observe that $\mathbf{Q}\text{ReLU}(\mathbf{E})=\text{ReLU}(\mathbf{Q}\mathbf{E})$ for any row permutation $Q$. Furthermore, non-singularity is preserved under row permutations and $\mathbf{Q}\mathbf{J}=\mathbf{J}$. Hence, if we define $\mathbf{X}=\mathbf{z}\mathbf{x}$ and use the permutation matrix $\mathbf{P}$, then:
\begin{align*}
\mathbf{P}\text{ReLU}(\mathbf{C}\mathbf{X}-m\mathbf{J})&=
\text{ReLU}(\mathbf{P}\mathbf{C}\mathbf{z}\mathbf{x}-m\mathbf{P}\mathbf{J})=\text{ReLU}(\mathbf{E}-m\mathbf{J}),
\end{align*}
and we have that $\text{ReLU}(\mathbf{C}\mathbf{X}-m\mathbf{J})$ is non-singular, as desired. This concludes the proof of the Lemma.
%So, the lemma is satisfied by taking $m$ as above and
%%$m=b_s/b_1$ and % G: this still looks wrong
%$\mathbf{X}=\mathbf{z}\mathbf{x}$.
\end{proof}

We now apply this lemma to the matrix $\mathsf{uniq}(\mathbf{M}^{(t)})$, with $\mathbf{M}^{(t)}\in\mathbb{A}^{n\times s}$ consisting of the rows $\pmb{\mu}^{(t)}_v$, for $v\in V$. Inspecting the expression~(\ref{eq:linearcomb}) for $\pmb{\mu}^{(t)}_v$ we see that each row in $\mathbf{M}^{(t)}$ holds non-negative values and no row consists entirely out of zeroes. Let $\mathbf{X}^{(t)}$ and $m^{(t)}$ be the matrix and constant 
returned by the Lemma such that $\text{ReLU}\left(\mathsf{uniq}(\mathbf{M}^{(t)})\mathbf{X}^{(t)}-m^{(t)}\mathbf{J}_{u\times u}\right)$ is an $u\times u$ non-singular matrix. We now define
$$
\pmb{\ell}_M^{(t)}:=\text{ReLU}\left(\mathbf{M}^{(t)}\mathbf{X}^{(t)}-m^{(t)}\mathbf{J}_{n\times u}\right).$$
\floris{There is minor mismatch here. We assumed so far that labelling take values in $\mathbf{A}^s$. Here, we create a labelling in $\mathbf{A}^u$. We may want to padd with zeroes...}
From the non-singularity of $\text{ReLU}\left(\mathsf{uniq}(\mathbf{M}^{(t)})\mathbf{X}^{(t)}-m^{(t)}\mathbf{J}\right)$ we can immediately infer that $\pmb{\ell}_M^{(t)}$ is row-independent modulo equality. It remains to argue that 
$\pmb{\ell}_M^{(t)}\equiv\pmb{\ell}_{M_{\textsl{WL}}}^{(t)}$. This now follows from the fact that $\pmb{\mu}^{(t)}\equiv \pmb{\ell}_{M_{\textsl{WL}}}^{(t)}$
and each of the $u$ unique labels assigned by $\pmb{\mu}^{(t)}$ uniquely corresponds to a row in $\mathsf{uniq}(\mathbf{M}^{(t)})$, which in turn can be mapped bijectively to a row in $\text{ReLU}\left(\mathsf{uniq}(\mathbf{M}^{(t)})\mathbf{X}^{(t)}-m^{(t)}\mathbf{J}_{u\times u
\right)$. We conclude by observing that the desired weight matrices and bias vector are now given by
$\mathbf{W}^{(t)}:=\mathbf{U}^{(t)}\mathbf{X}^{(t)}$ 
and $\mathbf{b}^{(t)}:=-m^{(t)}\mathbf{1}$. This concludes the proof of the theorem.\end{proof}

We remark that the previous proof can be used for $\architecture_{\textsl{GNN}}^{\textsl{sign}}$ as well. One just has to use Lemma 9 in~\cite{grohewl} instead of
Lemma~\ref{lem:ReLUlemma9}. We include the statement of Lemma 9
here for completeness.
\begin{lemma}[Lemma 9 in~\cite{grohewl}]\label{lem:signlemma9}
  Let  $\mathbf{C}\in \mathbb{A}^{u\times s}$ be a matrix in which 
  all entries are non-negative and  all rows are pairwise disjoint.
%  \footnote{I believe that this can be
%  guaranteed in 1-WL}).\todo{G: with our extended features we actually guarantee this for free by adding the 1 column; also, t as dimension is a bad choice\ldots}
  Then there exists a matrix $\mathbf{X}\in\mathbb{A}^{s\times s}$ and a constant $m\in\mathbb{A}$
  such that $\text{\normalfont sign}(\mathbf{CX}-\mathbf{J})$ is a
  non-singular matrix in $\mathbb{A}^{u\times u}$.
\end{lemma}

We observe that the bias vector for the sign activation function is the same for every $t$. A similar statement holds for the ReLU function.
Indeed, we recall that we apply Lemma~\ref{lem:ReLUlemma9} to
$\mathbf{uniq}(\mathbf{M}^{(t)})$. For every $t$,
the entries in this matrix are of the form $i+p<i+1$
or $i$, for $i\in[n]$. Hence, for every $t$, the maximal entry (denoted by $C$ in the proof Lemma~\ref{lem:ReLUlemma9}  is $n+1$. The value $m^{(t)}$ relates to the largest possible ratios, smaller than $1$, of elements in the matrix constructed in  Lemma~\ref{lem:ReLUlemma9}. This ratio is upper bounded by $\frac{(n+1)^s-1}{(n+1)^s}$. Hence, taking
any $m^{(t)}=m$ for $\frac{(n+1)^s-1}{(n+1)^s}<m<1$
suffices. We can take $m$ to be arbitrarily close to $1$, but not $1$ itself.

We can thus strengthen Theorem~\ref{thm:grohe_lower}, as follows. We denote by $\architecture_{\textsl{GNN}^-}$ the 
class of aMPNNs using message and update functions of the form:
\begin{equation}\textsc{Msg}^{(t)}\bigl(\mathbf{x},\mathbf{y},v,u):=\mathbf{y}\mathbf{W}^{(t)}
\text{ and } 
\textsc{Upd}^{(t)}(\mathbf{x},\mathbf{y}):=\sigma\left(p\mathbf{x}\mathbf{W}^{(t)}+\mathbf{y} -q \mathbf{1}^{(t)}\right) \label{eq:MPNN-GNN-simple}
\end{equation}
for some value $p,q\in\mathbb{A}$, $0<p,q<1$ and weight matrix $\mathbf{W}^{(t)}\in\mathbb{A}^{s\times s}$, and where
$\sigma$ can be either the sign or ReLU function.
\begin{corollary}
The class $\architecture_{\textsl{GNN}^-}$ is equally strong as $\architecture_{\textsl{GNN}}$ and is equally strong as $\architectureWL$.\qed
\end{corollary}
We remark that the factor two, needed for the ReLU activation function Theorem~\ref{thm:grohe_lower}, has been eliminated. Phrased in terms of graph neural networks, an aMPNN in $\architecture_{\textsl{GNN}^-}$ is of the form
$$
\mathbf{L}^{(t)}:=\sigma\left((\mathbf{A}+p\mathbf{I})\mathbf{L}^{(t-1)}\mathbf{W}^{(t)}-q\mathbf{J}\right),
$$
and thus these suffices to modelling WL.

In contrast, if one inspects the proof in
~\cite{grohewl} for the sign activation function, the miniaml
$$
[\mathbf{L}^{(0)},\mathbf{L}^{(t)}]:=\sigma\left([\mathbf{L}^{(0)},\mathbf{L}^{(t-1)}]\begin{pmatrix}
\mathbf{I}_{s\times s} & \mathbf{O}_{s\times s}\\
\mathbf{O}_{s\times s} & \mathbf{O}_{s\times s}\end{pmatrix}
+\mathbf{A}[\mathbf{L}^{(0)},\mathbf{L}^{(t-1)}]
\begin{pmatrix}
\mathbf{O}_{s\times s} & \mathbf{O}_{s\times s}\\
\mathbf{O}_{s\times s} & \mathbf{W}_{s\times s}^{(t)}\end{pmatrix}-
\begin{pmatrix}
\mathbf{O}_{s\times s} & \mathbf{J}_{s\times s}\\
\mathbf{O}_{s\times s} & \mathbf{J}_{s\times s}\end{pmatrix}
\right).
$$

%!TEX root =main.tex
\section{Upper bounding the distinguishing power}
\floris{
In this section we bound several class of GNNs by WL or WLL. Our key observations are the following: when GNNs do not incorporate self-features
then they are bounded by WWL, otherwise they are bounded by WL. Furthermore,
whenever $\mathbf{R}$ contains degree information, GNNs are one-step ahead
of WWL and WL.}

\tobdone{
Transition from the general GNN architecture using $\mathbf{N}$ in the previous section to the special cases. GNNs of the form~(\ref{eq:architecture}) below. Filip: how did you envisage this transition?}

We start by analysing the distinguishing power of GNN architectures of the form:
% We consider a GNN architecture which generalises commonly used GNN architectures. Given a labeled
% graph $(G,\pmb{\ell})$ with $G=(V,E)$, we denote by $\mathbf{F}^{(t)}$ the feature matrix assigning to each vertex $v\in V$ a feature vector $\mathbf{F}_{v\bullet}$. In layer $t$ of the the GNN architecture, $\mathbf{F}^{(t)}$ is updated as follows:
\begin{equation}
\mathbf{F}^{(t)}:=\sigma\left(\mathbf{F}^{(t-1)}\mathbf{W}_1^{(t-1)}+\mathbf{L}(\mathbf{A}+p\mathbf{I})\mathbf{R}\mathbf{F}^{(t-1)}\mathbf{W}_2^{(t-1)} + q\mathbf{B}^{(t-1)}\right), \label{eq:architecture}
\end{equation}
where $\mathbf{L}$ and $\mathbf{R}$ positive diagonal matrices, $p$ and $q$ are learnable parameters in $[0,1]$,  $\mathbf{W}_1^{(t-1)}$ and $\mathbf{W}_2^{(t-1)}$  are learnable weight matrices, $\mathbf{B}^{(t-1)}$ is a bias matrix with all the same rows, and finally, $\sigma$ is a non-linear activation function such as sign or ReLU. It should be clear that by choosing 
$\mathbf{L}$, $\mathbf{R}$, $p$ and $q$  in an appropriate way, one can obtain all GNN architectures mentioned so far.

Although it is possible to upper bound the distinguishing power of GNNs of the form~(\ref{eq:architecture}) in full generality\footnote{To this aim it suffices to incorporate the entries in the matrices $\mathbf{L}$ and $\mathbf{R}$ in the labelings and use WL on this extended labeling.}, we make the following additional assumptions on the matrices $\mathbf{L}$ and $\mathbf{R}$, motivated by the specific instantiations of $\mathbf{L}$ and $\mathbf{R}$ in GNNs found in the literature.

Let us denote by $\pmb{\ell}^L:V\to \Rb$ the vertex labeling defined by $\pmb{\ell}^L(v):=\mathbf{L}_{vv}$ and by $\pmb{\ell}^R:V\to \Rb$ the vertex labeling defined by 
$\pmb{\ell}^R(v):=\mathbf{R}_{vv}$ for all $v\in V$. We say that GNNs of the form~(\ref{eq:architecture}) are \textit{degree-determined} if, for any given  labeled graph $(G,\pmb{\ell})$, if $d_v=d_w$ then $\pmb{\ell}^L_v=\pmb{\ell}^L_w$ and $\pmb{\ell}^R_v=\pmb{\ell}^R_w$ hold. In other words,
the entries on the diagonals in $\mathbf{L}$ and $\mathbf{R}$ only depend on the degree of vertices.
We denote by ${\cal C}_{\textsl{deg}}$ the class of GNN architectures of the form~(\ref{eq:architecture}) that are degree-determined.

In ${\cal C}_{\textsl{deg}}$ we further zoom in on some special GNNs.
In particular, we also consider the case when  $\pmb{\ell}^R$ is a 
\textit{constant} labeling. We say that a  GNN of the form~(\ref{eq:architecture}) is \textit{constant on the right} if
it belongs to ${\cal C}_{\textsl{deg}}$ and
$\pmb{\ell}^R_v=\pmb{\ell}^R_w$ for all $v,w\in V$. In other words, the entries on the diagonal $\mathbf{R}$ are all the same, i.e., $\mathbf{R}$ is (a multiple of the) identity matrix $\mathbf{I}$.
We denote by ${\cal C}_{\textsl{cst}}$ the class of GNN architectures of the form~(\ref{eq:architecture}) that are constant on the right. By definition,
${\cal C}_{\textsl{cst}}\subseteq {\cal C}_{\textsl{deg}}$.

It is easily verified that all GNN architectures
mentioned earlier reside in the class ${\cal C}_{\textsl{deg}}$ and those
in which $\mathbf{R}=\mathbf{I}$  belong to 
the smaller class ${\cal C}_{\textsl{cst}}$.


With regards to their distinguishing power, GNNs in ${\cal C}_{\textsl{deg}}$
are still bounded by WL but with a factor of $1$. That is, they are one step ahead of WL. This is because  the diagonal entries in $\mathbf{R}$ carry information about degrees which are only determined during the first step of the WL algorithm. In contrast, GNNs in ${\cal C}_{\textsl{cst}}$ are bounded by WL, without any additional factor. This holds, even when $\mathbf{L}$ contains degree information. It shows an asymmetry between $\mathbf{L}$ and $\mathbf{R}$
and in terms of distinguishing power, $\mathbf{L}$ can be omitted from GNN architectures. \floris{There may be other arguments, perhaps spectral-based, that are in favour of keeping $\mathbf{L}$? Don't know.}

\tobdone{In the upper bound proofs it is always assumed that $\pmb{\ell}\sqsubseteq \mathbf{F}^{(0)}$. This seems something that needs to be incorporated when we say that one class is weaker than another one? There is also a comment wrt initial labeling below.}
\begin{proposition}\label{prop:boundconstantR}
The class ${\cal C}_{\textsl{cst}}$  is bounded by WL.
% Let $(G,\pmb{\ell})$ be a labeled graph and assume that $\pmb{\ell}\sqsubseteq\mathbf{F}^{(0)}$.
%  % for some $k\geq 0$.
% Then, GNN architectures of the form~(\ref{eq:architecture}) with constant $\pmb{\ell}^R$ are bounded by WL on $(G,\pmb{\ell})$.
\end{proposition}
\begin{proof}
We recall that ${\cal C}_{\textsl{WL}}$ just consists of the WL-algorithm, generating vertex labelings $\pmb{\ell}{}^{(0)}:=\pmb{\ell},\pmb{\ell}^{(1)},\ldots, \pmb{\ell}^{(k)}$. We show that for any GNN in 
${\cal C}_{\textsl{cst}}$, if we denote by
 $\mathbf{F}^{(0)},\mathbf{F}^{(1)},\ldots, \mathbf{F}^{(k)}$ the features
 computed in the different layers, then it holds that $\pmb{\ell}{}^{(t)}\sqsubseteq \mathbf{F}^{(t)}$ for any $t$. 

We verify this by induction on the number of iterations. For $t=0$, we have, by assumption, that 
$\pmb{\ell}\sqsubseteq \mathbf{F}^{(0)}$. Hence, the hypothesis holds for the base case.
% \pmb{\ell}{}^{(1)}\sqsubseteq
% Clearly,
%
% $\hat{\pmb{\ell}}{}^{(0)}\sqsubseteq \pmb{\ell}$ and hence also
% $\hat{\pmb{\ell}}{}^{(0)}\sqsubseteq\mathbf{F}^{(0)}$.
 We next assume that the induction hypothesis holds for all layers smaller than $t$ and consider layer $t$. More specifically, we assume that 
 $\pmb{\ell}^{(t-1)}\sqsubseteq \mathbf{F}^{(t-1)}$ and
 % For GNN architectures with non-constant $\pmb{\ell}^R$ (but degree-determined), we assume that $\pmb{\ell}^{(t+1)}\sqsubseteq \mathbf{F}^{(t)}$.
we need to show that 
$\pmb{\ell}{}^{(t)}_v=\pmb{\ell}{}^{(t)}_w$ implies that $\mathbf{F}^{(t)}_{v\bullet}=\mathbf{F}^{(t)}_{w\bullet}$. By definition,
$\pmb{\ell}{}^{(t)}_v=\pmb{\ell}{}^{(t)}_w$ implies
$\pmb{\ell}{}^{(t-1)}_v=\pmb{\ell}{}^{(t-1)}_w$ and 
$$
\ldbl \pmb{\ell}{}^{(t-1)}_u \st u \in N_G(v) \rdbl=
 \ldbl \pmb{\ell}{}^{(t-1)}_u \st u \in N_G(w) \rdbl.$$
 In other words, there exists a bijection between $N_G(v)$ and $N_G(w)$ such that for each $u\in N_G(v)$ and corresponding $u'\in N_G(w)$, $\pmb{\ell}{}^{(t-1)}_u=\pmb{\ell}{}^{(t-1)}_{u'}$. This bijection also implies 
 that $d_v=d_w$ and hence $\pmb{\ell}^L_{v}=\pmb{\ell}^L_{w}$ since we consider the class ${\cal C}_{\textsl{cst}}$. That is,
 $\mathbf{L}_{vv}=\mathbf{L}_{ww}$. Furthermore, we recall that in this class of GNNs, $\mathbf{R}_{uu}=\mathbf{R}_{u'u'}$ for any $u,u'\in V$. From the induction hypothesis we further know that 
 $\mathbf{F}^{(t-1)}_{v\bullet}=\mathbf{F}^{(t-1)}_{w\bullet}$ and 
 $\mathbf{F}^{(t-1)}_{u\bullet}=\mathbf{F}^{(t-1)}_{u'\bullet}$ for  every $u\in N_G(v)$ and corresponding $u'\in N_G(w)$.
%
%   Furthermore,
% since $\pmb{\ell}{}^{(t)}\sqsubseteq \pmb{\ell}{}^{(t-1)}\sqsubseteq \cdots\sqsubseteq \pmb{\ell}{}^{(1)}\sqsubseteq \pmb{\ell}{}^{(0)}$, we have that
%  $\pmb{\ell}{}^{(0)}_u=\pmb{\ell}{}^{(0)}_{u'}$ for every $u\in N_G(v)$ and corresponding
%  $u'\in N_G(w)$. In particular, we have that $d_u=d_{u'}$ and thus also $\mathbf{L}_{uu}=\mathbf{L}_{u'u'}$ and
%
%  this implies that
%  $\hat{\pmb{\ell}}{}^{(0)}_v=\hat{\pmb{\ell}}{}^{(0)}_w$ and that there is a bijection $b:N_G(v)\to N_G(w):u\mapsto u'$ such that $\hat{\pmb{\ell}}{}^{(t)}_u=\hat{\pmb{\ell}}{}^{(t)}_{u'}$ and hence also
%  $\hat{\pmb{\ell}}{}^{(0)}_u=\hat{\pmb{\ell}}{}^{(0)}_{u'}$. From the definition of $\hat{\pmb{\ell}}{}^{(0)}$, $\hat{\pmb{\ell}}{}^{(0)}_v=\hat{\pmb{\ell}}{}^{(0)}_w$ implies that
%  $\mathbf{L}_{vv}=\mathbf{L}_{ww}$ and
%  $\mathbf{R}_{vv}=\mathbf{R}_{ww}$. Similarly, for every $u\in N_G(v)$ and corresponding $u'\in N_G(w)$,
% $\hat{\pmb{\ell}}{}^{(0)}_u=\hat{\pmb{\ell}}{}^{(0)}_{u'}$ implies that   $\mathbf{L}_{uu}=\mathbf{L}_{u'u'}$ and $\mathbf{R}_{uu}=\mathbf{R}_{u'u'}$. By the induction hypothesis we also have that
%  $\mathbf{F}^{(t)}_{v\bullet}=\mathbf{F}^{(t)}_{w\bullet}$, and for every $u\in N_G(v)$
%    and corresponding $u'\in N_G(w)$, $\mathbf{F}^{(t)}_{u\bullet}=\mathbf{F}^{(t)}_{u'\bullet}$.
  It now suffices to observe that
  \begin{align*}
	  \mathbf{F}^{(t)}_{v\bullet}&=\sigma\Biggl(\mathbf{F}^{(t-1)}_{v\bullet}\mathbf{W}_1^{(t-1)}+\mathbf{L}_{vv}\biggl(\Bigl(\sum_{u\in N_G(v)} \mathbf{R}_{uu}\mathbf{F}^{(t-1)}_{u\bullet}\Bigr)+p\mathbf{R}_{vv}\mathbf{F}^{(t-1)}_{v\bullet}\biggr)\mathbf{W}_2^{(t-1)}+ q\mathbf{B}^{(t-1)}_{v\bullet}\Biggr)\\
	 &=\sigma\Biggl(\mathbf{F}^{(t-1)}_{w\bullet}\mathbf{W}_1^{(t-1)}+\mathbf{L}_{ww}\biggl(\Bigl(\sum_{u\in N_G(w)} \mathbf{R}_{uu}\mathbf{F}^{(t-1)}_{u\bullet}\Bigr)+p\mathbf{R}_{ww}\mathbf{F}^{(t-1)}_{v\bullet}\biggr)\mathbf{W}_2^{(t-1)}+ q\mathbf{B}^{(t-1)}_{w\bullet}\Biggr) \end{align*}
as desired. We here additionally used that $\mathbf{B}^{(t-1)}$ is a matrix consisting of all the same rows and hence $\mathbf{B}^{(t-1)}_{v\bullet}=\mathbf{B}^{(t-1)}_{w\bullet}$ for any $v,w\in V$.
\end{proof}

For the class ${\cal C}_{\textsl{deg}}$ we obtain the following upper bound.
% When $\pmb{\ell}^R$ is not constant, but nevertheless degree-determined, we obtain the following upper bound.
\begin{proposition}\label{prop:boundnonconstantR}
	The class ${\cal C}_{\textsl{deg}}$ is bounded by WL, with a factor $1$.	
%
% Let $(G,\pmb{\ell})$ be a labeled graph and assume that $\pmb{\ell}\sqsubseteq\mathbf{F}^{(0)}$.
%  % for some $k\geq 0$.
% Then, GNN architectures of the form~(\ref{eq:architecture}) with non-constant $\pmb{\ell}^R$ are bounded by WL on $(G,\pmb{\ell}^{(1)})$.
\end{proposition}
\begin{proof}
The proof is almost the same as the previous proof. That is,
we show the upper bound by WL by induction on the number of iterations. For $t=0$, we have, by assumption, that 
$\pmb{\ell}\sqsubseteq \mathbf{F}^{(0)}$. Hence, since $\pmb{\ell}{}^{(1)}\sqsubseteq\pmb{\ell}$ we also have that $\pmb{\ell}{}^{(1)}\sqsubseteq\mathbf{F}^{(0)}$ and 
the induction hypothesis holds for the base case.
% \pmb{\ell}{}^{(1)}\sqsubseteq
% Clearly,
%
% $\hat{\pmb{\ell}}{}^{(0)}\sqsubseteq \pmb{\ell}$ and hence also
% $\hat{\pmb{\ell}}{}^{(0)}\sqsubseteq\mathbf{F}^{(0)}$.
 We next assume that the induction hypotheses holds for $t\geq 0$ and consider $t+1$. More specifically, we assume that 
 $\pmb{\ell}^{(t+1)}\sqsubseteq \mathbf{F}^{(t)}$ and
 % For GNN architectures with non-constant $\pmb{\ell}^R$ (but degree-determined), we assume that $\pmb{\ell}^{(t+1)}\sqsubseteq \mathbf{F}^{(t)}$.
we need to show that 
$\pmb{\ell}{}^{(t+2)}_v=\pmb{\ell}{}^{(t+2)}_w$ implies that $\mathbf{F}^{(t+1)}_{v\bullet}=\mathbf{F}^{(t+1)}_{w\bullet}$. By definition,
$\pmb{\ell}{}^{(t+2)}_v=\pmb{\ell}{}^{(t+2)}_w$ implies
$\pmb{\ell}{}^{(t+1)}_v=\pmb{\ell}{}^{(t+1)}_w$ and 
$$
\ldbl \pmb{\ell}{}^{(t+1)}_u \st u \in N_G(v) \rdbl=
 \ldbl \pmb{\ell}{}^{(t+1)}_u \st u \in N_G(w) \rdbl.$$
 In other words, there exists a bijection between $N_G(v)$ and $N_G(w)$ such that for each $u\in N_G(v)$ and corresponding $u'\in N_G(w)$, $\pmb{\ell}{}^{(t+1)}_u=\pmb{\ell}{}^{(t+1)}_{u'}$. 
  This bijection also implies that
 that $d_v=d_w$ and hence $\pmb{\ell}^L_{v}=\pmb{\ell}^L_{w}$ and $\pmb{\ell}^{R}_{v}=\pmb{\ell}^R_{w}$. That is,
 $\mathbf{L}_{vv}=\mathbf{L}_{ww}$ and $\mathbf{R}_{vv}=\mathbf{R}_{ww}$ hold.
We next argue in a different way as in the previous proof. We observe that
 $\pmb{\ell}{}^{(t+1)}\sqsubseteq \pmb{\ell}{}^{(t)}\sqsubseteq \cdots\sqsubseteq \pmb{\ell}{}^{(1)}$. Hence, we have that $\pmb{\ell}{}^{(t+1)}_u=\pmb{\ell}{}^{(t+1)}_{u'}$ implies 
 $\pmb{\ell}{}^{(1)}_u=\pmb{\ell}{}^{(1)}_{u'}$. We note that this implication was not guaranteed in the previous proof. Indeed, we can only infer that  $\pmb{\ell}{}^{(0)}_u=\pmb{\ell}{}^{(0)}_{u'}$ when $t=1$. 
In the current setting, we thus know that $d_{u}=d_{u'}$ for every $u\in N_G(v)$ and $u'\in N_G(w)$. We may thus infer that $\mathbf{R}_{uu}=\mathbf{R}_{u'u'}$ for every $u\in N_G(v)$ and corresponding $u'\in N_G(w)$.
 From the induction hypothesis we further know that 
 $\mathbf{F}^{(t)}_{v\bullet}=\mathbf{F}^{(t)}_{w\bullet}$ and 
 $\mathbf{F}^{(t)}_{u\bullet}=\mathbf{F}^{(t)}_{u'\bullet}$ for  every $u\in N_G(v)$ and corresponding $u'\in N_G(w)$.
%
%   Furthermore,
% since $\pmb{\ell}{}^{(t)}\sqsubseteq \pmb{\ell}{}^{(t-1)}\sqsubseteq \cdots\sqsubseteq \pmb{\ell}{}^{(1)}\sqsubseteq \pmb{\ell}{}^{(0)}$, we have that
%  $\pmb{\ell}{}^{(0)}_u=\pmb{\ell}{}^{(0)}_{u'}$ for every $u\in N_G(v)$ and corresponding
%  $u'\in N_G(w)$. In particular, we have that $d_u=d_{u'}$ and thus also $\mathbf{L}_{uu}=\mathbf{L}_{u'u'}$ and
%
%  this implies that
%  $\hat{\pmb{\ell}}{}^{(0)}_v=\hat{\pmb{\ell}}{}^{(0)}_w$ and that there is a bijection $b:N_G(v)\to N_G(w):u\mapsto u'$ such that $\hat{\pmb{\ell}}{}^{(t)}_u=\hat{\pmb{\ell}}{}^{(t)}_{u'}$ and hence also
%  $\hat{\pmb{\ell}}{}^{(0)}_u=\hat{\pmb{\ell}}{}^{(0)}_{u'}$. From the definition of $\hat{\pmb{\ell}}{}^{(0)}$, $\hat{\pmb{\ell}}{}^{(0)}_v=\hat{\pmb{\ell}}{}^{(0)}_w$ implies that
%  $\mathbf{L}_{vv}=\mathbf{L}_{ww}$ and
%  $\mathbf{R}_{vv}=\mathbf{R}_{ww}$. Similarly, for every $u\in N_G(v)$ and corresponding $u'\in N_G(w)$,
% $\hat{\pmb{\ell}}{}^{(0)}_u=\hat{\pmb{\ell}}{}^{(0)}_{u'}$ implies that   $\mathbf{L}_{uu}=\mathbf{L}_{u'u'}$ and $\mathbf{R}_{uu}=\mathbf{R}_{u'u'}$. By the induction hypothesis we also have that
%  $\mathbf{F}^{(t)}_{v\bullet}=\mathbf{F}^{(t)}_{w\bullet}$, and for every $u\in N_G(v)$
%    and corresponding $u'\in N_G(w)$, $\mathbf{F}^{(t)}_{u\bullet}=\mathbf{F}^{(t)}_{u'\bullet}$.
  It now suffices to observe that
  \begin{align*}
	  \mathbf{F}^{(t+1)}_{v\bullet}&=\sigma\Biggl(\mathbf{F}^{(t)}_{v\bullet}\mathbf{W}_1^{(t)}+\mathbf{L}_{vv}\biggl(\Bigl(\sum_{u\in N_G(v)} \mathbf{R}_{uu}\mathbf{F}^{(t)}_{u\bullet}\Bigr)+p\mathbf{R}_{vv}\mathbf{F}^{(t)}_{v\bullet}\biggr)\mathbf{W}^{(t)}+ q\mathbf{J}_{v\bullet}\Biggr)\\
	 & =\sigma\Biggl(\mathbf{F}^{(t)}_{w\bullet}\mathbf{W}_1^{(t)}+\mathbf{L}_{ww}\biggl(\Bigl(\!\!\sum_{u'\in N_G(w)}\!\! \mathbf{R}_{u'u'}\mathbf{F}^{(t)}_{u'\bullet}\Bigr)+p\mathbf{R}_{ww}\mathbf{F}^{(t)}_{w\bullet}\biggr)\mathbf{W}^{(t)}+ q\mathbf{J}_{w\bullet}\Biggr)\\
	  &=\mathbf{F}^{(t+1)}_{w\bullet},
\end{align*}
as desired.
\end{proof}

In other words, we have that ${\cal C}_{\textsl{deg}}\sqsubseteq {\cal C}_{\textsl{WL}}$ with constant factor $1$, and ${\cal C}_{\textsl{cst}}\sqsubseteq {\cal C}_{\textsl{WL}}$, without any factor.
Whenever ${\cal C}\sqsubseteq {\cal C}'$ without any factor, then clearly
also ${\cal C}\sqsubseteq {\cal C}'$, with any constant factor. The converse does not hold, in general, as illustrated by the following example.
%
%
% when a class of GNNs is bounded by WL, without any
% It should be clear that GNNs of the form~(\ref{eq:architecture}) which are bounded by WL on $(G,\pmb{\ell})$ are also bounded by WL on $(G,\pmb{\ell}{}^{(1)})$. The converse is not true, as is illustrated by the following example.
\openprob{Is there a simpler example for the one below?}



%
%
%
% We conclude this section by observing that previous propositions generalise as follows.
% \begin{corollary}
% \begin{itemize}
% \item[(a)]Let $(G,\pmb{\ell})$ be a labeled graph and assume that $\pmb{\ell}^{(k)}\sqsubseteq\mathbf{F}^{(0)}$ for some $k\geq 0$.
% Then, GNN architectures of the form~(\ref{eq:architecture}) with constant $\pmb{\ell}^R$ are bounded by WL on $(G,\pmb{\ell}^{(k)})$.
% \item[(b)] Let $(G,\pmb{\ell})$ be a labeled graph and assume that $\pmb{\ell}^{(k)}\sqsubseteq\mathbf{F}^{(0)}$ for some $k\geq 0$.
% Then, GNN architectures of the form~(\ref{eq:architecture}) with non-constant $\pmb{\ell}^R$ are bounded by WL on $(G,\pmb{\ell}^{(k+1)})$.
% \end{itemize}
% \end{corollary}
% %  $\pmb{\ell}^{(k)}\sqsubseteq\hat{\pmb{\ell}}$
% % and $\pmb{\ell}^{(k)}\sqsubseteq\hat{\pmb{\ell}}$ and $$
% %
% % Remark: iterated degrees...
% % More generally, we have the following property.
% % \begin{corollary}\label{cor:augmented}
% % If $\pmb{\ell}^{(k)}\sqsubseteq\hat{\pmb{\ell}}$ holds, then the GNN architecture~(\ref{eq:architecture})  is bounded by 1-WL, starting from $\pmb{\ell}^{(k)}$.
% % \end{corollary}
% % Intuitively, this implies that the GNN architecture is $k$-steps ahead of the 1-WL algorithm.
% % More precisely, $\pmb{\ell}^{(t)}\sqsubseteq \mathbf{F}^{(t-k)}$ for $t\geq k$.
% \begin{proof}
% \tobdone{Proof.}
% % 	For (a)~, if 	$\pmb{\ell}^{(k)}\sqsubseteq\hat{\pmb{\ell}}$ holds, then also
% % $\pmb{\ell}^{(k+t)}\sqsubseteq\hat{\pmb{\ell}}{}^{(t)}$ holds for any $t\geq 0$. Hence,
% % $\pmb{\ell}^{(k+t)}\sqsubseteq \mathbf{F}^{(t)}$ or $\pmb{\ell}^{(k)}\sqsubseteq\mathbf{F}^{(t-k)}$ by Theorem~\ref{thm:generalbound}.
% % For (b)~, it is easily verified that the proof of Corollary~\ref{cor:weak} can be generalized to this setting.
% \end{proof}

\openprob{Perhaps we could point out that the one-step advantage of 
GNNs in ${\cal C}_{\textsl{deg}}$ over GNNs in ${\cal C}_{\textsl{cst}}$
can be easily overcome, at least theoretically. How? We just kickstart
GNNs in  ${\cal C}_{\textsl{cst}}$ with a labeling $\pmb{\ell}$ that contains
degree information, e.g., $\hat{\pmb{\ell}}_v=(\pmb{\ell},d_v)$ for each $v\in V$. The current definition of how we compare two classes of GNNs may need some
revision, since in that definition the same $\pmb{\ell}$ is assumed. More generally, one can start running GNNs from labelings that incorporate information
which WL only detects after a number of iterations. The reason that I want to do this is because a paper I found on  Graph Feature Networks (GFNs)~\cite{chen2019powerful}. In such networks, the initial feature vectors $\mathbf{F}^{(0)}$ are expanded to $\hat{\mathbf{F}}{}^{(0)}:=[\mathbf{d},\mathbf{F}^{(0)},\mathbf{N}\mathbf{F}^{(0)},\ldots,\mathbf{N}^k\mathbf{F}^{(0)}]$ with $\mathbf{N}=\tilde{\mathbf{D}}^{-1/2}(\mathbf{A}+p\mathbf{I})\tilde{\mathbf{D}}^{-1/2}$ and 
$\mathbf{F}^{(1)}:=\sigma(\hat{\mathbf{F}}{}^{(0)}\mathbf{W}^{(0)})$ and for $t>1$: $\mathbf{F}^{(t)}:=\sigma(\mathbf{F}{}^{(t-1)}\mathbf{W}^{(t-1)})$.
It needs further investigation what precisely can be said here. The paper seems to say that in this way less layers are needed. We may want to make this precise.
}

As previously mentioned, some of the GNN architectures do not incorporate the features of the vertices themselves in each layer. We next show that such GNNs are bounded in their distinguishing power by WWL. More specifically, let ${\cal C}_{\textsl{noself}}$ be the class of GNN architectures of the form ~(\ref{eq:architecture}) such that in each layer $\mathbf{W}_1^{(t)}$ is absent (i.e., it is always set to $\mathbf{O}$) and $p=0$ (i.e., the identity matrix $\mathbf{I}$ is ignored).

To state our result we need a notion of labeled graphs which are consistent with degree information. More specifically, these are labeled graphs $(G,\pmb{\ell})$ such that $d_v=d_w\Rightarrow \pmb{\ell}_v=\pmb{\ell}_w$. In other words, vertices with the same degree should not be labeled differently.

\begin{proposition}
(a)~The class  ${\cal C}_{\textsl{noself}}\cap {\cal C}_{\textsl{cst}}$ is bounded by WWL (with no additional factor). (b)~
The class ${\cal C}_{\textsl{noself}}\cap {\cal C}_{\textsl{deg}}$ is bounded by WWL with a factor $1$ on labeled graphs $(G,\pmb{\ell})$ consistent with degree information.
\end{proposition}
\begin{proof}
In the proof we use $\pmb{\mu}$ for $\pmb{\ell}$ to make the distinction between WL and WWL clear.

(a)~We show the upper bound by WWL by induction on the number of iterations. For $t=0$, we have, by assumption, that 
$\pmb{\mu}\sqsubseteq \mathbf{F}^{(0)}$. 
% Since we consider labeled graphs which are consistent with degree information, and $\pmb{\mu}^{(1)}_v=\pmb{\mu}^{(1)}_w$
% implies $d_v=d_w$, we also h
We next assume that the induction hypothesis holds for $t\geq 0$ and consider $t+1$. We need to show that 
$\pmb{\mu}{}^{(t+1)}_v=\pmb{\mu}{}^{(t+1)}_w$ implies that $\mathbf{F}^{(t+1)}_{v\bullet}=\mathbf{F}^{(t+1)}_{w\bullet}$. By definition,
$\pmb{\mu}^{(t+1)}_v=\pmb{\mu}{}^{(t+1)}_w$ implies
$$
\ldbl \pmb{\mu}{}^{(t)}_u \st u \in N_G(v) \rdbl=
 \ldbl \pmb{\mu}{}^{(t)}_u \st u \in N_G(w) \rdbl.$$
We remark that this implies that $d_v=d_w$. Since our GNNs belong to ${\cal C}_{\textsl{cst}}$, this in turn implies that $\mathbf{L}_{vv}=\mathbf{L}_{ww}$.
In addition, $\mathbf{R}_{uu}=\mathbf{R}_{u'u'}$ for all $u,u'\in V$.
%
% It is readily verified that $\hat{\pmb{\mu}}{}^{(t)}\sqsubseteq \hat{\pmb{\mu}}{}^{(t-1)}\sqsubseteq \cdots\sqsubseteq \hat{\pmb{\mu}}{}^{(1)}$ for $t>1$. We remark that in general, $\hat{\pmb{\mu}}{}^{(1)}\not\sqsubseteq \hat{\pmb{\mu}}{}^{(0)}$.
Furthermore, there is a bijection between $N_G(v)$ and $N_G(w)$ such that for each $u\in N_G(v)$ and corresponding $u'\in N_G(w)$, $\pmb{\mu}{}^{(t)}_u=\pmb{\mu}{}^{(t)}_{u'}$.
% %  and hence also
% %  $\pmb{\mu}{}^{(1)}_u=\pmb{\mu}{}^{(1)}_{u'}$.
% % %
% %
% % Since $\h\pmb{\mu}}{}^{(0)}\sqsubseteq \pmb{\mu}^{(1)}$, this implies that for every $u\in N_G(v)$ and corresponding $u'\in N_G(w)$,  $\pmb{\mu}{}^{(1)}_u=\pmb{\mu}{}^{(1)}_{u'}$.
% % By our assumption on $\pmb{\mu}^{(1)}$, this implies that
% $\mathbf{L}_{uu}=\mathbf{L}_{u'u'}$ and $\mathbf{R}_{uu}=\mathbf{R}_{u'u'}$. Similarly,
% $\hat{\pmb{\mu}}{}^{(t+1)}_v=\hat{\pmb{\mu}}{}^{(t+1)}_w$  implies
% $\pmb{\mu}{}^{(1)}_v=\pmb{\mu}{}^{(1)}_{w}$ and thus also $\mathbf{L}_{vv}=\mathbf{L}_{ww}$.
By the induction hypothesis we also have that for  every $u\in N_G(v)$
   and corresponding $u'\in N_G(w)$, $\mathbf{F}^{(t)}_{u\bullet}=\mathbf{F}^{(t)}_{u'\bullet}$. It now suffices to observe that
  \begin{align*}
  \mathbf{F}^{(t+1)}_{v\bullet}&=\sigma\Biggl(\mathbf{L}_{vv}\Bigl(\sum_{u\in N_G(v)} \mathbf{R}_{uu}\mathbf{F}^{(t)}_{u\bullet}\Bigr)\mathbf{W}^{(t)}+ q\mathbf{B}^{(t)}_{v\bullet}\Biggr)\\
	 & =\sigma\Biggl(\mathbf{L}_{ww}\Bigl(\!\!\sum_{u'\in N_G(w)}\!\! \mathbf{R}_{u'u'}\mathbf{F}^{(t)}_{u'\bullet}\Bigr)\mathbf{W}^{(t)}+ q\mathbf{B}^{(t)}_{w\bullet}\Biggr)\\
	  &=\mathbf{F}^{(t+1)}_{w\bullet},
\end{align*}
as desired.


(b)~We show the upper bound by WWL by induction on the number of iterations. For $t=0$, we have, as before that $\pmb{\mu}\sqsubseteq \mathbf{F}^{(0)}$. 
We remark, however, that in general $\pmb{\mu}^{(1)}\not\sqsubseteq\pmb{\mu}$.
It is here that the restriction on labeled graphs come into play. More precisely,
$\pmb{\mu}^{(1)}_v=\pmb{\mu}^{(1)}_w$ implies that $d_v=d_w$. Since $(G,\pmb{\mu})$ is consistent with degree information, $d_v=d_w$ implies
$\pmb{\mu}_v=\pmb{\mu}_w$. So, on such graphs, $\pmb{\mu}^{(1)}\sqsubseteq\pmb{\mu}\sqsubseteq\mathbf{F}^{(0)}$.

% Since we consider labeled graphs which are consistent with degree information, and $\pmb{\mu}^{(1)}_v=\pmb{\mu}^{(1)}_w$
% implies $d_v=d_w$, we also h
We next assume that the induction hypothesis holds for $t\geq 0$ and consider $t+1$. We need to show that 
$\pmb{\mu}{}^{(t+2)}_v=\pmb{\mu}{}^{(t+2)}_w$ implies that $\mathbf{F}^{(t+1)}_{v\bullet}=\mathbf{F}^{(t+1)}_{w\bullet}$. By definition,
$\pmb{\mu}^{(t+2)}_v=\pmb{\mu}{}^{(t+2)}_w$ implies
$$
\ldbl \pmb{\mu}{}^{(t+1)}_u \st u \in N_G(v) \rdbl=
 \ldbl \pmb{\mu}{}^{(t+1)}_u \st u \in N_G(w) \rdbl.$$
We remark that this implies that $d_v=d_w$. Since our GNNs belong to ${\cal C}_{\textsl{deg}}$, this in turn implies that $\mathbf{L}_{vv}=\mathbf{L}_{ww}$.
%
% It is readily verified that $\hat{\pmb{\mu}}{}^{(t)}\sqsubseteq \hat{\pmb{\mu}}{}^{(t-1)}\sqsubseteq \cdots\sqsubseteq \hat{\pmb{\mu}}{}^{(1)}$ for $t>1$. We remark that in general, $\hat{\pmb{\mu}}{}^{(1)}\not\sqsubseteq \hat{\pmb{\mu}}{}^{(0)}$.
Furthermore, there is a bijection between $N_G(v)$ and $N_G(w)$ such that
for every $u\in N_G(v)$ and $u'\in N_G(w)$, $\pmb{\mu}{}^{(t+1)}_u=\pmb{\mu}{}^{(t+1)}_{u'}$. This in turn implies that
$d_u=d_{u'}$ for every $u\in N_G(v)$ and corresponding $u'\in N_G(w)$. We note that this holds because $t+1\geq 1$. Since our GNNs belong to ${\cal C}_{\textsl{deg}}$, we have that $\mathbf{R}_{uu}=\mathbf{R}_{u'u'}$ for every
$u\in N_G(v)$ and corresponding $u'\in N_G(w)$. 
% %  and hence also
% %  $\pmb{\mu}{}^{(1)}_u=\pmb{\mu}{}^{(1)}_{u'}$.
% % %
% %
% % Since $\h\pmb{\mu}}{}^{(0)}\sqsubseteq \pmb{\mu}^{(1)}$, this implies that for every $u\in N_G(v)$ and corresponding $u'\in N_G(w)$,  $\pmb{\mu}{}^{(1)}_u=\pmb{\mu}{}^{(1)}_{u'}$.
% % By our assumption on $\pmb{\mu}^{(1)}$, this implies that
% $\mathbf{L}_{uu}=\mathbf{L}_{u'u'}$ and $\mathbf{R}_{uu}=\mathbf{R}_{u'u'}$. Similarly,
% $\hat{\pmb{\mu}}{}^{(t+1)}_v=\hat{\pmb{\mu}}{}^{(t+1)}_w$  implies
% $\pmb{\mu}{}^{(1)}_v=\pmb{\mu}{}^{(1)}_{w}$ and thus also $\mathbf{L}_{vv}=\mathbf{L}_{ww}$.
By the induction hypothesis we also for every $u\in N_G(v)$
   and corresponding $u'\in N_G(w)$, $\mathbf{F}^{(t)}_{u\bullet}=\mathbf{F}^{(t)}_{u'\bullet}$. It now suffices to observe that
  \begin{align*}
  \mathbf{F}^{(t+1)}_{v\bullet}&=\sigma\Biggl(\mathbf{L}_{vv}\Bigl(\sum_{u\in N_G(v)} \mathbf{R}_{uu}\mathbf{F}^{(t)}_{u\bullet}\Bigr)\mathbf{W}^{(t)}+ q\mathbf{B}^{(t)}_{v\bullet}\Biggr)\\
	 & =\sigma\Biggl(\mathbf{L}_{ww}\Bigl(\!\!\sum_{u'\in N_G(w)}\!\! \mathbf{R}_{u'u'}\mathbf{F}^{(t)}_{u'\bullet}\Bigr)\mathbf{W}^{(t)}+ q\mathbf{B}^{(t)}_{w\bullet}\Biggr)\\
	  &=\mathbf{F}^{(t+1)}_{w\bullet},
\end{align*}
as desired.
\end{proof}

\openprob{I do not know whether the assumption on the initial labeling being consistent with degree information can be dropped in (b).}

\tobdone{Show example for WWL bounded with factor 1 but not bounded by WWL, no factor.}

\tobdone{Show example WL vs WLL bounded.}

\floris{I would like to make a connection with the work by Grohe. Note that WWL on unlabeled graphs corresponds to WL (Grohe paper). The above upper bounds using WWL can thus be seen as a generalisation of their unlabeled case to the labeled case. We recover the WWL (and thus WL) upper bound of $\sigma(\mathbf{A}\mathbf{F}^{(t-1)}\mathbf{W}^{(t-1)}-\mathbf{J})$ in their paper.}



% Intuitively, when $\pmb{\ell}^{(k)}\sqsubseteq\hat{\pmb{\ell}}$ holds for any labeled graph $(G,\pmb{\ell})$, then this implies that the entries in $\mathbf{L}$ and $\mathbf{R}$ are functionally determined by the so-called \textit{$k$-iterated degrees} of vertices. The $k$-iterated degree of a vertex $v$ is the label assigned by $k$-steps of  WL on the uniformily labeled graph $(G,\pmb{a})$ in which every vertex is assigned the label $a$.
% Similarly, $\pmb{\ell}^{(k')}_v=\pmb{\ell}^{(0)}_w\Rightarrow \mathbf{R}_{vv}=\mathbf{R}_{ww}$ implies that the entries in $\mathbf{R}$ are functionally determined by the $k'$-iterated degrees of vertices. We illustrate this by the following example.
% \begin{example}
% 	\floris{Show example for say, $k=2$, and $k=2$ and $k'=1$, if we think this is interesting.}
% \end{example}

%!TEX root =main.tex

\section{Lower bounding the expressive power}\label{sec:lowerb}

\floris{I think we need to show a lower bound for two classes. See later.
We may need to revise this section accordingly.}

We start this section by showing Theorem~\ref{thm:lowerb_general}.
More precisely, we show that for any given labeled graph
$(G,\pmb{\ell})$, there exist parameters $p$, $q$ and weight matrices $\mathbf{W}^{(t)}$, such that when $\mathbf{F}^{(0)}\equiv \hat{\pmb{\ell}}{}^{(t)}$, then for all $t\geq 0$, $\hat{\pmb{\ell}}{}^{(t)}\equiv \mathbf{F}^{(t)}$ where
$$
\mathbf{F}^{(t)}:=\sigma\left(\mathbf{L}(\mathbf{A}+p\mathbf{I})\mathbf{R}\mathbf{F}^{(t-1)}\mathbf{W}^{(t-1)} + q\mathbf{J}\right).$$ 

To show this, we follow the same strategy as in~\cite{grohewl}.
Crucial in establishing the lower bound are the following definitions (see also~\cite{grohewl}):
\begin{definition}[row independent modulo equality]\label{def:label2}\normalfont
A matrix $\mathbf{F}$ is \textit{row-independent modulo equality} if the unique row vectors in $\mathbf{F}$ are linearly independent. \qed
\end{definition}

\begin{definition}[a good matrix]\label{def:label3}\normalfont
A matrix $\mathbf{F}$ is \textit{good relative to an other matrix} $\mathbf{F}'$ if $\mathbf{F}\equiv \mathbf{F}'$ and $\mathbf{F}$ is row-independent modulo equality.\qed
\end{definition}

To show the lower bound we will inductively show that when $\mathbf{F}^{(t-1)}$ is good relative to $\hat{\pmb{\ell}}{}^{(t-1)}$, then. there exists a weight matrix $\mathbf{W}^{(t-1)}$, such that 
$\mathbf{F}^{(t)}$ is also good relative to $\hat{\pmb{\ell}}{}^{(t)}$. This clearly suffices to infer the lower bound. 

The induction hypothesis will be shown in a number of steps, by gradually adding the different components in the GNN architecture. We start by showing that multiplication with $\mathbf{R}$ preserves the goodness of the feature matrix.

%
% We first show, by induction on $t$, that when $\mathbf{F}^{(t)}$ is good relative to $\hat{\pmb{\ell}}^{(t)}$, then there exists a constant $p$ and constants $q^{(t)}$ and weight matrices
% $\mathbf{W}^{(t)}$, such that
% $$
% \mathbf{F}^{(t+1)}:=\sigma\Bigl(\mathbf{L}(\mathbf{A}+p\mathbf{I})\mathbf{R}\mathbf{F}^{(t)}\mathbf{W}^{(t)}+q^{(t)}\mathbf{B}\Bigr)
% $$
% is also good relative to $\hat{\pmb{\ell}}^{(t+1)}$. In a next step, we show that we can choose $q^{(t)}$ uniformly across all layers.


\begin{lemma}\label{lem:rightgood}
	For any $t\geq 0$,
	If $\mathbf{F}^{(t)}$ is good relative to $\hat{\pmb{\ell}}{}^{(t)}$, then $\mathbf{R}\mathbf{F}^{(t)}$ is also good relative to $\hat{\pmb{\ell}}{}^{(t)}$.
\end{lemma}
\begin{proof}
By assumption, we have that $\mathbf{F}^{(t)}_{v\bullet}=\mathbf{F}^{(t)}_{w\bullet}$ implies 
$\hat{\pmb{\ell}}{}^{(t)}_{v}=\hat{\pmb{\ell}}{}^{(t)}_w$. Furthermore, since $\hat{\pmb{\ell}}{}^{(t)}\sqsubseteq\hat{\pmb{\ell}}{}^{(0)}$,
$\mathbf{F}^{(t)}_{v\bullet}=\mathbf{F}^{(t)}_{w\bullet}$ implies  $\hat{\pmb{\ell}}{}^{(0)}_{v}=\hat{\pmb{\ell}}{}^{(0)}_w$. This in turn means that 
$\mathbf{R}_{vv}=\mathbf{R}_{ww}$. Hence, $\mathbf{F}^{(t)}_{v\bullet}=\mathbf{F}^{(t)}_{w\bullet}$  implies that 
$\mathbf{R}_{vv}\mathbf{F}^{(t)}_{v\bullet}=\mathbf{R}_{ww}\mathbf{F}^{(t)}_{w\bullet}$.
For the converse, suppose that $\mathbf{R}_{vv}\mathbf{F}^{(t)}_{v\bullet}=\mathbf{R}_{ww}\mathbf{F}^{(t)}_{w\bullet}$ holds.
Then, $\mathbf{F}^{(t)}_{v\bullet}=\mathbf{F}^{(t)}_{w\bullet}$ since otherwise we would have to
distinct rows $\mathbf{F}^{(t)}_{v\bullet}$ and $\mathbf{F}^{(t)}_{w\bullet}$ which are linearly dependent. This contradicts the assumption that $\mathbf{F}^{(t)}$ is row-independent modulo equality.
Hence, $\mathbf{R}\mathbf{F}^{(t)}\equiv\mathbf{F}^{(t)}\equiv\hat{\pmb{\ell}}{}^{(t)}$. Using the same reasoning as above, $\mathbf{R}\mathbf{F}^{(t)}$ is also row-independent modulo equality. Indeed, suppose that the unique rows in $\mathbf{R}\mathbf{F}^{(t)}$ are linearly dependent. Then, this implies that also the unique rows in $\mathbf{F}^{(t)}$ are linearly dependent, a contradiction. 
\end{proof}

We next consider multiplication with $\mathbf{A}+q\mathbf{I}$. 

\begin{lemma}\label{lem:findingp}
If $\mathbf{F}^{(t-1)}$ is good relative to $\hat{\pmb{\ell}}{}^{(t-1)}$, then there exists a  weight matrix $\mathbf{U}^{(t-1)}$ such that 
$\mathbf{G}^{(t)}:=(\mathbf{A}+p\mathbf{I})\mathbf{F}^{(t-1)}\mathbf{U}^{(t-1)}$ is equivalent to 
$\hat{\pmb{\ell}}{}^{(t)}$. Furthermore, this holds for any $p$ such that $0<p<1$.
\end{lemma}
\begin{proof}
By assumption, $\mathbf{F}^{(t-1)}$ is row-independent modulo equality.
Suppose that $\mathbf{F}^{(t-1)}$ is an $n\times f$-matrix, with $n$ the number of vertices in $(G,\pmb{\ell})$.
Let 
$\widetilde{\mathbf{F}}^{(t-1)}$ be the $u\times f$-matrix consisting of the unique rows in $\mathbf{F}^{(t-1)}$. Because the rows in $\widetilde{\mathbf{F}}^{(t-1)}$ are linearly independent,
there exists a $f\times u$-matrix $\mathbf{U}^{(t-1)}$ such that $\widetilde{\mathbf{F}}^{(t-1)}\mathbf{U}^{(t-1)}=\mathbf{I}_{u\times u}$. By induction $\mathbf{F}^{(t-1)}\equiv\hat{\pmb{\ell}}{}^{(t-1)}$. Let $\Sigma^{(t-1)}$ be the set of  labels assigned by $\hat{\pmb{\ell}}{}^{(t-1)}$ to vertices $v\in V$. 
For any $v\in V$ and $c\in\Sigma^{(t-1)}$, we denote by $\mathbf{F}^{(t-1)}_{v\bullet}\sim c$ that
$\hat{\pmb{\ell}}{}^{(t)}_v=c$.
Then for each $v\in V$ and $c\in \Sigma^{(t-1)}$ we have:
\begin{align*}
(\mathbf{A}\mathbf{F}^{(t-1)}\mathbf{M}^{(t-1)})_{vc}&=|\{u\in N_G(v)\mid \mathbf{F}^{(t-1)}_{u\bullet}\sim c\}|.
\intertext{Furthermore,} 
p\mathbf{I}(\mathbf{F}^{(t-1)}\mathbf{M}^{(t-1)})_{vc}&=p\delta_{vc},
\end{align*}
with $\delta_{vc}=1$ if $\mathbf{F}^{(t-1)}_{v\bullet}\sim c$ and $\delta_{vc}=0$ otherwise.

We next show that $\mathbf{G}^{(t)}:=(\mathbf{A}+p\mathbf{I})\mathbf{F}^{(t-1)}\mathbf{M}^{(t-1)}$ is equivalent to $\hat{\pmb{\ell}}{}^{(t)}$. Indeed, suppose that 
$\hat{\pmb{\ell}}{}^{(t)}_v=\hat{\pmb{\ell}}{}^{(t)}_w$. Then,
$\hat{\pmb{\ell}}{}^{(t-1)}_v=\hat{\pmb{\ell}}{}^{(t-1)}_w$ and for every $u\in N_G(v)$ and corresponding $u'\in N_G(w)$, $\hat{\pmb{\ell}}{}^{(t-1)}_u=\hat{\pmb{\ell}}{}^{(t-1)}_{u'}$. Since $\mathbf{F}^{(t-1)}\equiv\hat{\pmb{\ell}}{}^{(t-1)}$ this implies for any $c\in \Sigma^{(t-1)}$,
 $\delta_{vc}=\delta_{wc}$ and 
$|\{u\in N_G(v)\mid \mathbf{F}^{(t-1)}_{u\bullet}\sim c\}|=
|\{u\in N_G(w)\mid \mathbf{F}^{(t-1)}_{u\bullet}\sim c\}|$. In other words,
$\mathbf{G}^{(t)}_{v\bullet}=\mathbf{G}^{(t)}_{w\bullet}$. Conversely,
suppose that $\mathbf{G}^{(t)}_{v\bullet}=\mathbf{G}^{(t)}_{w\bullet}$. Then, for every $c\in \Sigma^{(t-1)}$,
$|\{u\in N_G(v)\mid \mathbf{F}^{(t-1)}_{u\bullet}\sim c\}|+p\delta_{vc}=
|\{u\in N_G(w)\mid \mathbf{F}^{(t-1)}_{u\bullet}\sim c\}|+p\delta_{wc}$. We next distinguish between a number of cases. Suppose first that $\delta_{vc}=\delta_{wc}$. Then, we must have that $|\{u\in N_G(v)\mid \mathbf{F}^{(t-1)}_{u\bullet}\sim c\}|=
|\{u\in N_G(w)\mid \mathbf{F}^{(t-1)}_{u\bullet}\sim c\}|$. Since $\mathbf{F}^{(t-1)}\equiv\hat{\pmb{\ell}}^{(t-1)}$ this implies that $\hat{\pmb{\ell}}{}^{(t-1)}_v=\hat{\pmb{\ell}}{}^{(t-1)}_w$ and for every $u\in N_G(v)$ and corresponding $u'\in N_G(w)$, $\hat{\pmb{\ell}}{}^{(t-1)}_u=\hat{\pmb{\ell}}{}^{(t-1)}_{u'}$. In other words, 
$\hat{\pmb{\ell}}{}^{(t)}_v=\hat{\pmb{\ell}}{}^{(t)}_w$. By contrast, if $\delta_{vc}\neq \delta_{wc}$ then either $|\{u\in N_G(v)\mid \mathbf{F}^{(t-1)}_{u\bullet}\sim c\}|=
|\{u\in N_G(w)\mid \mathbf{F}^{(t-1)}_{u\bullet}\sim c\}|+p$ or
$|\{u\in N_G(v)\mid \mathbf{F}^{(t-1)}_{u\bullet}\sim c\}|+p=
|\{u\in N_G(w)\mid \mathbf{F}^{(t-1)}_{u\bullet}\sim c\}|$. This, however, is impossible for any $p$ such that $0<p<1$ because  $|\{u\in N_G(w)\mid \mathbf{F}^{(t-1)}_{u\bullet}\sim c\}|$ and $|\{u\in N_G(w)\mid \mathbf{F}^{(t-1)}_{u\bullet}\sim c\}|$ are integers.
\end{proof}


We remark that the previous lemma only guarantees that $\mathbf{H}^{(t)}$ is equivalent to $\hat{\pmb{\ell}}{}^{(t)}$. We show later how row-independence modulo equality is ensured.
Furthermore, the lemma is stated in terms of $\mathbf{F}^{(t)}$ instead of $\mathbf{R}\mathbf{F}^{(t)}$.
From Lemma~\ref{lem:rightgood} we know, however, that $\mathbf{F}^{(t)}\equiv\mathbf{R}\mathbf{F}^{(t)}$
so we can apply Lemma~\ref{lem:findingp} to $\mathbf{R}\mathbf{F}^{(t)}$ as well.

We next consider multiplication with $\mathbf{L}$. In the following, we let $\mathbf{G}^{(t)}:=(\mathbf{A}+p\mathbf{I})\mathbf{R}\mathbf{F}^{(t-1)}\mathbf{U}^{(t-1)}$.

% \todo{The lemma below is new. It allows to use a simplified bias matrix. Our previous argument required the bias matrix to be $\mathbf{L}\mathbf{J}$. We now only have $\mathbf{J}$. This lemma needs to be checked carefully ;-)}
\begin{lemma}\label{lemma:findp}
There exists a constant $m_p$, only dependent on $\mathbf{L}$ and the number of vertices $V$ in $G$, such that for all $p$ for which $m_p<p<1$, $\mathbf{L}\mathbf{G}^{(t)}$ is equivalent to $\hat{\pmb{\ell}}{}^{(t)}$.
\end{lemma}
\begin{proof}
	We know from Lemma~\ref{lem:findingp} that
 $\mathbf{G}^{(t)}\equiv\hat{\pmb{\ell}}{}^{(t)}$ for all $p$, $0<p<1$.
	Suppose that $\hat{\pmb{\ell}}{}^{(t)}_v=\hat{\pmb{\ell}}{}^{(t)}_w$, or equivalently, that	$\mathbf{G}^{(t)}_{v\bullet}=\mathbf{G}^{(t)}_{w\bullet}$. This implies that
	  $\hat{\pmb{\ell}}{}^{(0)}_v=\hat{\pmb{\ell}}{}^{(0)}_w$. In particular, we must have that $\mathbf{L}_{vv}=\mathbf{L}_{ww}$. This in turn implies
$\mathbf{L}_{vv}\mathbf{G}^{(t)}_{v\bullet}=\mathbf{L}_{ww}\mathbf{G}^{(t)}_{w\bullet}$.

We next describe a procedure of how to obtain $m_p$. Later on, we define $m_p$ directly
from $\mathbf{L}$ and $n$.

Initially, $m_p=0$. Then, we update $m_p$ as long as
$\mathbf{L}\mathbf{G}^{(t)}\not\equiv\hat{\pmb{\ell}}{}^{(t)}$. We have just seen that
$\hat{\pmb{\ell}}{}^{(t)}\sqsubseteq \mathbf{L}\mathbf{G}^{(t)}$ for all $p$, $0<p<1$.
Hence, we update $m_p$ as long as $\mathbf{L}\mathbf{G}^{(t)}\not\sqsubseteq\hat{\pmb{\ell}}{}^{(t)}$.
The latter implies that
there exists two vertices $v,w\in V$ such that
\begin{equation}
\mathbf{L}_{vv}\mathbf{G}^{(t)}_{v\bullet}=\mathbf{L}_{ww}\mathbf{G}^{(t)}_{w\bullet},\label{eq:contra}
\end{equation}
but $\hat{\pmb{\ell}}{}^{(t)}_v\neq \hat{\pmb{\ell}}{}^{(t)}_w$, or equivalently, $\mathbf{G}^{(t)}_{v\bullet}\neq \mathbf{G}^{(t)}_{w\bullet}$.
Clearly, in this case $\mathbf{L}_{vv}\neq\mathbf{L}_{ww}$. Hence, the equality~(\ref{eq:contra}) implies that for every $c\in\Sigma^{(t-1)}$, either
(i)~$\mathbf{G}^{(t)}_{vc}=0=\mathbf{G}^{(t)}_{wc}$; or (ii)~ $0\neq \mathbf{G}^{(t)}_{vc}\neq \mathbf{G}^{(t)}_{wc}\neq 0$. Here, $\Sigma^{(t-1)}$ is the set of labels as used in the proof of  Lemma~\ref{lem:findingp}. Furthermore, for all $c\in\Sigma^{(t-1)}$ for which (ii) holds, 
\begin{equation}
\frac{\mathbf{G}^{(t)}_{vc}}{\mathbf{G}^{(t)}_{wc}}=\frac{\mathbf{L}_{ww}}{\mathbf{L}_{vv}}\neq 1.\label{eq:ratio}
\end{equation}
Let $\alpha:=\frac{\mathbf{L}_{ww}}{\mathbf{L}_{vv}}$.
We also observe that, since $p$ occurs in precisely one entry in $\mathbf{G}^{(t)}_{v\bullet}$ and $\mathbf{G}^{(t)}_{w\bullet}$, there is at least one $c\in\Sigma^{(t-1)}$ for which (ii) holds. Furthermore, the proof of  Lemma~\ref{lem:findingp} tells precisely how the elements $\mathbf{G}^{(t)}_{vc}$ and $\mathbf{G}^{(t)}_{wc}$ look like. That is, these entries are of the form 
$i$ or $i+p$ for $i\in[1,n]$ with $n=|V|$. If we consider equation~(\ref{eq:ratio}) for
the entries containing $p$ we are in one of the following three cases:
$$
\text{(a) } \frac{i+p}{j}=\alpha
\text{; (b) } \frac{i}{j+p}=\alpha
\text{; or (c) } \frac{i+p}{j+p}=\alpha
$$
for some $i,j\in[1,n]$. Hence, in case (a)~$p=\alpha j- i$, in case (b)~$p=\frac{i-\alpha j}{\alpha}$, and in case (c)~$p=\frac{\alpha(j-i)}{1-\alpha}$. It now suffices to update
$m_p=\max\{m_p,\alpha j- i,\frac{i-\alpha j}{\alpha},\frac{\alpha(j-i)}{1-\alpha}\}$ and
hence for any $p>m_p$, equation~(\ref{eq:ratio}) is not satisfied anymore for $v$ and $w$.
We proceed in this way, as long is there exists a pair of vertices  $v,w\in V$ for which equation~\ref{eq:contra} holds  but $\hat{\pmb{\ell}}{}^{(t)}_v\neq \hat{\pmb{\ell}}{}^{(t)}_w$. Note that $m_p$ has been updated for a particular pair of vertices, that pair will not need to considered anymore. The final $m_p$ is the value when
for all vertices $v,w\in V$, equation~\ref{eq:contra} implies $\hat{\pmb{\ell}}{}^{(t)}_v= \hat{\pmb{\ell}}{}^{(t)}_w$. In other words, when $\mathbf{L}\mathbf{G}^{(t)}\sqsubseteq\hat{\pmb{\ell}}{}^{(t)}$ holds.

It should be clear now that we can define $m_p$ more roughly, as follows.
First, define $L':=\{\frac{\mathbf{L}_{vv}}{\mathbf{L}_{ww}}\mid \mathbf{L}_{vv}\neq\mathbf{L}_{ww}, v,w\in V\}$. Then, consider
\begin{align*}
P_1&:=\left\{ \alpha j -i \mid 0\leq \alpha i -j < 1, i,j\in[1,n],\alpha\in L'\right\}\\
P_2&:=\left\{ \frac{i -\alpha j}{\alpha} \mid 0\leq \frac{i -\alpha j}{\alpha} < 1, i,j\in[1,n],\alpha\in L'\right\}\\
P_3&:=\left\{ \frac{\alpha(j-i)}{1-\alpha} \mid 0\leq \frac{\alpha(j-i)}{1-\alpha} < 1, i,j\in[1,n],\alpha\in L'\right\}
\end{align*}
and let $m_p=\max\{P_1\cup P_2\cup P_3\cup\{0\}\}$. In other words, we just consider all possible $p$ values as determined before that could lead to 
$\mathbf{L}\mathbf{G}^{(t)}\not\sqsubseteq\hat{\pmb{\ell}}{}^{(t)}$ and eliminate these possibility by choosing $p$ large enough, as explained earlier.
%
% We will sho
% consider the set $L':=\{\frac{\mathbf{L}_{vv}}{\mathbf{L}_{ww}}\mid \mathbf{L}_{vv}\neq\mathbf{L}_{ww}, v,w\in V\}$. We remark this, by assumption $L_{vv}\neq 0$ for any $v\in V$ so we do not have zero denominators. We define $m_p$ as follows.
% Initially, we set $m_p:0$. Then,
% suppose that
% $\mathbf{L}_{vv}\mathbf{G}^{(t)}_{v\bullet}=\mathbf{L}_{ww}\mathbf{G}^{(t)}_{w\bullet}$ for some
% vertices $v,w\in V$. We will show how to change $m_p$ such for all $p$ satisfying
% $m_p<p<1$, $\mathbf{L}_{vv}\mathbf{G}^{(t)}_{v\bullet}\neq \mathbf{L}_{ww}\mathbf{G}^{(t)}_{w\bullet}$.
% (Recall from the proof of  Lemma~\ref{lem:findingp} that every $\mathbf{G}^{(t)}_{v\bullet}$
% has at least one entry containing $p$.)
%
%
%
%
% We next consider $M'=\{\frac{\mathbf{G}^{(t)}_{wc}}{\mathbf{G}^{(t)}_{vc}}\mid \mathbf{G}^{(t)}_{vc}\neq \mathbf{G}^{(t)}_{wc},\mathbf{G}^{(t)}_{vc}\neq 0, v,w\in V,c\in\Sigma^{(t-1)}\}$, where $\Sigma^{(t-1)}$ is the set of labels as used in the proof of  Lemma~\ref{lem:findingp}. From that proof we also know precisely how the ratio's in $M'$ look like. That is, every such a ratio is of the form
% $\frac{i+p}{j}$, $\frac{i}{j+p}$ or $\frac{i+p}{j+p}$ for $i,j\in[0,|V|]$.
%
%  Let
% $G$ be the collection of all these fractions and choose $p$ such that $G\cap L'=\emptyset$
% (almost any $p$ will do.). Suppose now that $\hat{\pmb{\ell}}^{(t)}_v\neq \hat{\pmb{\ell}}^{(t)}_w$
% and thus $\mathbf{G}^{(t)}_{v\bullet}\neq \mathbf{G}^{(t)}_{w\bullet}$. Suppose, for the sake of
% contradiction that $\mathbf{L}_{vv}\mathbf{G}^{(t)}_{v\bullet}=\mathbf{L}_{ww}\mathbf{G}^{(t)}_{w\bullet}$. This implies
% that the non-zero entries in $\mathbf{G}^{(t)}_{v\bullet}$ correspond to non-zero (but possibly different)
% entries in $\mathbf{G}^{(t)}_{w\bullet}$. We are guaranteed at least one non-zero entry in $\mathbf{G}^{(t)}_{v\bullet}$ which contains $p$. Assume that this entry is of the form $i+p$. Then,
% the corresponding non-zero entry in $\mathbf{G}^{(t)}_{w\bullet}$ is of the form $j$ or $j+p$. We also know that  $\mathbf{G}^{(t)}_{v\bullet}\neq \mathbf{G}^{(t)}_{w\bullet}$ and $\mathbf{L}_{vv}\mathbf{G}^{(t)}_{v\bullet}=\mathbf{L}_{ww}\mathbf{G}^{(t)}_{w\bullet}$ implies that
% $\mathbf{L}_{vv}\neq\mathbf{L}_{ww}$. Hence, the assumption implies that
% $\frac{\mathbf{L}_{vv}}{\mathbf{L}_{ww}}=\frac{j}{i+p}$ or $\frac{\mathbf{L}_{vv}}{\mathbf{L}_{ww}}=\frac{j+p}{i+p}$. In other words, $G\cap L'\neq\emptyset$, a contradiction.
\end{proof}

At this point, we know that when $\mathbf{F}^{(t-1)}$ is good relative to $\hat{\pmb{\ell}}{}^{(t-1)}$ then there exists a $\mathbf{M}^{(t-1)}$ such that $\mathbf{H}^{(t)}:=\mathbf{L}(\mathbf{A}+p\mathbf{I})\mathbf{R}\mathbf{F}^{(t-1)}\mathbf{M}^{(t-1)}$
is equivalent to $\hat{\pmb{\ell}}{}^{(t)}$, and this for any $p$ such that $m_p<p<1$.
We now come to the point that will ensure that there exists a $q$ and $\mathbf{N}^{(t-1)}$ such that
$$
\mathbf{F}^{(t)}:=\sigma(\mathbf{L}(\mathbf{A}+p\mathbf{I})\mathbf{R}\mathbf{F}^{(t-1)}\underbrace{\mathbf{M}^{(t-1)}\mathbf{N}^{(t-1)}}_{\mathbf{W}^{(t-1)}}+q\mathbf{J})$$
 is good relative to $\hat{\pmb{\ell}}{}^{(t)}$, where $\sigma$ is either the sign or the ReLU activation function. We here follow closely the approach taken in~\cite{grohewl}.
Lemma 9 in~\cite{grohewl} states:

 \begin{lemma}[Lemma 9 in~\cite{grohewl}]\label{lem:signlemma9}
  Let
  $\mathbf{B}\in \Nb^{u\times f}$ be a matrix in which all
  rows are pairwise disjoint.
%  \footnote{I believe that this can be
%  guaranteed in 1-WL}).\todo{G: with our extended features we actually guarantee this for free by adding the 1 column; also, t as dimension is a bad choice\ldots}
  Then there exists a matrix $\mathbf{N}$ 
  such that $\text{\normalfont sign}(\mathbf{BN}-\mathbf{J})$ is
  non-singular.
\end{lemma}
Although~\cite{grohewl} also consider the ReLU activation function, they simulate each application of the ReLu function by two applications of the sign function. As a consequence,
the overall number of layers doubled. A more direct approach can be applied, however. More
precisely, we show the following:
 \begin{lemma}\label{lem:relulemma9}
  Let
  $\mathbf{B}\in \Nb^{u\times f}$ be a matrix in which all
  rows are pairwise disjoint and such that no row consists entirely
  out of zeroes.
%  \footnote{I believe that this can be
%  guaranteed in 1-WL}).\todo{G: with our extended features we actually guarantee this for free by adding the 1 column; also, t as dimension is a bad choice\ldots}
  Then there exists a matrix $\mathbf{N}$ and a constant $q$
  such that $\text{\normalfont ReLU}(\mathbf{BN}-q\mathbf{J})$ is
  non-singular. In fact, if we denote by $M$ the maximal entry in $\mathbf{B}$,
  any $q$ such that  $\frac{M^f-1}{M^f} < q < 1$ will do.
\end{lemma}
\begin{proof}
Let $M$ be the maximal entry in $\mathbf{B}$ and consider the column vector $\mathbf{z}=(1,M,M^2,\ldots,M^{f-1})^{\textsc{t}}$.
Then each entry in $\mathbf{b}=\mathbf{B}\mathbf{z}$ is positive and they are all pairwise distinct. 
Let $\mathbf{P}$ be a permutation matrix in $\Rb^{u\times u}$ such that $\mathbf{b}'=\mathbf{P}\mathbf{b}$ is such that  $\mathbf{b}'=(b_1',b_2',\ldots,b_u')^{\textsc{	t}}\in\Rb^{u\times 1}$ with $ b_1'> b_2'>\cdots > b_u'>0$. 
Consider the $\mathbf{x}=\left(\frac{1}{b_1'},\ldots,\frac{1}{b_u'}\right)\in \Rb^{1\times u}$. Then, for $\mathbf{C}=\mathbf{b}'\mathbf{x}$
$$
\mathbf{C}_{ij}=\frac{b_i'}{b_j'}  \text{ and } \mathbf{C}_{ij}=\begin{cases}  1 & \text{if $i=j$}\\
>1 & \text{if $i<j$}\\
< 1 & \text{if $i>j$}.
\end{cases}
$$
Let $q$ be the greatest value  in $\mathbf{C}$ smaller than $1$.
% G: I think the m instantiated here is not correct
%, i.e., $m=\frac{b_s}{b_1}$.
Consider $\mathbf{E}=\mathbf{C}- q\mathbf{J}$.
Then,
$$
\mathbf{E}_{ij}=\frac{b_i'}{b_j'}- q \text{ and } \mathbf{E}_{ij}=\begin{cases}  1-m & \text{if $i=j$} \\
> 0 & \text{if $i<j$}\\
\leq 0  & \text{if $i>j$}.
\end{cases}
$$
As a consequence,
$$
\text{ReLU}(\mathbf{E})_{ij}=\begin{cases}  1-q & \text{if $i=j$}\\
>0 & \text{if $i<j$}\\
0  & \text{if $i>j$}.
\end{cases}
$$
This is an upper triangular matrix with (nonzero) value $1-q$ on its diagonal. It is therefore non-singular. 
We observe that $\mathbf{Q}\text{ReLU}(\mathbf{E})=\text{ReLU}(\mathbf{Q}\mathbf{E})$ for any row permutation $\mathbf{Q}$. Furthermore, non-singularity is preserved under row permutations and $\mathbf{Q}\mathbf{J}=\mathbf{J}$. Hence, if we define $\mathbf{N}=\mathbf{z}\mathbf{x}$ and use the permutation matrix $\mathbf{P}$:
\begin{align*}
\mathbf{P}\text{ReLU}(\mathbf{B}\mathbf{N}-q\mathbf{J})&=
\text{ReLU}(\mathbf{P}\mathbf{B}\mathbf{z}\mathbf{x}-q\mathbf{P}\mathbf{J})=\text{ReLU}(\mathbf{E}-q\mathbf{J}),
\end{align*}
we have that $\text{ReLU}(\mathbf{B}\mathbf{N}-q\mathbf{J})$ is non-singular, as desired.
%So, the lemma is satisfied by taking $m$ as above and
%%$m=b_s/b_1$ and % G: this still looks wrong
%$\mathbf{X}=\mathbf{z}\mathbf{x}$.
To validate second claim in the statement of the lemma, it suffices to observe that every
in $\mathbf{b'}$ is bounded by $M^f$ and hence every 
$\frac{b_i'}{b_j'}$ in $\mathbf{C}$ such that $\frac{b_i'}{b_j'}<1$ is bounded by
$\frac{M^f-1}{M^f}$. Choosing $q$ larger than this number clearly
ensures the correctness of the above construction.
\end{proof}

We are now almost at the end of the lower bound proof. Indeed,
consider the matrix $\widetilde{\mathbf{H}}{}^{(t)}$ consisting of the unique rows
of $\mathbf{H}^{(t)}:=\mathbf{L}(\mathbf{A}+p\mathbf{I})\mathbf{R}\mathbf{F}^{(t-1)}\mathbf{M}^{(t-1)}$. Suppose that $\widetilde{\mathbf{H}}{}^{(t)}$ is of dimension $u\times f$.
Then, when $\sigma$ is the sign function, Lemma~\ref{lem:signlemma9} implies 
the existence of an $f\times u$-matrix $\mathbf{N}^{(t)}$ such that
$\sigma(\widetilde{\mathbf{H}}{}^{(t)}\mathbf{N}^{(t)}-\mathbf{J})$ is non-singular
and thus is row-independent. As a consequence, 
$\mathbf{F}^{(t)}:=\sigma(\mathbf{H}{}^{(t)}\mathbf{N}^{(t)}-\mathbf{J})$ is 
row-independent modulo equality. Since  $\mathbf{H}{}^{(t)}\equiv\hat{\pmb{\ell}}{}^{(t)}$
we have that there is a bijection between $\widetilde{\mathbf{H}}{}^{(t)}$ and the labels
used by $\hat{\pmb{\ell}}{}^{(t)}$. There are $u$ such labels and these again bijectively
map on $\sigma(\widetilde{\mathbf{H}}{}^{(t)}\mathbf{N}^{(t)}-\mathbf{J})$.
It now easily follows that
$\mathbf{F}^{(t)}\equiv\hat{\pmb{\ell}}{}^{(t)}$ as well.

When the activation function is ReLU, we will apply Lemma~\ref{lem:relulemma9} in precisely the same way. We first remark that $\widetilde{\mathbf{H}}{}^{(t)})$ does not contain rows consisting of zeroes only. Indeed, there is at least one entry that has the $p$ parameter.
\marginpar{We do not self-loop assumption, it seems.}
We note, however, that this lemma provides a different lower bound for $q$ in each layer. It suffices to observe $(\widetilde{\mathbf{H}}{}^{(t)})_{v,c}$ is either of the form $\mathbf{L}_{vv}(i+p)$ or $\mathbf{L}_{vv}(i)$ for some $i\in [1,n]$. This holds for every $t\geq 0$. Hence, we know that for all $t\geq 0$, the elements in 
$\widetilde{\mathbf{H}}{}^{(t)}$ are upper bounded by $\ell(n+1)$ where $\ell=\max\{\mathbf{L}_{vv}\mid v\in V\}$. As a consequence, we can take 
$\frac{(\ell(n+1))^f-1}{(\ell(n+1))^f}$ as the lower bound for $q$, independent of layer under consideration.

This concludes the proof of  Theorem~\ref{thm:lowerb_general}.
%
% are of the $\ell(i+p)$ or $\ell i$ where
% It now suffices to apply this Lemmas for the matrix $\mathbf{B}$ consisting of the unique rows of $\mathbf{H}^{(t)}:=\mathbf{L}(\mathbf{A}+p\mathbf{I})\mathbf{R}\mathbf{F}^{(t-1)}\mathbf{M}^{(t-1)}$. Then, we define  $\mathbf{N}^{(t-1)}:=\mathbf{X}$ and take $q> \frac{M^q-1}{M^q}$
% as given by the lemma.
%
% The Lemma then guarantees that $\mathbf{F}^{(t+1)}$ is row-independent modulo equality. Furthermore,
% we have that $\sigma(\mathbf{H}^{(t)}\mathbf{N}^{(t)}+q^{(t)}\mathbf{J})\equiv \mathbf{H}^{(t)}$ and
% thus $\mathbf{F}^{(t+1)}\equiv\hat{\pmb{\ell}}^{(t+1)}$, as desired.
%
% As anticipated, we finally argue that the parameters $q^{(t)}$ can be chosen uniformly across all layers.
% If we inspect the proof of Lemma~\ref{lem:relulemma9} then the constant $m$ needs to be chosen such that
% it is (i)~smaller than $1$ and (ii)~larger than any ratio $\mathbf{C}_{vw}=\mathbf{b}_{v}/\mathbf{w}$, where $\mathbf{b}$ is obtained based on $\mathbf{B}$ and $M$, an upper bound on the largest number in $\mathbf{B}$.
%
% As mentioned earlier, we apply  Lemma~\ref{lem:relulemma9} for $\mathbf{B}=\mathbf{H}^{(t)}$.
% For each $t\geq 0$, we know that
% $$\mathbf{B}_{vc}=\mathbf{L}_{vv}|\{u\in N_G(v)\mid \mathbf{F}^{(t)}_{v\bullet}\sim c\}|+\delta_{vc}p
% \leq \ell(|V|+1),
% $$
% with $\ell:=\max\{\mathbf{L}_{vv}\mid v\in V\}$. Let $M=\ell(|V|+1)$. Following the proof of Lemma~\ref{lem:relulemma9} we thus have that every element in $\mathbf{H}^{(t)}\mathbf{b}$ is
% of the form $\sum_{i=1}^{n-1}\alpha_i M^i$ and is thus smaller than $M^n$. This implies that the
% largest ratio of elements in $\mathbf{H}^{(t)}\mathbf{b}$, smaller than $1$, is upper bounded by
% $\frac{M^n-1}{M^n}$. In other words,
% \begin{lemma}
% If we choose $q> \frac{(\ell(|V|+1))^n-1}{(\ell(|V|+1))^n}$, then for any $t\geq 0$, there exists a matrix $\mathbf{X}^{(t)}$ such that $\text{ReLU}(\tilde{\mathbf{H}^{(t)}}\mathbf{X}^{(t)}-q\mathbf{J})$
% is non-singular, where $\tilde{\mathbf{H}^{(t)}}$ denotes the matrix consisting of unique rows from
% $\mathbf{H}^{(t)}$ and $\ell=\max\{\mathbf{L}_{vv}\mid v\in V\}$.
% \end{lemma}
%


\floris{The lower bound so far do not cover the two GCN cases. Question: 
Can the lower bound be established also for 
$$
\mathbf{F}^{(t)}:=\sigma((\mathbf{L}\mathbf{A}\mathbf{R}+p\mathbf{I})\mathbf{F}^{(t-1)}\mathbf{W}^{(t-1)}+q\mathbf{J})??
$$. I believe what is written below works. In fact, I think we can use this approach also for the other lower bound. It may simplify matter. {\bf FIRST, please check whether I didn't make a mistake below. Of perhaps there is a simpler way of getting the lower bound based on our previous result?}}


We note that Theorem~\ref{thm:lowerb_general} established a lower bound for 
architecture of the form 
$$
\mathbf{F}^{(t)}:=\sigma\left(\mathbf{L}(\mathbf{A}+p\mathbf{I})\mathbf{R}\mathbf{F}^{(t-1)}\mathbf{W}^{(t-1)} + q\mathbf{J}\right).$$
As such, it does not imply a lower bound for architectures for the 1-GCN and 1-GCNs architectures. To resolve this, we show that following:
\begin{theorem}\label{thm:GCNlowerb}
The class of GNNs of the form
\begin{equation}
\mathbf{F}^{(t)}:=\sigma\left((\mathbf{L}\mathbf{A}\mathbf{R}+p\mathbf{I})\mathbf{F}^{(t-1)}\mathbf{W}^{(t-1)} + q\mathbf{J}\right), \label{eq:architecture_lb_GCN}
\end{equation}
is WL-strong, where WL now starts from $(G,\hat{\pmb{\ell}})$. Here, $\sigma$ can be either the sign or ReLU activation function.
\end{theorem}
\begin{proof}
Looking at the proof of Theorem~\ref{thm:lowerb_general}, it suffices to show the following. Let $\mathbf{F}^{(0)}\equiv\hat{\pmb{\ell}}$ and define for $t>0$:
\begin{align*}
\mathbf{G}^{(t)}&:=(\mathbf{L}\mathbf{A}\mathbf{R}+p\mathbf{I})\mathbf{F}^{(t-1)}\\
\mathbf{F}^{(t)}&:=\sigma(\mathbf{G}^{(t)}\mathbf{W}^{(t-1)}+q\mathbf{J}).
\end{align*}
Then, if we assume that $\mathbf{F}^{(t-1)}$ is good for $\hat{\pmb{\ell}}{}^{(t-1)}$ and if we can show that $\mathbf{G}^{(t)}\equiv \hat{\pmb{\ell}}{}^{(t-1)}$, then 
Lemmas~\ref{lem:signlemma9} and
~\ref{lem:relulemma9} can be applied to find $\mathbf{W}^{(t-1)}$ and $q$ such that $\mathbf{F}^{(t)}$ is good for  $\hat{\pmb{\ell}}{}^{(t-1)}$.

We observe that the upper bound proof of Theorem~\ref{thm:generalbound}  implies that  $\hat{\pmb{\ell}}{}^{(t)}\sqsubseteq \mathbf{G}^{(t)}$. It thus suffices to show that $\hat{\pmb{\ell}}{}^{(t)}\sqsubseteq \mathbf{G}^{(t)}$.


Suppose, for the sake of contradiction, that there exists two vertices $v,w\in V$ such that 
\begin{equation}
\hat{\pmb{\ell}}{}^{(t)}_v\neq\hat{\pmb{\ell}}{}^{(t)}_w \text{ and }
\mathbf{G}^{(t)}_{v\bullet}=\mathbf{G}^{(t)}_{w\bullet}. \label{eq:contra}
\end{equation} We show that we can set the parameter $p$ such that this cannot happen. We follow a similar approach as in the proof of Lemma~\ref{lemma:findp}. That is, we find a lower bound $m_p$ such that for all $p$ such that $m_p<p<1$, the condition~(\ref{eq:contra}) does not hold for any pair $v,w\in V$. In other words, for such $p$,  $\hat{\pmb{\ell}}{}^{(t)}\sqsubseteq \mathbf{G}^{(t)}$.

We define $m_p$ again in a procedural way. Initially, $m_p:=0$. Then as long as there are vertices $v$ and $w$ such that~(\ref{eq:contra}) holds, we update $m_p$, as follows. Let $v$ and $w$ be two distinct vertices for which~(\ref{eq:contra}) holds.
Then, $\hat{\pmb{\ell}}{}^{(t)}_v\neq\hat{\pmb{\ell}}{}^{(t)}_w$ can mean two things: (i)~either $\hat{\pmb{\ell}}{}^{(t-1)}_v\neq\hat{\pmb{\ell}}{}^{(t-1)}_w$;
or (ii)~~$\hat{\pmb{\ell}}{}^{(t-1)}_v=\hat{\pmb{\ell}}{}^{(t-1)}_w$ and
$$
\ldbl \hat{\pmb{\ell}}{}^{(t-1)}_{u} \st u \in N_G(v) \rdbl\neq
\ldbl \hat{\pmb{\ell}}{}^{(t-1)}_{u} \st u \in N_G(w) \rdbl.
$$
By assumption $\mathbf{F}^{(t-1)}$ is good for $\hat{\pmb{\ell}}{}^{(t-1)}$.
In particular, if we consider the unique row vectors in  $\mathbf{F}^{(t-1)}$, then these are linearly independent. Let us denote the unique row vectors in $\mathbf{F}^{(t-1)}$ by $\mathbf{F}_1,\ldots,\mathbf{F}_s$ for some $s$.

Suppose that we are in case (i). Then,  $\hat{\pmb{\ell}}{}^{(t-1)}_v\neq\hat{\pmb{\ell}}{}^{(t-1)}_w$ implies that
$\mathbf{F}^{(t-1)}_{v\bullet}\neq \mathbf{F}^{(t-1)}_{w\bullet}$. We can assume, wlog, that $\mathbf{F}^{(t-1)}_{v\bullet}=\mathbf{F_1}$ and
$\mathbf{F}^{(t-1)}_{w\bullet}=\mathbf{F_2}$. It should be clear from the definition of $\mathbf{G}^{(t)}$ that every of its rows is a linear combination of 
the unique row vectors $\mathbf{F}_1,\ldots,\mathbf{F}_s$. More specifically:
\begin{align*}
\mathbf{G}^{(t)}_{v\bullet}&:=(\alpha_1+p)\mathbf{F}_1+ \alpha_2\mathbf{F}_2+ \sum_{i=3}^s \alpha_i\mathbf{F}_i\\
%\intertext{and similarly,} 
\mathbf{G}^{(t)}_{w\bullet}&=\beta_1\mathbf{F}_1+ (\beta_2+p)\mathbf{F}_2+ \sum_{i=3}^s \beta_i\mathbf{F}_i,
\end{align*}
for some constants $\alpha_i$ and $\beta_i$, for $i\in[1,s]$.
We recall that the vertices $v$ and $w$ are such that $\mathbf{G}^{(t)}_{v\bullet}=\mathbf{G}^{(t)}_{w\bullet}$. This in turn implies that
$$
(\alpha_1+p-\beta_1)\mathbf{F}_1 + (\alpha_2-\beta_2-p)\mathbf{F}_2 +\sum_{i=3}^s (\alpha_i-\beta_i)\mathbf{F}_s=0.
$$
So, unless all these coefficients are zero, we have that $\mathbf{F}_1,\ldots,\mathbf{F}_s$
are not linearly independent. We observe that the $\alpha_i$'s and $\beta_i$'s do not depend on $p$. Hence, if we take  $p\neq \beta_1-\alpha_1$ and/or $p\neq \alpha_2-\beta_2$ then $\mathbf{G}^{(t)}_{v\bullet}$ must become different from $\mathbf{G}^{(t)}_{w\bullet}$. We now update $m_p$ accordingly. That is,
we define the new $m_p$ as
$$
\begin{cases}
\max\{m_p,\beta_1-\alpha_1,\alpha_2-\beta_2\} & \text{if $\beta_1-\alpha_1$ and $\alpha_2-\beta_2$ are smaller than $1$}\\
\max\{m_p,\beta_1-\alpha_1\} & \text{if $\beta_1-\alpha_1<1$ and $\alpha_2-\beta_2>1$}\\
\max\{m_p,\alpha_2-\beta_2\} & \text{if $\beta_1-\alpha_1>1$ and $\alpha_2-\beta_2<1$}\\
m_p &\text{otherwise (no update)}
\end{cases}.
$$
We can do these as long as we find vertices $v$ and $w$ such that~(\ref{eq:contra}) holds and such that we are in case (i). At the end of this process, we have ensure that for any $p$, such that $m_p < p< 1$, if (\ref{eq:contra}) holds for two vertices $v$ and $w$, we must be in case (ii).

It remains to rule out case (ii). Recall that for case (ii), we have $\hat{\pmb{\ell}}{}^{(t-1)}_v=\hat{\pmb{\ell}}{}^{(t-1)}_w$.
Using the same notation as above, we may assume that $\mathbf{F}^{(t-1)}_{v\bullet}=\mathbf{F_1}=\mathbf{F}^{(t-1)}_{w\bullet}$. In case (ii), however, we have that
$
\ldbl \hat{\pmb{\ell}}{}^{(t-1)}_{u} \st u \in N_G(v) \rdbl\neq
\ldbl \hat{\pmb{\ell}}{}^{(t-1)}_{u} \st u \in N_G(w) \rdbl
$.
That is, there must exist a label -- say ``$a$'' -- assigned by $\hat{\pmb{\ell}}{}^{(t-1)}$ that does not occurs the same number of times in the neighborhoods of $v$ and $w$. Recall that $\mathbf{F}^{(t-1)}\equiv \hat{\pmb{\ell}}{}^{(t-1)}$. Hence, the label $a$ uniquely corresponds with one of the unique rows in $\mathbf{F}^{(t-1)}$. Assume that label ``$a$'' corresponds to $\mathbf{F}_2$. We develop $\mathbf{G}^{(t)}_{v\bullet}$ and $\mathbf{G}^{(t)}_{w\bullet}$ again as linear combinations of $\mathbf{F}_1,\ldots,\mathbf{F}_s$. That is,
\begin{align*}
\mathbf{G}^{(t)}_{v\bullet}&:=(\alpha_1+p)\mathbf{F}_1+ \alpha_2\mathbf{F}_2+ \sum_{i=3}^s \alpha_i\mathbf{F}_i\\
%\intertext{and similarly,} 
\mathbf{G}^{(t)}_{w\bullet}&=(\beta_1+p)\mathbf{F}_1+ \beta_2\mathbf{F}_2+ \sum_{i=3}^s \beta_i\mathbf{F}_i.
\end{align*}
Our assumption (also in case (ii)) is still that $\mathbf{G}^{(t)}_{v\bullet}=\mathbf{G}^{(t)}_{w\bullet}$. Using the same argument as before, based on linear independence of $\mathbf{F}_1,\ldots,\mathbf{F}_s$, we must have that 
$\alpha_i=\beta_i$ for all $i\in [1,s]$.

But now comes the trouble. Let us focus on $\alpha_2$ and $\beta_2$, corresponding to $\mathbf{F}_2$. For the  vertex $v$, 
$$
\alpha_2= \mathbf{L}_{vv}\bigl(\sum_{u\in N_G(v), \hat{\pmb{\ell}}{}^{(t-1)}_u=a}
\mathbf{R}_{uu}\bigr)$$
Similarly, for the vertex $w$,
$$
\beta_2= \mathbf{L}_{ww}\bigl(\sum_{u\in N_G(w), \hat{\pmb{\ell}}{}^{(t-1)}_u=a}
\mathbf{R}_{uu}\bigr)$$
Since the vertices $u$ in these summands all carry the same label, i.e., $\hat{\pmb{\ell}}{}^{(t-1)}_u=a$ this implies that all corresponding $\mathbf{R}_{uu}$ are equal. Denote this value by $\rho$. Furthermore, in case~(ii) $\hat{\pmb{\ell}}{}^{(t-1)}_v=\hat{\pmb{\ell}}{}^{(t-1)}_w$
holds, which implies that $\mathbf{L}_{vv}=\mathbf{L}_{ww}$. Denote this value by $\lambda$. As consequence,
\begin{align*}
\alpha_2&= \lambda\rho|\{u\in N_G(v), \hat{\pmb{\ell}}{}^{(t-1)}_u=a\}|\\
\beta_2&= \lambda\rho|\{u\in N_G(w), \hat{\pmb{\ell}}{}^{(t-1)}_u=a\}|.
\end{align*}
As a consequence, $\alpha_2=\beta_2$ implies that 
$$
|\{u\in N_G(v), \hat{\pmb{\ell}}{}^{(t-1)}_u=a\}|=|\{u\in N_G(w), \hat{\pmb{\ell}}{}^{(t-1)}_u=a\}|,
$$
which contradicts that the label ``$a$'' is assumed to occur a different number of times in $N_G(v)$ and $N_G(w)$, respectively. In summary, case (ii) cannot occur.

Hence, we have shown that there exists a $p$, $0<m_p<p<1$ such that 
$$\mathbf{G}^{(t)}:=(\mathbf{L}\mathbf{A}\mathbf{R}+p\mathbf{I})\mathbf{F}^{(t-1)}$$
is equivalent to $\hat{\pmb{\ell}}{}^{(t)}$, as desired.

\openprob{Can we set $m_p$ to a value in a non-procedural way?}
\end{proof}

\floris{If the above is correct, we actually have a stronger lower bound: Already for
architectures of the form $\sigma((\mathbf{L}\mathbf{A}\mathbf{R}+p\mathbf{I})\mathbf{F}^{(t)}\mathbf{v}\mathbf{w}^t+q\mathbf{J{}})$ with $\mathbf{v}$ and $\mathbf{w}$ two vectors we can show WL strongness. These vectors originate from Lemmas~\ref{lem:signlemma9} and
~\ref{lem:relulemma9}. }

\subsection{Special cases}
% We assume, for $t=0$, that $\mathbf{F}^{(0)}$ is good re
% \todo{The case $t=0$ may need to be revisited. Since only $\pmb{\ell}\equiv\mathbf{F}^{(0)}$ is required I believe.}
We next consider the special cases of GNNs discussed in Section~\ref{subsec:specialcases}.
For each of these cases we can derive matching lower bounds from the general lower bound from the previous section.

\begin{corollary}
GNNs of the form~(\ref{eq:architecture}) for which 
$\pmb{\ell}\sqsubseteq\hat{\pmb{\ell}}$ holds, for any labeled graph $(G,\pmb{\ell})$, are WL-strong.
\end{corollary}
\begin{proof}
It suffices to observe again that $\pmb{\ell}\sqsubseteq\hat{\pmb{\ell}}$  implies that
$\pmb{\ell}\equiv\hat{\pmb{\ell}}$. As a consequence, also $\pmb{\ell}^{(t)}\equiv\hat{\pmb{\ell}}{}^{(t)}$ for any $t\geq 0$. So, the general lower bound implies that there exist parameters $p$ and $q$ and weight matrices $\mathbf{W}^{(t)}$
such that $\mathbf{F}^{(t)}\equiv\hat{\pmb{\ell}}{}^{(t)}\equiv\pmb{\ell}^{(t)}$.
\end{proof}

As mentioned earlier, this lower bound holds in particular for GNNs of the form~(\ref{eq:groheGNNwithJ}). Hence we recover the WL-strongness result from~\cite{grohewl}
for both the sign and ReLU activation function.

\begin{corollary}
GNNs of the form~(\ref{eq:architecture}) for which 
$\pmb{\ell}^{(1)}\sqsubseteq\hat{\pmb{\ell}}$ holds, for any labeled graph $(G,\pmb{\ell})$, are WL-strong starting from $\pmb{\ell}^{(1)}$.
\end{corollary}
\begin{proof}
A closer inspection of the general lower bound proof shows that it suffices to re-establish
Lemmas~\ref{lem:rightgood} and~\ref{lemma:findp}. More specifically, suppose that 
$\mathbf{F}^{(0)}$ is good for $\pmb{\ell}^{(1)}$. We need to show that (base case)
$\mathbf{R}\mathbf{F}^{(0)}$ is also good for $\pmb{\ell}^{(1)}$ and 
$\mathbf{R}\mathbf{F}^{(t)}$ is good for $\pmb{\ell}^{(t+1)}$ for any $t>0$.
For the base case, suppose that 
$\pmb{\ell}^{(1)}_v=\pmb{\ell}^{(1)}_w$. By assumption, $\mathbf{F}^{(0)}_{v\bullet}=\mathbf{F}^{(0)}_{w\bullet}$ and 
$\hat{\pmb{\ell}}_v=\hat{\pmb{\ell}}_w$. The latter implies that $\mathbf{L}_{vv}=\mathbf{L}_{ww}$ and $\mathbf{R}_{vv}=\mathbf{R}_{ww}$. Hence,
also $\mathbf{R}_{vv}\mathbf{F}^{(0)}_{v\bullet}=\mathbf{R}_{ww}\mathbf{F}^{(0)}_{w\bullet}$.
Conversely, suppose that 
$\mathbf{R}_{vv}\mathbf{F}^{(0)}_{v\bullet}=\mathbf{R}_{ww}\mathbf{F}^{(0)}_{w\bullet}$.
Then, because all unique rows in $\mathbf{F}^{(0)}$ are independent, $\mathbf{F}^{(0)}_{v\bullet}=\mathbf{F}^{(0)}_{w\bullet}$ and thus also 
$\pmb{\ell}^{(1)}_v=\pmb{\ell}^{(1)}_w$, by assumption. In other words, 
$\mathbf{R}\mathbf{F}^{(0)}\equiv\pmb{\ell}^{(1)}$.
Clearly, $\mathbf{R}\mathbf{F}^{(0)}$ is row-independent modulo equality, because
$\mathbf{F}^{(0)}$ is row-independent modulo equality.
For $t>0$, we proceed in a similar way using that $\pmb{\ell}^{(t)}\sqsubseteq \pmb{\ell}^{(1)}$. 

It can be verified in the general lower bound proof that Lemma~\ref{lem:findingp}
works for any labeling. So we can apply it to $\pmb{\ell}^{t+1}$ rather than $\hat{\pmb{\ell}}{}^{)t}$. 

For Lemma~\ref{lemma:findp}, we have to find a $p$ such that 
$\mathbf{L}\mathbf(A+p\mathbf{I})\mathbf{R}^{\mathbf{F}^{(t-1)}}\mathbf{U}^{(t-1)}$
is equivalent to $\pmb{\ell}^{(t+1)}$, for $t>0$. We know from Lemma~\ref{lem:findingp} that
$\mathbf(A+p\mathbf{I})\mathbf{R}^{\mathbf{F}^{(t-1)}}\mathbf{U}^{(t-1)}$ is already equivalent to $\pmb{\ell}^{(t+1)}$. We then use for all $t>0$, $\pmb{\ell}^{(t+1)}\sqsubseteq\pmb{\ell}^{(1)}\sqsubseteq \hat{\pmb{\ell}}$. Hence,
whenever $\pmb{\ell}^{(t+1)}_v=\pmb{\ell}^{(t+1)}_w$ this implies $\mathbf{L}_{vv}=\mathbf{L}_{ww}$. This implies that 
$\mathbf{L}\mathbf(A+p\mathbf{I})\mathbf{R}^{\mathbf{F}^{(t-1)}}\mathbf{U}^{(t-1)}\sqsubseteq \pmb{\ell}^{(t+1)}$. The rest of the proof of Lemma~\ref{lem:findingp}  carries over by replacing $\hat{\pmb{\ell}}{}^{(t)}$ by $\pmb{\ell}^{(t+1)}$.

Finally, Lemmas~\ref{lem:signlemma9} and~\ref{lem:relulemma9} and the remainder of the proof of the general lower bound carry over almost verbatim.
\end{proof}

We observe that the above proof does not work in general when $\mathbf{F}^{(0)}\equiv\pmb{\ell}$. The problem is with the base case of Lemma~\ref{lem:rightgood}. More specifically, the condition that $\pmb{\ell}^{(1)}\sqsubseteq \hat{\pmb{\ell}}$ holds for every labeled graph $(G,\pmb{\ell})$ is not sufficient to derive that $\mathbf{R}\mathbf{F}^{(0)}\equiv\pmb{\ell}$. This is precisely why the last special case in  Section~\ref{subsec:specialcases} additionally requires $\pmb{\ell}_v=\pmb{\ell}_w\Rightarrow \mathbf{R}_{vv}=\mathbf{R}_{ww}$ to hold.
Under this assumption, if $\mathbf{F}^{(0)}_{v\bullet}=\mathbf{F}^{(0)}_{w\bullet}$
then $\mathbf{R}_{vv}=\mathbf{R}_{ww}$ and hence also 
$\mathbf{R}_{vv}\mathbf{F}^{(0)}_{v\bullet}=\mathbf{R}_{ww}\mathbf{F}^{(0)}_{w\bullet}$.
This suffices to derive Lemma~\ref{lem:rightgood}. The rest of the general lower bound proof
carries over by replacing $\hat{\pmb{\ell}}{}^{(t)}$ by $\pmb{\ell}^{(t)}$. As a consequence:
\begin{corollary}
GNNs of the form~(\ref{eq:architecture}) for which 
$\pmb{\ell}^{(1)}\sqsubseteq\hat{\pmb{\ell}}$ and $\pmb{\ell}_v=\pmb{\ell}_w\Rightarrow \mathbf{R}_{vv}=\mathbf{R}_{ww}$ hold, for any labeled graph $(G,\pmb{\ell})$, are WL-strong starting from $\pmb{\ell}$.
\end{corollary}

Furthermore, 
\begin{corollary}
GNNs of the form~\ref{eq:architecture_lb_GCN} for which 
$\pmb{\ell}^{(1)}\sqsubseteq\hat{\pmb{\ell}}$  hold, for any labeled graph $(G,\pmb{\ell})$, are WL-strong starting from $\pmb{\ell}^{(1)}$.
\end{corollary}
\begin{proof}
\openprob{Is this true?}
\end{proof}


We summarize our results in Table~\ref{tbl:strongGNN}
\begin{table}
	\centering
\begin{tabular}{|c|c|}\hline
	GNN architecture &  WL-strong\\\hline\hline
$\mathbf{F}^{(t)}:=\sigma((\mathbf{L}(\mathbf{A}+p\mathbf{I})\mathbf{R}\mathbf{F}^{(t-1)}\mathbf{W}^{(t-1)}+q\mathbf{J})$ & \multirow{2}{*}{$(G,\hat{\pmb{\ell}})$} \\
$\mathbf{F}^{(t)}:=\sigma\left((\mathbf{L}\mathbf{A}\mathbf{R}+p\mathbf{I})\mathbf{F}^{(t-1)}\mathbf{W}^{(t-1)} + q\mathbf{J}\right)$ & 
\\\hline
$\mathbf{F}^{(t)}:=\sigma((\mathbf{A}+p\mathbf{I})\mathbf{F}^{(t-1)}\mathbf{W}+q\mathbf{J})$ & \multirow{3}{*}{$(G,\pmb{\ell})$}\\
$\mathbf{F}^{(t)}:=\sigma(\mathbf{D}^{-1}(\mathbf{A}+p\mathbf{I})\mathbf{F}^{(t-1)}\mathbf{W}+q\mathbf{J})$ & \\
$\mathbf{F}^{(t)}:=\sigma(\tilde{\mathbf{D}}{}^{-1}(\mathbf{A}+p\mathbf{I})\mathbf{F}^{(t-1)}\mathbf{W}+q\mathbf{J})$ & \\\hline
$\mathbf{F}^{(t)}:=\sigma(\mathbf{D}^{-1/2}(\mathbf{A}+p\mathbf{I})\mathbf{D}^{-1/2}\mathbf{F}^{(t-1)}\mathbf{W}+q\mathbf{J})$ & \multirow{3}{*}{$(G,\pmb{\ell}^{(1)})$}\\
$\mathbf{F}^{(t)}:=\sigma(\tilde{\mathbf{D}}{}^{-1/2}(\mathbf{A}+p\mathbf{I})\tilde{\mathbf{D}}{}^{-1/2}\mathbf{F}^{(t-1)}\mathbf{W}+q\mathbf{J})$ & \\
$\mathbf{F}^{(t)}:=\sigma((\mathbf{D}^{-1/2}\mathbf{A}\mathbf{D}^{-1/2}+p\mathbf{I})\mathbf{F}^{(t-1)}\mathbf{W}^{(t-1)}+q\mathbf{J})$ & \\\hline
 % $\sigma((\mathbf{L}\mathbf{A}\mathbf{R}+p\mathbf{I})\mathbf{F}^{(t-1)}\mathbf{W}^{(t-1)}+q\mathbf{J})$ & ??\\\hline
% $\mathbf{F}^{(t)}:=\sigma((r\mathbf{I}+(1-r)\mathbf{D})^{-1/2}(\mathbf{A}+p\mathbf{I})r\mathbf{I}+(1-r)\mathbf{D})^{-1/2}\mathbf{F}^{(t-1)}\mathbf{W}+q\mathbf{J})$ & \\\hline
\end{tabular}
\caption{Strong GNN architectures. Restricted class of GNNs in which $p=0$ or $p=1$ are not strong. }\label{tbl:strongGNN}	
\end{table}


\subsection{Negative results}

\openprob{The aim of this section is to show that having $p$ different from $0$ or $1$ is necessary to obtain lower bounds. It should serve as a justification to consider, perhaps an the experimental section, how each of the existing models compares to their relaxation by introducing $p\mathbf{I}$ with $p$ a learnable parameter. {\bf We need to see what kind of examples we need. Some can be found below.}}

It is readily verified that Lemma~\ref{lem:findingp} does not hold in general when $p=1$, as
is illustrated by the following example.
\begin{example}
We consider the augmented adjacency architecture on input graph $(G,\pmb{\ell})$
(\raisebox{-0.5ex}{\includegraphics[height=0.4cm]{graph3.pdf}}) with 
vertex labelling with $\pmb{\ell}_{v_1}=a$ and $\pmb{\ell}_{v_2}=b$. We know from 
Corollary~\ref{cor:augmented} that this architecture is upper bounded by 1-WL, starting
from $\pmb{\ell}^{(1)}$. More precisely, we have that $\pmb{\ell}^{(t)}\sqsubseteq\mathbf{F}^{(t-1)}$ for $t\geq 1$.
Consider $\mathbf{F}^{(0)}:=\left(\begin{smallmatrix}1 & 0\\
0 & 1\end{smallmatrix}\right)$ such that $\mathbf{F}^{(0)}$ is good for $\pmb{\ell}$. We also note that $\pmb{\ell}^{(1)}_{v_1}=(a,\{b\})$ and $\pmb{\ell}^{(1)}_{v_2}=(b,\{a\})$. Hence, 
$\mathbf{F}^{(0)}$ is also good for $\pmb{\ell}^{(1)}$. We next show that there does not exists a
weight matrix $\mathbf{W}^{(0)}$ such that $\pmb{\ell}^{(2)}\equiv \mathbf{F}^{(1)}$. Indeed,
$$
\mathbf{F}^{(1)}:=\sigma\Biggl(\frac{1}{2}\biggl(\begin{pmatrix}0 & 1 \\
1 & 0
\end{pmatrix}+
\begin{pmatrix}
	 1 & 0 \\
0 & 1 \end{pmatrix}\biggl)\mathbf{F}^{(0)}\mathbf{W}^{(0)}\Biggr)
=\sigma(\Biggl(\frac{1}{2}\begin{pmatrix}1 & 1 \\
1 & 1
\end{pmatrix}
\mathbf{F}^{(0)}\mathbf{W}^{(0)}\Biggr)=
\sigma\Biggl(\frac{1}{2}\begin{pmatrix}
1& 1\\
1 & 1\\
\end{pmatrix}\mathbf{W}^{(0)}\Biggr).
$$
Hence, independent of the choice of $\mathbf{W}^{(0)}$, both vertices will be assigned the same
feature vector. By contrast, $\pmb{\ell}^{(2)}\sqsubseteq\pmb{\ell}^{(1)}$ and hence
$\pmb{\ell}^{(2)}$ must assign distinct labels to these vertices, since $\pmb{\ell}^{(1)}$ does so. Hence, $\mathbf{F}^{(2)}\not\equiv\pmb{\ell}^{(2)}$.\qed
\end{example}
The example shows that the upper bound for the augmented adjacency architecture as given in Corollary~\ref{cor:augmented} cannot be matched with a lower bound.
In fact the same example can be used to show that the augmented random walk architecture is not as strong as 1-WL, starting from $\pmb{\ell}^{(0)}$. That is the upper bound from Corollary~\ref{cor:augrw1wl} cannot be matched with a lower bound.

In the proof of  Lemma~\ref{lem:findingp} it is also important that $p\neq 0$. THe following simple example shows that the adjacency, random walk and and normalised adjacency architectures are not as strong as 1-WL (recall that for these architectures $p=0$).

\begin{example}\label{example:piszero}
consider the  adjacency architecture on input graph  $(G,\pmb{\ell})$
(\raisebox{-0.4ex}{\includegraphics[height=0.5cm]{graph4.pdf}}) with 
vertex labelling with $\pmb{\ell}_{v_1}=a$, $\pmb{\ell}_{v_2}=b$ and $\pmb{\ell}_{v_3}=c$.
Consider $\mathbf{F}^{(0)}:=\left(\begin{smallmatrix}1 & 0 & 0\\
0 & 1 & 0\\
0 & 0 & 1\end{smallmatrix}\right)$ such that $\mathbf{F}^{(0)}$ is good for $\pmb{\ell}$.
Then,
$$
\mathbf{F}^{(1)}:=\sigma\Biggl(\begin{pmatrix}0 & 1 & 0 \\
1 & 0 & 1\\
0 & 1 & 0
\end{pmatrix}\mathbf{F}^{(0)}\mathbf{W}^{(0)}\Biggr)
=\sigma(\mathbf{W}^{(0)}\Biggr)=
\sigma\Biggl(\begin{pmatrix}0 & 1 & 0 \\
1 & 0 & 1\\
0 & 1 & 0
\end{pmatrix}\mathbf{W}^{(0)}\Biggr).
$$
Hence, independent of the choice of $\mathbf{W}^{(0)}$,  vertices $v_1$ and $v_3$ will be assigned the same
feature vector. By contrast, $\pmb{\ell}^{(1)}_{v_1}=(a,\{b\})\neq \pmb{\ell}^{(1)}_{v_2}=(c,\{b\})$.
This example also works for the random walk and and normalised adjacency architectures. \qed
\end{example}

We thus see that having $0<p<1$ is a necessary condition for GNN architectures~(\ref{eq:architecture})
to be as strong as 1-WL.

\subsection{Separating examples}

\openprob{Can we argue, based on results so far how to separate the different GNN models.}
\begin{itemize}
	\item I believe we can use the strongness results do so between the architectures in each group in Table~\ref{tbl:strongGNN}.
	\item It would be great if we can separate all GNNs, even within a group in  Table~\ref{tbl:strongGNN}. Perhaps using $\hat{\pmb{\ell}}$. Also, can one GNN architecture simulate another one? That is, can one find learnable parameters of one model to simulate another one (this is an extension of WL strongness where WL is replaced by the labeling computer by another GNN architecture.) This makes only sense when both GNN architecture are known to be bounded in the same way.
\end{itemize}


\subsection{Bias}
\floris{Perhaps too ambitious...}
\openprob{Can we find examples that show that the bias is needed, i.e., $q\neq 0$ is necessary in general? Note we only need to look at GNN architectures for which $0<p<1$.}
Consider  GNN architectures 
$$\mathbf{F}^{(t+1)}:=\sigma(\mathbf{L}(\mathbf{A}+p\mathbf{I})\mathbf{R}\mathbf{F}^{(t)}\mathbf{W}^{t}),	
$$
with $0<p<1$. We next show that bias is needed.

\begin{example}
	Consider the relaxed augmented random walk architecture
$$\mathbf{F}^{(t+1)}:=\text{sgn}(\tilde{\mathbf{D}}^{-1/2}(\mathbf{A}+p\mathbf{I})\mathbf{F}^{(t)}\mathbf{W}^{t}),
$$
with $0<p<1$. We know that it is upper bounded by 1-WL, starting from $\pmb{\ell}$. We next show that without bias, we cannot get a matching lower bound. Note that this example uses sgn as activation function.

Consider the graph  $(G,\pmb{\ell})$
(\raisebox{-0.4ex}{\includegraphics[height=0.5cm]{graph5.pdf}}) with $\pmb{\ell}_{v_1}=\pmb{\ell}_{v_2}=\pmb{\ell}_{v_3}=a$. Let $\mathbf{F}^{(0)}:=\left(\begin{smallmatrix} 1\\1\\1\end{smallmatrix}\right)$.
We have 
$$
\mathbf{F}^{(1)}:=\text{sgn}\Biggl(
\begin{pmatrix}
1/\sqrt{2} & 0 & 0\\
0 & 1/\sqrt{3} & 0\\
0 & 0 & 1/\sqrt{2}
\end{pmatrix}
\begin{pmatrix}
p & 1 & 0\\
1 & p & 1\\
0 & 1 & p
\end{pmatrix}
\begin{pmatrix}
	1\\
	1\\
	1
	\end{pmatrix}\mathbf{W}^{(0)}\Biggr)
=\text{sgn}\Biggl(
\begin{pmatrix}
\frac{1}{\sqrt{2}}(1+p)\\
\frac{1}{\sqrt{3}}(2+p)\\
\frac{1}{\sqrt{2}}(1+p)\end{pmatrix}
\mathbf{W}^{(0)}
\Biggr)
$$
We remark that $\mathbf{W}^{(0)}$ is a $1\times f$-matrix. Clearly, all entries in
$\mathbf{F}^{(1)}$ will have the same sign. By contrast, $\pmb{\ell}^{(1)}_{v_1}=(a,\{a\})\neq
(a,\{a,a\})=\pmb{\ell}^{(1)}_{v_2}$. So, $\pmb{\ell}^{(1)}\not\equiv\mathbf{F}^{(1)}$ for any choice of $\mathbf{W}^{(0)}$.\qed
\end{example}

% \begin{corollary}
% Suppose that $\pmb{\ell}\not\equiv\hat{\pmb{\ell}}$ but $\pmb{\ell}^{(1)}\sqsubseteq \hat{\pmb{\ell}}$. Then, the architecture is bounded by 1-WL, starting from $\pmb{\ell}^{(1)}$.
% \end{corollary}
% \begin{proof}
% Let $t=0$. We have that $\pmb{\ell}^{(1)}\sqsubseteq \pmb{\ell}\sqsubset \mathbf{F}^{(0)}$. For $t=1$, assume that $\pmb{\ell}^{(2)}_v=\pmb{\ell}^{(2)}_w$. We show that
% $\mathbf{F}_{v\bullet}^{(1)}=\mathbf{F}_{w\bullet}^{(1)}$.
% Indeed, we have that $\pmb{\ell}^{(1)}_v=\pmb{\ell}^{(1)}_w$ and
%  for every $u\in N_G(v)$ and corresponding $u'\in N_G(w)$,
%  $\pmb{\ell}^{(1)}_u=\pmb{\ell}^{(1)}_{u'}$. Then,
%  $\pmb{\ell}^{(1)}\sqsubseteq \hat{\pmb{\ell}}$ implies that
%   $\mathbf{L}_{vv}=\mathbf{L}_{ww}$, $\mathbf{R}_{vv}=\mathbf{R}_{ww}$ and
% $\mathbf{B}_{v\bullet}=\mathbf{B}_{w\bullet}$.
% Furthermore, for every $u\in N_G(v)$ and corresponding $u'\in N_G(w)$, $\mathbf{R}_{uu}=\mathbf{R}_{u'u'}$. By induction, also $\pmb{\ell}^{(1)}\subseteq \mathbf{F}^{(0)}$. This suffices to conclude that $\mathbf{F}^{(1)}_{v\bullet}=\mathbf{F}^{(1)}_{w\bullet}$. For $t>2$, assume that $\pmb{\ell}^{(t)}_v=\pmb{\ell}^{(t)}_w$. Then,
%  $\pmb{\ell}^{(t-1)}_v=\pmb{\ell}^{(t-1)}_w$ and for every $u\in N_G(v)$ and corresponding $u'\in N_G(w)$,  $\pmb{\ell}^{(t-1)}_u=\pmb{\ell}^{(t-1)}_{u'}$. In particular,
%  since $\pmb{\ell}^{(t-1)}\sqsubseteq \pmb{\ell}^{(1)}$,
%  $\pmb{\ell}^{(1)}_v=\pmb{\ell}^{(1)}_w$ and for every $u\in N_G(v)$ and corresponding $u'\in N_G(w)$,  $\pmb{\ell}^{(1)}_u=\pmb{\ell}^{(1)}_{u'}$. Using that $\pmb{\ell}^{(1)}\sqsubseteq \hat{\pmb{\ell}}$ holds, this implies that
%  $\mathbf{L}_{vv}=\mathbf{L}_{ww}$, $\mathbf{R}_{vv}=\mathbf{R}_{ww}$ and
% $\mathbf{B}_{v\bullet}=\mathbf{B}_{w\bullet}$. Furthermore, for every $u\in N_G(v)$ and corresponding $u'\in N_G(w)$, $\mathbf{R}_{uu}=\mathbf{R}_{u'u'}$. By induction, also $\pmb{\ell}^{(t-1)}\subseteq \mathbf{F}^{(t-2)}$. This suffices to conclude that $\mathbf{F}^{(t-1)}_{v\bullet}=\mathbf{F}^{(t-1)}_{w\bullet}$.
% \end{proof}
% This corollary applies, for example, for the normalized adjacency and augmented adjacency architecture.

%
% \todo{From an upper bound perspective, $\mathbf{L}$ has no apparent impact when it is degree related. By contrast, $\mathbf{R}$ speeds up
% 1-WL with one step. The role of $p$ and $q$ and $\mathbf{B}$ also do seem to have an impact for upper bounds. Can we say something more here?}
%!TEX root = quiver.tex

%!TEX root =main.tex
\section{GNNs without self}
\floris{The following needs to be developed in more detail.}
\floris{The idea in this section is to reconsider GNNs without $\mathbf{I}$. The lower bound proof shows that having $\mathbf{I}$ is required to encode the labels of the vertices themselves. So, when absent, we should be able to upper and lower bound using a variation of WL in which only neighbor labels are accounted for. In some sense, this is a generalization of the unlabeled case by~\cite{grohewl} but lifted to labeled graphs. It will allow us to seperate some  more architectures.}

We have seen in Example~\ref{example:piszero} that GNN architectures in which $p=0$ are not necessarily WL-strong, starting from $\hat{\pmb{\ell}}$. Intuitively, the reason is that when $p=0$, the only features that are propagated relate to neighborhoods of vertices and not the features of the vertices themselves. In this section we investigate the expressive power of architectures of the form:
\begin{equation}
\mathbf{F}^{(t)}:=\sigma\left(\mathbf{L}\mathbf{A}\mathbf{R}\mathbf{F}^{(t-1)}\mathbf{W}^{(t-1)} + q\mathbf{J}\right). \label{eq:architecture_noid}
\end{equation}
In a nutshell, we will upper and lower bound such architectures by 
 considering a \textit{weaker variant of WL}  which only takes neighborhood information into account.
 %  Indeed,
% in GNN architectures of the form~\ref{eq:architecture_noid}, the absence of the identity has a consequence that features are updated based on adjacency information alone. We make this now more precise.

We define the \textit{neighbor-only WL}, NWL for short, process as follows. Let $G=(V,\pmb{\mu})$ be a labeled graph and let $\Sigma$ be a set of labels. Initially, let $\pmb{\mu}^{(0)}:=\pmb{\mu}$. 
Then, the NWL procedure computes a labeling $\pmb{\mu}^{(t)}$, for $t> 0$, as follows: 
$$
\pmb{\mu}^{(t)}_v:=\textsc{Hash}\bigl(\ldbl \pmb{\mu}_u^{(t-1)} \st u \in N_G(v) \rdbl\bigr),
$$
where $\textsc{Hash}$ bijectively maps the multi-set $\ldbl \pmb{\mu}_u^{(t-1)} \st u \in N_G(v) \rdbl$ of labels of $v$'s neighbors to a unique label in $\Sigma$, which has not been used in previous iterations. When the number of distinct labels in $\pmb{\mu}^{(t)}$ and $\pmb{\mu}^{(t-1)}$ is the same, the NWL algorithm terminates.
Termination is guaranteed in at most $n$ steps. We refer to the resulting labeling as the \textit{NWL labeling of $(G,\pmb{\mu})$}. 

As before, given matrices $\mathbf{L}$ and $\mathbf{R}$ we define the set of 
extended labels as $\hat{\Sigma}=\Sigma\cup (\Sigma\times L\times R)$ with
$L:=\{\mathbf{L}_{vv}\mid v\in V\}$ and $R:=\{\mathbf{R}_{vv}\mid v\in V\}$.
The following counterpart of Theorem~\ref{thm:generalbound} is now easily verified.

\begin{theorem}\label{thm:generalbound_noid}
GNN architectures of the form~(\ref{eq:architecture_noid}) for which 
$\pmb{\mu}^{(1)}\sqsubseteq \hat{\pmb{\mu}}$ holds for any labeled graph $(G,\pmb{\mu})$, are bounded by NWL on $(G,\hat{\pmb{\ell}})$.
\end{theorem}
Compared to Theorem~\ref{thm:generalbound} the extra condition involving $\pmb{\mu}^{(1)}\sqsubseteq \hat{\pmb{\mu}}$ is needed to ensure that the values in $\mathbf{L}$ are functionally determined by degree information of the vertices.

\begin{proof}
We show the upper bound by NWL by induction on the number of iterations. For $t=0$, we have, by assumption, that 
$\pmb{\ell}\sqsubseteq \mathbf{F}^{(0)}$. Clearly,
$\hat{\pmb{\mu}}{}^{(0)}\sqsubseteq \pmb{\ell}$ and hence also 
$\hat{\pmb{\mu}}{}^{(0)}\sqsubseteq\mathbf{F}^{(0)}$. We next assume that the induction hypothesis holds for $t\geq 0$ and consider $t+1$. We need to show that 
$\hat{\pmb{\mu}}{}^{(t+1)}_v=\hat{\pmb{\mu}}{}^{(t+1)}_w$ implies that $\mathbf{F}^{(t+1)}_{v\bullet}=\mathbf{F}^{(t+1)}_{w\bullet}$. By definition,
$\hat{\pmb{\mu}}{}^{(t+1)}_v=\hat{\pmb{\mu}}{}^{(t+1)}_w$ implies
$$
\ldbl \hat{\pmb{\mu}}{}^{(t)}_u \st u \in N_G(v) \rdbl=
 \ldbl \hat{\pmb{\mu}}{}^{(t)}_u \st u \in N_G(w) \rdbl.$$
 Since $\hat{\pmb{\mu}}{}^{(t)}\sqsubseteq \hat{\pmb{\mu}}{}^{(t-1)}\sqsubseteq \cdots\sqsubseteq \hat{\pmb{\mu}}{}^{(0)}$, this implies that  there is a bijection $b:N_G(v)\to N_G(w):u\mapsto u'$ such that $\hat{\pmb{\mu}}{}^{(t)}_u=\hat{\pmb{\mu}}{}^{(t)}_{u'}$ and hence also 
 $\hat{\pmb{\mu}}{}^{(0)}_u=\hat{\pmb{\mu}}{}^{(0)}_{u'}$.
From the definition of $\hat{\pmb{\mu}}{}^{(0)}$, this implies that for every $u\in N_G(v)$ and corresponding $u'\in N_G(w)$, $\mathbf{L}_{uu}=\mathbf{L}_{u'u'}$ and $\mathbf{R}_{uu}=\mathbf{R}_{u'u'}$. By the assumption that on $\mathbf{L}$ we further have that
$\mathbf{L}_{vv}=\mathbf{L}_{ww}$.
By the induction hypothesis we also for every $u\in N_G(v)$
   and corresponding $u'\in N_G(w)$, $\mathbf{F}^{(t)}_{u\bullet}=\mathbf{F}^{(t)}_{u'\bullet}$. It now suffices to observe that
  \begin{align*}
	  \mathbf{F}^{(t+1)}_{v\bullet}&=\sigma\Biggl(\mathbf{L}_{vv}\Bigl(\sum_{u\in N_G(v)} \mathbf{R}_{uu}\mathbf{F}^{(t)}_{u\bullet}\Bigr)\mathbf{W}^{(t)}+ q\mathbf{J}_{v\bullet}\Biggr)\\
	 & =\sigma\Biggl(\mathbf{L}_{ww}\Bigl(\!\!\sum_{u'\in N_G(w)}\!\! \mathbf{R}_{u'u'}\mathbf{F}^{(t)}_{u'\bullet}\Bigr)\mathbf{W}^{(t)}+ q\mathbf{J}_{w\bullet}\Biggr)\\
	  &=\mathbf{F}^{(t+1)}_{w\bullet},
\end{align*}
as desired.
\end{proof}

We can also show that GNN architectures of the form~(\ref{eq:architecture_noid}) 
are NWL-strong, starting from $\hat{\pmb{\mu}}$.
\begin{proposition}
The class of GNN architectures of the form~(\ref{eq:architecture_noid}) for which 
	$\pmb{\mu}^{(1)}\sqsubseteq \hat{\pmb{\mu}}$ holds for any labeled graph $(G,\pmb{\mu})$, are NWL-strong, starting from $\hat{\pmb{\mu}}$.
\end{proposition}
\begin{proof}
We closely follow the proof of Theorem~\ref{thm:lowerb_general}.
 By assumption, $\mathbf{F}^{(0)}\equiv \hat{\pmb{\mu}}$ and $\mathbf{F}^{(0)}$
 row independent modulo equality. Consider $t>0$ and assume that $\mathbf{F}^{(t-1)}$ is good for $\hat{\pmb{\mu}}$. It is easily verified that the proof of Lemma~\ref{lem:rightgood} also works here. Hence, $\mathbf{R}\mathbf{F}^{(t-1)}$ is also good for  $\hat{\pmb{\mu}}$. We can also use first part of the proof of Lemma~\ref{lem:findingp}. Using the notation in that proof, we know that there exists a matrix $M^{(t-1)}$ such that for every $v\in V$ and $c\in\Sigma^{(t-1)}$:
 $$
 (\mathbf{A}\mathbf{R}\mathbf{F}^{(t-1)}\mathbf{M}^{(t-1)})_{vc}=|u\in N_G(v)| \mathbf{R}\mathbf{F}_{u\bullet}^{(t-1)}\sim c\}|
 $$

\floris{To be worked out further.... }
\end{proof}

Clearly, the two architectures without $\mathbf{I}$, RW-GNN and NA-GNN satisfy the conditions in the Theorem. The same holds for the following architecture:
\begin{description}
 \item[\textit{Adjacency} (A-GNN):]
% $\mathbf{L}=\mathbf{R}:=\mathbf{I}$, $p=q:=0$. Hence,
$
\mathbf{F}^{(t)}:=\sigma\left(\mathbf{A}\mathbf{F}^{(t-1)}\mathbf{W}^{(t)}\right)
$
\end{description}
which was shown to be bounded by WL and WL-strong on unlabeled graphs in ~\cite{grohewl}.
\floris{We need to make the connection with the unlabelled case more precise.}

\subsection{Special cases}
\openprob{All what follows needs to be shown (if we want ;-)}
\begin{corollary}
The A-GNN,  NA-GNN  and RW-GNN architectures are bounded by WWL, starting from $(G,\pmb{\mu})$.	
\end{corollary}
Since $\hat{\pmb{\ell}}{}^{(k)}\sqsubseteq \hat{\pmb{\mu}}{}^{(k)}$, this corollary provides a stronger upper bound than Corollary. Indeed, it says that these architecture cannot classify vertices in a finer way than the weak version of WL. 

We can again zoom in on these three architectures and show that these architecture can be bounded by
$\pmb{\mu}^{(t)}$ rather than $\hat{\pmb{\mu}}^{(t)}$. 
\begin{corollary}
The A-GNN  and RW-GNN architectures are bounded by WWL, starting from $(G,\pmb{\ell})$.
The NA-GNN architecture is bounded by WWL, starting from $(G,\pmb{\mu}{}^{(1)})$.
\end{corollary}

\subsection{Lower bounds}

\openprob{We can tell, I believe, the same story as before but now for WWL. It should allow us to separate classes with $\mathbf{I}$.}



\bibliographystyle{apalike}
\bibliography{refs}

% \section{Introduction}
\begin{itemize}
  \item After going through Kipf's code, I did not find any sign of bias being
    used in the activation-function calls. In fact, it seems to me that they
    explicitly set it to 0 (see 
    https://github.com/tkipf/gcn).
    \item Interestingly, in~\cite{xhlj19} they claim to have shown Kipf's
      architecture is strictly less powerful than the 1-WL algorithm!
      (However, in his website, Kipf observes that they assume ``mean
      pooling''.)
   \item In~\cite{Wu2019}, it is said "We hypothesize that the nonlinearity
     between GCN layers is not critical - but that the majority of the
     benefit arises from the local averaging." This worth making more
     precise or even invalidate it theoretically.
   \item In~\cite{hyl17} they have a paragraph on Relation to the Weisfeiler-Lehman
     Isomorphism Test which makes more sense than Kipf. They clarify that
     isomorphism is not the end goal!
\end{itemize}

\section{Preliminaries}
Let $M = \ldbl m_0, m_1, \dots \rdbl$ denote a \emph{multiset} and $M(x)$
stand for the \emph{multiplicity} of the element $x$ within $M$.

\paragraph{Graphs.}
An \emph{(undirected) graph} $G$ is a pair $(V,E)$ where $V$ is a finite set
of \emph{vertices} and $E \subseteq \{\{u,v\} \st (u,v) \in V \times V, u \neq
v\}$ is a finite set of \emph{edges}. Let $V(G)$ and $E(G)$ denote the set of
vertices and edges of $G$ respectively. Let $N_G(u)$ denote the
\emph{neighbourhood} of vertex $u$ in $G$. That is to say, $N_G(u) := \{v \in V(G)
\st \{u,v\} \in E(G)\}$.

We will mostly work with the \emph{adjacency matrix} $\mathbf{A}$ of the
graph $G$. That is, $\mathbf{A}$ is the square $|V(G)| \times |V(G)|$ matrix
such that the entry $\mathbf{A}_{ij}$ is $1$ if $\{i,j\} \in E(G)$ and $0$
otherwise.

A \emph{vertex labelling} is a function $\ell: V(G) \to \Lambda$ with an
arbitrary co-domain $\Lambda$ of \emph{labels} and a \emph{labelled
graph} is a pair $(G,\ell)$ where $G$ is a graph and $\ell : V(G) \to \Lambda$
is a vertex-labelling function. A \emph{label class} $Q \subseteq V(G)$, with
respect to a vertex labelling $\ell$, is a maximal subset of $V(G)$ such that
$\ell(u) = \ell(v)$ for all $u,v \in Q$.

Let $\ell,\ell' : V(G) \to \Lambda$ be vertex labellings.
We say that $\ell$ \emph{refines} $\ell'$, written $\ell \sqsubseteq \ell'$,
if and only if for all $u,v \in V(G)$ we have
\[
    \ell(u) = \ell(v) \implies \ell'(u) = \ell'(v).
\]
Furthermore, we say that $\ell$ and $\ell'$ are \emph{equivalent}, written
$\ell \equiv \ell'$, if and only if $\ell \sqsubseteq \ell'$
and $\ell' \sqsubseteq \ell$. 

\todo{F. We should clarity what an unlabeled graph is.}
\paragraph{Isomorphism.}
We say two labelled graphs $(G,\ell_G)$ and $(H,\ell_H)$ are \emph{isomorphic}
if there exists an edge-preserving label-consistent bijection $\beta : V(G)
\to V(H)$. That is, $\beta$ is such that 
\begin{itemize}
    \item $\{u,v\} \in E(G)$ if and only if $\{\beta(u),\beta(v)\} \in E(H)$ and
    \item $\ell_G(u) = \ell_G(v)$
      if and only if $\ell_H(\beta(u)) = \ell_H(\beta(v))$.
\end{itemize}



\subsection{The Weisfeiler-Leman algorithm}
Let $(G,\ell)$ be a labelled graph. The $1$-WL algorithm computes, in each
iteration $t \geq 0$, a vertex labelling $\ell^{(t)} : V(G) \to \Lambda$ which
depends on the labelling from the previous iteration. Initially, we set
$\ell^{(0)} := \ell$. For $t > 0$ we set
\[
   \ell^{(t)}(u) := 
    \hash\left(\ell^{(t-1)}(u), \ldbl \ell^{(t-1)}(v) \st v \in N_G(u) \rdbl\right)
\]
where $\hash$ bijectively maps pairs of labels and label-multisets to a label
from $\Lambda$.  The algorithm terminates when a \emph{stable labelling} is
reached, i.e. when $\ell^{(t+1)} \equiv \ell^{(t)}$. Observe that
$\ell^{(t+1)} \sqsubseteq \ell^{(t)}$ for all $t \geq 0$ and that termination
is therefore guaranteed after at most $|V(G)|$ iterations.

\paragraph{Isomorphism heuristic.} 
If the stable labellings of two labelled graphs $(G,\ell_G)$ and $(H,\ell_H)$
have a different number of vertices labelled $\lambda$, for some label
$\lambda \in \Lambda$, the algorithm correctly concludes that they are not
isomorphic.

\subsection{Graph neural networks}
Consider a labelled graph $(G,\ell)$ with $\ell : V(G) \to \mathbb{R}^{1
\times a_0}$.  This, intuitively, means that every vertex $v$ is annotated
with a \emph{feature vector} $\ell(v) \in \mathbb{R}^{1 \times a_0}$.

A graph neural network (GNN, for short) is composed of layers which aggregate
the labels of the neighbours of a vertex, as computed by the previous layer,
and feed this aggregated information to the next layer.  A basic GNN model can
be implemented by setting $f^{(0)} := \ell$ and having each layer $t \geq 0$
compute a new feature vector $f^{(t+1)}(v) \in \mathbb{R}^{1 \times a_{t+1}}$ as
follows
\[
  \sigma\left(
    f^{(t)}(v) \mathbf{W}_1^{(t)} +
    \sum_{w \in N_G(v)} f^{(t)}(w) \mathbf{W}_2^{(t)}
  \right)
\]
where $\mathbf{W}_1^{(t)}, \mathbf{W}_2^{(t)} \in \mathbb{R}^{a_t \times
a_{t+1}}$ are parameter matrices and $\sigma$ denotes a non-linear activation
function such as the rectified linear unit (ReLU for short).\footnote{For ease
of comparison with the work of Kipf and Welling~\shortcite{kipf-loose} we do
not use a \emph{bias}.}
Note that the feature-vector update can be re-written in matrix
form as
\begin{equation}\label{eqn:gnn}
  \mathbf{F}^{(t+1)} = \sigma\left(
    \mathbf{F}^{(t)}\mathbf{W}_1^{(t)} +
    \mathbf{AF}^{(t)}\mathbf{W}_2^{(t)}
  \right)
\end{equation}
where $\mathbf{F}^{(t)}$ is the $|V(G)| \times a_t$ matrix such that $i$th row 
$\mathbf{F}^{(t)}_{i\bullet}$ of  $\mathbf{F}^{(t)}$ is the row vector $f^{(t)}(i)$ and $\mathbf{A}$ is the
adjacency matrix of $G$.

\subsubsection{GNNs are Weisfeiler-Leman powerful}
It has been shown that GNNs are as powerful as the $1$-WL
algorithm~\cite{grohewl}, a formal statement follows. For a labelled graph
$(G, \ell)$, let us write $\ell \equiv \lambda$ if $\ell$ maps every vertex of
$G$ to the same label $\lambda$.
\begin{proposition}[Corollary 12 from~\cite{grohewl}]\label{pro:grohe}
  Let $(G,\ell)$ be a labelled graph such that $\ell \equiv \lambda$. Then
  there exists a sequence of matrices such that for all $t \in \mathbb{N}$ and
  for $f^{(t)}$, as defined by Equation~\eqref{eqn:gnn} with $f^{(0)} = \ell$
  and $\sigma$ being ReLU, we have $f^{(t)} \equiv \ell^{(t)}$.
\end{proposition}
\todo{F. What is the $\lambda$ here?}

\subsection{GNNs and Kipf and Welling}
In~\cite{kipf-loose}, the following notion of GNNs was proposed.
Let $\mathbf{D}$ be the \emph{degree matrix} of $G$, that is $\mathbf{D}$
is the diagonal matrix such that $
    \mathbf{D}_{ii} = |N_G(i)|$,
 and the feature update rule used has the following form:
\begin{equation}\label{eqn:kipf-update}
    \mathbf{F}^{(t+1)} = \sigma\left(
        \mathbf{D}^{-1/2}\mathbf{A}\mathbf{D}^{-1/2}
        \mathbf{F}^{(t)}\mathbf{W}^{(t)}
    \right),
\end{equation}
where $\mathbf{D}^{-1/2}$
is the diagonal matrix with
$\mathbf{D}^{-1/2}_{ii} =
\frac{1}{\sqrt{\mathbf{D}_{ii}}}$. Compared with~(\ref{eqn:gnn}) this
is simpler update rule. It is remarked in~\cite{kipf-loose} that this update
rule is ``loosely speaking'' 1-WL. In this paper, we want to make this connection 
precise. More precisely, we show that
\begin{theorem}\label{pro:grohe}
  Let $(G,\ell)$ be a labelled graph. Then there exists a sequence of matrices such that for all $t \in \mathbb{N}$ and
  for $f^{(t)}$, as defined by Equation~\eqref{eqn:kipf-update} with $f^{(0)} = \ell$
  and $\sigma$ being ReLU, we have $f^{(t)} \equiv \ell^{(t)}$.
\end{theorem}
In other words, the GNNs by Kipf and Welling are as powerful as 1-WL.
\todo{F. Can we say anything about the upper bound of their expressive power.
I guess this follows from Grohe's paper?}

\subsection{Assumptions}
For notational convenience, we assume that all vertices have a non-empty
neighbourhood. This can be achieved, for instance, by introducing a new
vertex with a fresh new label and connecting all vertices without neighbours
to it.

\section{A linear-update architecture for unlabelled graphs}
Our first result is to ``simplify'' Proposition~\ref{pro:grohe} by simulating
their affine-update architecture using a linear update instead thus allowing
us to work with a single parameter matrix---at the price of having to extend
the feature vectors. We state below the formal claim.

\subsection{Look ma, one matrix}
Let us re-define the basic GNN we deal with. In each
layer $t \geq 0$, we compute a new feature vector
\[
    f^{(t+1)}(v) = \sigma\left(
        \sum_{w \in N_G(v)} f^{(t)}(w) \mathbf{W}^{(t)}
    \right)
\]
in $\mathbb{R}^{1 \times a_{t+1}}$ for $u$ where $\mathbf{W}^{(t)}$ is a
parameter matrix from $\mathbb{R}^{a_t \times a_{t+1}}$ and $\sigma$
is the ReLU activation function. Once more, we work with the matrix
form of the update:
\begin{equation}\label{eqn:gnn-linear}
    \mathbf{F}^{(t+1)} = \sigma\left(\mathbf{AF}^{(t)}\mathbf{W}^{(t)}\right).
\end{equation}

We will show that, if we set $f^{(t)}(v) := (\ell(v), 1)$ for all vertices
$v$, then GNNs with this architecture are also as
powerful as the $1$-WL algorithm.

\begin{proposition}
  Let $(G,\ell)$ be a labelled graph such that $\ell \equiv \lambda$. Then
  there exists a sequence of matrices such that for all $t \in \mathbb{N}$ and
  for $f^{(t)}$, as defined by Equation~\eqref{eqn:gnn-linear} with
  $f^{(0)}(v) = (\lambda, 1)$, for all $v \in V(G)$, and $\sigma$ being ReLU,
  we have $f^{(t)} \equiv \ell^{(t)}$.
\end{proposition}

\todo{G: I am up to here with cleaning a bit}

As a starting point, we re-establish Lemma 9
from~\cite{grohewl} for the ReLU function.
\begin{lemma}
  Let
  $\mathbf{B}\in \Nb^{s\times t}$ be a matrix in which all
  rows are pairwise disjoint (and no row consists entirely
  out of zeroes\footnote{I believe that this can be
  guaranteed in 1-WL}).\todo{G: with our extended features we actually guarantee this for free by adding the 1 column; also, t as dimension is a bad choice\ldots}
  Then there exists a matrix $\mathbf{X}$ and a constant $m$
  such that $\textsf{ReLU}(\mathbf{BX}-m\mathbf{J})$ is
  non-singular.
\end{lemma}
\begin{proof}
Let $M$ be the maximal entry in $\mathbf{B}$ and consider the column vector $\mathbf{z}=(1,M,M^2,\ldots,M^{t-1})^{\textsc{t}}$.
Then each entry in $\mathbf{b}=\mathbf{B}\mathbf{z}$ is positive and they are all pairwise distinct. Assume that $\mathbf{b}=(b_1,b_2,\ldots,b_t)^{\textsc{t}}\in\Rb^{s\times 1}$
such that $0< b_1< b_2<\cdots < b_s$. Consider the row vector $\mathbf{x}=\left(\frac{1}{b_1},\ldots,\frac{1}{b_s}\right)\in \Rb^{1\times s}$. Then, for $\mathbf{C}=\mathbf{b}\mathbf{x}$
$$
(\mathbf{C})_{ij}=\frac{b_i}{b_j}  \text{ and } (\mathbf{C})_{ij}=\begin{cases}  1 & \text{if $i=j$}\\
< 1 & \text{if $i<j$}\\
> 1 & \text{if $i>j$}.
\end{cases}
$$
Let $m$ be the greatest value  in $\mathbf{C}$ smaller than $1$.
% G: I think the m instantiated here is not correct
%, i.e., $m=\frac{b_s}{b_1}$.
Consider $\mathbf{D}=\mathbf{C}- m\mathbf{J}$.
Then,
$$
\mathbf{D}_{ij}=\frac{b_i}{b_j}- m \text{ and } (\mathbf{D})_{ij}=\begin{cases}  1-m & \text{if $i=j$} \\
\leq 0 & \text{if $i<j$}\\
>0  & \text{if $i>j$}.
\end{cases}
$$
As a consequence,
$$
\textsf{ReLU}(\mathbf{D})_{ij}=\begin{cases}  1-m & \text{if $i=j$}\\
0 & \text{if $i<j$}\\
>0  & \text{if $i>j$}.
\end{cases}
$$
This is an upper triangular matrix with (nonzero) value $1-m$ on its diagonal. It is therefore non-singular. So, the lemma is satisfied by taking $m$ as above and
%$m=b_s/b_1$ and % G: this still looks wrong
$\mathbf{X}=\mathbf{z}\mathbf{x}$.
\end{proof}

It is now easy to see that we can define weight matrices such that
\[
    \mathbf{F}^{(t+1)} = \textsf{ReLU}\left(\mathbf{AF}^{(t)}\mathbf{W}^{(t)}-m^{(t)}\mathbf{J}\right).
\]
is again equivalent to $c_\ell^{(t+1)}$.  We show that we modify $\mathbf{F}^{(t)}$ and $\mathbf{W}^{(t)}$
such that can rewrite the update rule in Kipf form.

Let $\mathbf{d}^{(t)}:=\mathbf{A}^{t}\mathbf{1}^\textsc{t}$. That is, $\mathbf{d}^{(t)}_v$ counts the paths from vertex $v$ of length $t$. We note that for undirected graphs $\mathbf{d}^{(t)}$ holds non-negative entries\footnote{For directed graphs, we may consider $I+A$ instead.}
We now use a similar construction of $\mathbf{W'}^{(t)}$ in the proof of Theorem~\ref{thm:simpler-grohe}, i.e.,
\[
\mathbf{W'}^{(t)}=\begin{pmatrix}
\mathbf{W}^{(t)} & \mathbf{0}_{d\times 1}\\
\left(-\frac{m^{(t)}}{d_1^{(t+1)}},\ldots,-\frac{m^{(t)}}{d_n^{(t+1)}}\right) & 1
\end{pmatrix}.
\]
and we replace $\mathbf{F}$ by $\mathbf{F'}=[\mathbf{F},\mathbf{1}]$. Then,
\begin{align}
    \mathbf{F'}^{(t+1)} 
        &=\textsf{ReLU}(\mathbf{A}\mathbf{F'}\mathbf{W'}^{(t)}) \nonumber \\
        &=[\textsf{ReLU}(\mathbf{A}\mathbf{F}^{(t)}\mathbf{W}^{(t)}-m^{(t)}\mathbf{J}),\textsf{ReLU}(\mathbf{d}^{(t+1)})]. \nonumber \\
        &=[\mathbf{F}^{(t+1)}, \mathbf{d}^{(t+1)}] \label{eq:Fc}
\end{align}
It now suffices to show that $\mathbf{F'}^{(t)}$ is equivalent to $c_\ell^{(t)}$. Since $\mathbf{F}^{(t)} \equiv c_\ell^{(t)}$ and by~\eqref{eq:Fc} $\mathbf{F'}^{(t+1)} \sqsubseteq \mathbf{F}^{(t+1)}$ it suffices to prove that $c_\ell^{(t)} \sqsubseteq \mathbf{F'}^{(t)}$.
This boils down to the following lemma.

\begin{lemma}\label{lem:deg-in-WL}
    Let $(G,c)$ be a labeled graph.
    Then for all $t \geq 0$ we have that 
    $c_\ell^{(t)} \sqsubseteq \mathbf{d}^{(t)}$. \filip{I think it's a bit ugly that $c$ is a function and $d$ is a vector but I find this statement better. Maybe we should define $c$ in bold (as a vector)?}
%     \[
%         c^{(t)}(u) = c^{(t)}(v) \implies d^{(t)}_u = d^{(t)}_v.
%     \]
\end{lemma}
\begin{proof}
Since $c_\ell^{(0)}$ assigns every vertex the same label, and $\mathbf{d}^{(0)}=\mathbf{1}^t$, our hypothesis holds for the base case.
%Filip: I started modifying here
For the induction step suppose that $c_\ell^{(t-1)}(v)=c_\ell^{(t-1)}(w)$ implies $\mathbf{d}^{(t-1)}_v=\mathbf{d}^{(t-1)}_w$.
Given a label $c$ we will use the notation $\mathbf{d}^{(t-1)}_c$, which is equal to $\mathbf{d}^{(t-1)}_v$ for any $v$ such that $c_\ell^{(t-1)} = c$. By the induction assumption this definition does not depend on the choice of $v$.

Take two vertices $v$ and $w$ such that $c_\ell^{(t)}(v)=c_\ell^{(t)}(w)$. By definition of the 1-WL algorithm 
$$
|N_G(v)\cap (c_\ell^{(t-1)})^{-1}(c)|=|N_G(w)\cap (c_\ell^{(t-1)})^{-1}(c)|
$$
for any label $c$ in ${\cal C}^{(t-1)}$ (i.e., the image of $c_\ell^{(t-1)}(V)$). Then
\begin{align*}
\mathbf{d}^{(t)}_v&=\sum_{x\in N_G(v)} \mathbf{d}^{(t-1)}_{x}\\
&=\sum_{c\in{\cal C}^{(t-1)}} |N_G(v)\cap (c_\ell^{(t-1)})^{-1}(c)|\mathbf{d}^{(t-1)}_{c}\\
&=\sum_{c\in{\cal C}^{(t-1)}} |N_G(w)\cap (c_\ell^{(t-1)})^{-1}(c)|\mathbf{d}^{(t-1)}_{c}\\
&=\sum_{y\in N_G(w)} \mathbf{d}^{(t-1)}_{y}\\
&=\mathbf{d}^{(t)}_w,
\end{align*}
as required.
%Floris' old stuff
% Suppose that $c_\ell^{(t-1)}(v)=c_\ell^{(t-1)}(w)$ implies that $\mathbf{d}^{(t-1)}_v=\mathbf{d}^{(t-1)}_w$.
% Take two vertices $v$ and $w$ such that $c_\ell^{(t)}(v)=c_\ell^{(t)}(w)$. This implies that for any label $c$ in ${\cal C}^{(t-1)}$ (i.e., the image of $c_\ell^{(t-1)}(V)$), 
% $$|N_G(v)\cap (c_\ell^{(t-1)})^{-1}(c)|=|N_G(w)\cap (c_\ell^{(t-1)})^{-1}(c)|.$$
% By induction, we know that for any two vertices $x$ and $y$ in $N_G(v)\cap (c_\ell^{(t-1)})^{-1}(c)$,
% $\mathbf{d}^{(t-1)}_{x}=\mathbf{d}^{(t-1)}_{y}$. Let us denote by $x_c$ an arbitrary vertex in 
% $N_G(v)\cap (c_\ell^{(t-1)})^{-1}(c)$. Then,
% \begin{align*}
% \mathbf{d}^{(t)}_v&=\sum_{x\in N_G(v)} \mathbf{d}^{(t-1)}_{x}\\
% &=\sum_{c\in{\cal C}^{(t-1)}} |N_G(v)\cap (c_\ell^{(t-1)})^{-1}(c)|\mathbf{d}^{(t-1)}_{x_c}\\
% &=\sum_{c\in{\cal C}^{(t-1)}} |N_G(w)\cap (c_\ell^{(t-1)})^{-1}(c)|\mathbf{d}^{(t-1)}_{x_c}\\
% &=\sum_{c\in{\cal C}^{(t-1)}} |N_G(w)\cap (c_\ell^{(t-1)})^{-1}(c)|\mathbf{d}^{(t-1)}_{y_c}\\
% &=\sum_{y\in N_G(w)} \mathbf{d}^{(t-1)}_{y}\\
% &=\mathbf{d}^{(t)}_w,
% \end{align*}
% where $y_c$ denotes an arbitrary vertex in  $N_G(w)\cap (c_\ell^{(t-1)})^{-1}(c)$ and hence, by induction,
% $\mathbf{d}^{(t-1)}_{x_c}=\mathbf{d}^{(t-1)}_{y_c}$.
\end{proof}

\begin{theorem}\label{thm:denorm-kipf}
    Theorem~\ref{thm:simpler-grohe} holds as well 
    %(modulo, perhaps, additional layers needed) 
    if $\sigma$ is the ReLU
    activation function.
\end{theorem}

\section{Labelled graphs}
\todo{G: we need to argue that what we worked out in the previous section
works for labelled graphs too (even if only for ReLU...)}


\section{Normalized convolutional architecture}
Theorem~\ref{thm:denorm-kipf}
can be seen as a
``denormalized'' version of the update rule introduced
in~\cite{kipf-loose}.

Let $\mathbf{D}$ be the \emph{degree matrix} of $G$, that is $\mathbf{D}$
is the diagonal matrix such that
\[
    \mathbf{D}_{ii} = |N_G(i)|.
\]
In this section we consider the following update rule
\begin{equation}
    \mathbf{H}^{(t+1)} = \sigma\left(
        \mathbf{D}^{-1/2}\mathbf{A}\mathbf{D}^{-1/2}
        \mathbf{H}^{(t)}\mathbf{W}^{(t)}
    \right),
\end{equation}
where $\mathbf{D}^{-1/2}$
is the diagonal matrix with
$\mathbf{D}^{-1/2}_{ii} =
\frac{1}{\sqrt{\mathbf{D}_{ii}}}$.

\subsection{(Right-)half Kipf}
We will repeat the argument leading to Theorem~\ref{thm:denorm-kipf} in order
to prove that we can modify $\mathbf{F}^{(t)}$ and $\mathbf{W}^{(t)}$
such that
\begin{equation}\label{eqn:half-kipf}
    \mathbf{F}^{(t+1)} \equiv
    \mathbf{F'}^{(t+1)} :=
    \textsf{ReLU}\left(\mathbf{AD}^{-1/2}\mathbf{F'}^{(t)}\mathbf{W'}^{(t)}\right).
\end{equation}
Once more, we let $\mathbf{F'}^{(0)} = [\mathbf{F}^{(0)},
\mathbf{1}^\textsc{T}]$. For the weight matrices, we define
\[
    \mathbf{W'}^{(t)}=
    \begin{pmatrix}
        \mathbf{W}^{(t)} & \mathbf{0}_{d\times 1}\\
        \left(
            -\frac{m^{(t)}\sqrt{d_1^{(1)}}}{d_1^{(t+1)}},
            \ldots,
            -\frac{m^{(t)}\sqrt{d_n^{(1)}}}{d_n^{(t+1)}}
        \right) & 1
    \end{pmatrix}.
\]
It is easy to verify that
\begin{align}
    \mathbf{F'}^{(t+1)} = [\textsf{ReLU}(\mathbf{AF}^{(t)}\mathbf{W}^{(t)} - m^{(t)}\mathbf{J}),
    \mathbf{d'}^{(t+1)}] \nonumber \\
    = [\mathbf{F}^{(t+1)},\mathbf{d'}^{(t+1)}] \label{eq:FF'}
\end{align}
where $\mathbf{d'}^{(t+1)}$ is the column vector with
\[
    d'^{(t+1)}_i = \frac{d^{(t+1)}_i}{\sqrt{d^{(1)}_i}}.
\]
By~\eqref{eq:FF'} we have $\mathbf{F'}^{(t+1)} \sqsubseteq \mathbf{F}^{(t+1)} \equiv c_\ell^{(t+1)}$. To prove~\eqref{eqn:half-kipf} it only remains to prove that $c_\ell^{(t+1)} \sqsubseteq \mathbf{F'}^{(t+1)}$.
By Lemma~\ref{lem:deg-in-WL} we know that $c_\ell^{(t+1)} \sqsubseteq d^{(t+1)}$. The proof follows since $d^{(t)}$ is a sequence of refinements and thus $d^{(t+1)}_i \sqsubseteq d^{(1)}_i$.

\subsection{The full Kipf}
Let $h^{(t)}$ be a vertex labelling obtained by applying
the update rule from Equation~\eqref{eqn:kipf-update}. We will
presently 
make use of Equation~\eqref{eqn:half-kipf} to prove the following
claim.

\begin{theorem}
    Let $(G,\ell)$ be a labelled graph and $h^{(0)}$ be equivalent
    to $c_\ell^{(0)}$. Then for all $t \geq 0$
    there exists a sequence of weights $\mathbf{W}^{(t)}$ such that
    $h^{(t)}$, as defined by Equation~\eqref{eqn:kipf-update},
    is equivalent to $c^{(t)}$.
\end{theorem}
\begin{proof}
  Let $\mathbf{M}$ be the matrix $\mathbf{AD}^{-1/2}\mathbf{F'}^{(t)}\mathbf{W'}^{(t)}$
  and recall that $\mathbf{M}$ is row independent modulo equality.
  \todo{G. row independence modulo equality is not yet explicit about $AFW-mJ$
  right? I seem to need it here}
  Since $\mathbf{D}^{-1/2}$ is a diagonal matrix with positive entries, we
  have that an entry of $\mathbf{D}^{-1/2}\mathbf{M}$ is negative if and only
  if it is negative in $\mathbf{M}$. Hence, by the definition of
  $\mathsf{ReLU}$ and Equation~\eqref{eqn:half-kipf}, to prove the desired
  claim it suffices to argue that any two rows in
  $\mathbf{N} := \mathbf{D}^{-1/2}\mathbf{M}$ are equal if and only if they
  are also equal in $\mathbf{M}$.
  
  Note that if $\mathbf{D}_{ii} = \mathbf{D}_{jj}$ then $\mathbf{N}_i =
  \mathbf{N_j}$ if and only if $\mathbf{M}_i = \mathbf{M}_j$. From (the
  contrapositive of) Lemma~\ref{lem:deg-in-WL} we know that if
  $\mathbf{D}_{ii} \neq \mathbf{D}_{jj}$ then the rows are not equal in
  $\mathbf{M}$. Further, if
  \[
    \mathbf{N}_{i} = d_i \mathbf{M}_i = d_j \mathbf{M}_j = \mathbf{N}_j
  \]
  then, since $d_i,d_j > 0$, we know that $\mathbf{M}$ is not row independent
  modulo equality. The latter contradicts our initial assumptions.
  \filip{I think we need to add to Lemma~1 that the constructed matrix is upper-triangular. Then we are using here the fact that an upper triangular matrix is non-singular iff the elements on its diagonal are nonzero. BTW I still don't understand the last paragraph in this proof.}
\end{proof}

\subsection{Beyond Kipf}
This begs the
following questions:
\begin{itemize}
    \item can we show an analogue of Grohe's theorem 1 for GCNs? i.e. are they always at most as powerful as the 1-WL algorithm? 
    \item can one define $k$-tuple graph convolutional networks (GCNs) and show
they are as powerful as the $k$-WL algorithm
following~\cite[Proposition 4]{grohewl}?
    \item can we show that Kipf's architecture is not 1-WL powerful without
      extending the feature vectors? this would mean that they are crucially
      lacking the bias!
\end{itemize}


\section{Other Questions}
The $k$-GNN implementation from~\cite{grohewl} uses the \emph{negative log-likelihood} loss function.\footnote{See
\url{https://github.com/chrsmrrs/k-gnn/blob/master/examples/1-2-3-imdb.py\#L118}.} On
the other hand, the GCN proposal by Kipf and Welling uses a semi-supervised
loss function.
\begin{quote}
    Is one of these loss functions guaranteeing that sufficient training
    will almost surely lead to learning a NN version of the WL algorithm?
\end{quote}

How do things change when considering \emph{directed} graphs? What is even
the proper notion of $1$-WL on such graphs.

\paragraph{Initial musings on the NLL.}
Using the negative log-likelihood loss function should guarantee that
training with ever larger data-sets labelled according to the $1$-WL
gets us ever closer to a NN version of the $1$-WL algorithm. 
\begin{quote}
    What if the
    data is more precisely labelled than what the $1$-WL
    algorithm yields? Do we
    lose convergence?
\end{quote}

\begin{quote}
\begin{itemize}
    \item Is it possible to express 1-WL with only linear transformations?
    \item Is it true that $u,v$ have the same color after k-steps in 1-WL iff for every $i \le k$ and every node $x \in G$ the number of paths from $u$ to $x$ is the same as the number of paths from $v$ to $x$ (this shouldn't be the same $x$ but some $\rho(x)$ for some permuation of vertices $\rho$.
    \item What if we replace $D^{-1/2}AD^{-1/2}$ with a Jordanian
      mutiplication? I.e. $A \circ D^{-1} = (AD^{-1} + D^{-1}A)/2$
\end{itemize}{}
\end{quote}

\section*{Acknowledgements}
Funding acks go here
\appendix
%!TEX root =main.tex



\newpage

\section{Lower bounding the expressive power}

In this paper we consider the following architectures. Below, $\sigma$ denotes a non-linear activation function such as sgn or ReLU. 

\begin{definition}\label{def:label}
Let $\mathbf{F}$ be a labeling defined by an $n\times q$-matrix. 
We say that $\mathbf{F}$ is good with respect to another labeling $\mathbf{F}'$ if:
\begin{enumerate}
\item[(a)] \textit{row-independent modulo equality}, i.e., the unique row vectors in $\mathbf{F}$ are linearly independent; and
\item[(b)] 
% the vertex labelling induced by  $\mathbf{F}^{(t)}$ is \textit{equivalent} to the vertex labelling induced by  $\mathbf{c}^{(t)}$
% obtained by applying 1-WL on the graph with vertex labelling induced by $\mathbf{F}^{(t-1)}$. (If $t=0$,
The vertex labelling induced by $\mathbf{F}'$ is coarser than  $\mathbf{F}$.
\end{enumerate}
\end{definition}
Most of the time we will use this definition for $\mathbf{F}' = \mathbf{D} \mathbf{1}_{n \times 1}$.
\todo{Floris: What do you mean by this? What is $ \mathbf{D} \mathbf{1}_{n \times 1}$?}
\paragraph*{Graph neural networks}
A graph neural network (GNN) model consists of layers, where each layer specifies how to update the vertex labelling $\mathbf{F}$. A GNN with $k$ layers is defined by updates of $\mathbf{F}^{(t)}$ for $t = 0, \ldots,k$, which denotes the labelling obtained after $t$ layers. A new labelling $\mathbf{F}^{(t+1)}$ is obtained inductively by transformations defined on the previous labelling $\mathbf{F}^{(t)}$. An \emph{architecture} specifies what kind of transformations are allowed. In this paper we consider the following architectures. Below, $\sigma$ denotes a non-linear activation function such as sgn or ReLU. 
% $$
% \mathbf{F}^{(t+1)} = \sigma\left(\mathbf{A}(p,q)\mathbf{F}^{(t)}\mathbf{W}_1^{(t)}+ r\mathbf{B}(p,q)
% \right)
% $$
% with 
% $$
% \mathbf{A}(p,q):=
% \left((p+q)\mathbf{I}+(1-(p+q))\mathbf{D}\right)^{-1/2}(\mathbf{A}+q\mathbf{I})
% $$

\begin{itemize}
 \item \emph{Basic architecture.} (See e.g.~\cite{hyl17})
\begin{equation}\label{architecture:basic}
  \mathbf{F}^{(t+1)} = \sigma\left(
   \mathbf{AF}^{(t)}\mathbf{W}_1^{(t)} +
    \mathbf{F}^{(t)}\mathbf{W}_2^{(t)} +
    \mathbf{W}_3^{(t)}
  \right),
\end{equation}
where $\mathbf{W}_1^{(t)}, \mathbf{W}_2^{(t)} \in \Rb^{(q \times q')}$ are weight matrices.
% and $\sigma$ is a nonlinear function usually ReLU.
\item \emph{Normalised architecture.} 
\begin{equation}\label{architecture:normalised}
  \mathbf{F}^{(t+1)} = \sigma\left(
   \mathbf{D}^{-1/2}\mathbf{AD}^{-1/2}\mathbf{F}^{(t)}\mathbf{W}_1^{(t)} +
    \mathbf{F}^{(t)}\mathbf{W}_2^{(t)} +
    \mathbf{W}_3^{(t)}
  \right),
\end{equation}
which differs from the basic architecture by normalising the adjacency matrix using the degree matrix $\mathbf{D}$.
\item \emph{Kipf-Welling architecture.}
\begin{equation}\label{architecture:kipf}
  \mathbf{F}^{(t+1)} = \sigma\left(
   \tilde{\mathbf{D}}^{-1/2}\tilde{\mathbf{A}}\tilde{\mathbf{D}}^{-1/2}\mathbf{F}^{(t)}\mathbf{W}_1^{(t)}
  \right),
\end{equation}
where $\tilde{\mathbf{A}} = \mathbf{A} + \mathbf{I}$ and $\tilde{\mathbf{D}}$ is the diagonal matrix with degrees of $\tilde{\mathbf{A}}$.
\item \emph{Kipf-Welling architecture with bias.}
\begin{equation}\label{architecture:kipfbiased}
  \mathbf{F}^{(t+1)} = \sigma\left(
   \tilde{\mathbf{D}}^{-1/2}\tilde{\mathbf{A}}\tilde{\mathbf{D}}^{-1/2}\mathbf{F}^{(t)}\mathbf{W}_1^{(t)} +
    \mathbf{W}_3^{(t)}
  \right),
\end{equation}
which is the same as the Kipf-Welling architecture but extended with a bias $\mathbf{W}_3^{(t)}$.
\item \emph{Symmetric normalisation.}
\begin{equation}\label{architecture:symmetric}
  \mathbf{F}^{(t+1)} = \sigma\left(
   \frac{\mathbf{D}^{-1}\mathbf{A} + \mathbf{AD}^{-1}}{2} \mathbf{F}^{(t)}\mathbf{W}_1^{(t)} +
    \mathbf{F}^{(t)}\mathbf{W}_2^{(t)} +
    \mathbf{W}_3^{(t)}
  \right),
\end{equation}
which is the same as normalised architecture, but the normalisation is achieved with $\frac{\mathbf{D}^{-1}\mathbf{A} + \mathbf{AD}^{-1}}{2}$. Notice that since $\mathbf{D}$ and $\mathbf{A}$ are both symmetric this always results in a symmetric matrix (whereas for example $\mathbf{D}^{-1}\mathbf{A}$ does not need to be symmetric).
\item \emph{Linear architectures.} These are all variants of the previous architectures, where the nonlinear part $\sigma$ is removed.
\end{itemize}

q\todo{floris: This can be further complemented with random walk $\mathbf{D}^{-1}\mathbf{A}$, augmented random walks $\tilde{\mathbf{D}}^{-1}\tilde{\mathbf{A}}$ as in \cite{Wu2019}.}

\todo{floris: also, aren't we missing ``the''architecture we want to put forward, i.e., the variation of Kipf with perturbed $\mathbf{I}$ and special bias??}
Notice that~\eqref{architecture:kipf} is a particular case of~\eqref{architecture:kipfbiased} when $\mathbf{W}_3^{(t)}$ is a zero matrix. Otherwise, all architectures are probably incomparable.
\todo{Filip: Do we want to formalise this last statement (is this even true)?}
\todo{Guillermo: It is not really as important as knowing which architectures can simulate the 1WL. For now Kipf-Welling is the only one for which we do not have a positive answer but that implies nothing about how it compares to the others.}
\todo{Filip: Hopefully we could write something like ``in practise they are different and give different results''. Otherwise we should come up with some justification for introducing all these architectures.}

\paragraph*{Weisfeiler-Leman Algorithm}%there should be more about this here
When executing 1-WL on $G$, we denote the corresponding vertex labelling in iteration $t$ by the $n\times 1$-matrix (column vector) 
$\mathbf{c}^{(t)}$, i.e., the entry $\mathbf{c}^{(t)}_v$ is a number encoding the colour of vertex $v$ after $t$ iterations of 1-WL.
\todo[inline]{Guillermo: the notion of a number encoding a color is vague here, I guess we want to have that the induced labelling is equivalent to the colouring.}
\todo{Filip: once we write properly the WL paragraph I would just never talk about coloring, only about labelling. Or define coloring as labelling for $q = 1$.}

\begin{definition}\label{def:gen+bias}
Let $\mathbf{F}^{(0)}$ be a feature matrix satisfying Definition~\ref{def:label} for $\mathbf{F}' = \mathbf{c}^{(k)}_v$ for some given $k$.
We say that an architecture is 1-WL strong if for every $t\geq 0$ there exist weight matrices $\mathbf{W}^{(t)}$
% and constant $m^{(t)}$
such that the vertex labelling induced by $\mathbf{F}^{(t)}$ defined in this architecture
is equivalent to the vertex labelling after $k+t$ iterations of 1-WL, starting from a uniform vertex labelling.
\end{definition}

\section{Nodes labelled with respect to $1$-WL.}
We first  consider unlabelled graphs, i.e., labelled graphs $(G,\mathbf{l})$ with $\mathbf{l}$ an $n\times 1$-vector, such that the
vertex labelling induced by $\mathbf{l}$ assigns some labelling coarser than $\mathbf{c}^{(k)}_v$ for some $k$.
% The GNNs which we will
% consider are closely related to GNNs with update rules of the form (see also~\cite{hyl17}):
% \begin{equation}\label{eqn:gnn2}
%   \mathbf{F}^{(t+1)} = \sigma\left(
%     \mathbf{F}^{(t)}\mathbf{W}_1^{(t)} +
%     \mathbf{AF}^{(t)}\mathbf{W}_2^{(t)}
%   \right),
% \end{equation}
% where $\mathbf{F}^{(t)}$ are the feature vectors, $\mathbf{A}$ is an adjacency matrix of
% an undirected graph, and $\mathbf{W}_1^{(t)}$ and $\mathbf{W}_2^{(t)}$ are weight matrices
% (which are to be learned).

It was shown in~\cite{grohewl} that the basic architecture can perform as good as 1-WL.
More specifically, that there is a sequence
$(\mathbf{W}_1^{(t)})_{t\in\mathbb{N}}$ of weight matrices in $\Rb^{n\times n}$ such that
the vertex labelling induced by
\begin{equation}\label{eqn:grohegnn}
  \mathbf{F}^{(t+1)} = \text{sign}\left(
    \mathbf{A}\mathbf{F}^{(t)}\mathbf{W}_1^{(t)} - \mathbf{J}  \right)
\end{equation}
gives $\mathbf{F}^{(t)}$ equivalent to $\mathbf{c}^{(t)}$ for all $t$ (assuming that both $\mathbf{F}^{(0)}$ and $\mathbf{c}^{(0)}$ start with a labelling coarser than $\mathbf{c}^{(k)}_v$ for some $k$).
% is equivalent to the vertex labelling induced by $\mathbf{c}^{(t+1)}$, computed by 1-WL, starting from the initial uniform vertex labelling $\mathbf{c}^{(0)}=\mathbf{l}$.
This is a particular case of the basic architecture~\eqref{architecture:basic}, where the matrix $\mathbf{W}_2^{(t)}$ is not needed
in any iteration, and $\mathbf{W}_3^{(t)} = -\mathbf{J}$ is fixed for all iterations (where $\mathbf{J}$ denotes the matrix consisting of entries all equal to one).
Moreover, when $\sigma$ is taken to be ReLU, then the vertex labelling induced by  $\mathbf{F}^{(2t)}$ 
is shown to correspond to the vertex labelling induced by $\mathbf{c}^{(t)}$, computed by 1-WL~\cite{grohewl}. The factor $2$ originates
from a simulation of $\text{sign}(\cdot)$ by means of a two-fold application of ReLU.

% \subsubsection{Grohe with some spice.}
% We start by describing (and slightly  generalizing) the proof strategy used
% in~\cite{grohewl}.

In Sections~\ref{subsec:left} and~\ref{subsec:right} we will generalise the result of~\cite{grohewl} proving that the architectures~\eqref{architecture:normalised} and~\eqref{architecture:kipfbiased} also have these properties. For this we will introduce a uniform architecture that captures all architectures~\eqref{architecture:basic}, \eqref{architecture:normalised} and~\eqref{architecture:kipfbiased}. We do not know whether a similar result holds for the remaining architecture~\eqref{architecture:kipf}.
But first, in Section~\ref{subsec:relu} we strengthen one of the results from~\cite{grohewl}.

\subsection{ReLU}\label{subsec:relu}
We show that instead of simulating the sign function by means of ReLU, one can directly
use ReLU by means of a minor modification of the proof given in~\cite{grohewl}. 
As a consequence, we avoid the factor $2$ in the correspondence between the 1-WL
vertex labelling and the labelling induced by the feature vectors.
An inspection of the proof given in~\cite{grohewl} shows that it suffices to re-establish Lemma 9 from~\cite{grohewl} for the ReLU function. \begin{lemma}\label{lem:relulemma9}
  Let
  $\mathbf{B}\in \Nb^{p\times q}$ be a matrix in which all
  rows are pairwise disjoint and such that no row consists entirely
  out of zeroes\footnote{Compared to Lemma 9,
 we additionally require non-zero rows. This can be guaranteed provided that there are no isolated vertices}.
%  \footnote{I believe that this can be
%  guaranteed in 1-WL}).\todo{G: with our extended features we actually guarantee this for free by adding the 1 column; also, t as dimension is a bad choice\ldots}
  Then there exists a matrix $\mathbf{X}$ and a constant $m$
  such that $\text{\normalfont ReLU}(\mathbf{BX}-m\mathbf{J})$ is
  non-singular.
\end{lemma}
\begin{proof}
Let $M$ be the maximal entry in $\mathbf{B}$ and consider the column vector $\mathbf{z}=(1,M,M^2,\ldots,M^{q-1})^{\textsc{t}}$.
Then each entry in $\mathbf{b}=\mathbf{B}\mathbf{z}$ is positive and they are all pairwise distinct. 
Let $\mathbf{P}$ be a permutation matrix in $\Rb^{p\times p}$ such that $\mathbf{b}'=\mathbf{P}\mathbf{b}$ is such that  $\mathbf{b}'=(b_1',b_2',\ldots,b_p')^{\textsc{	t}}\in\Rb^{p\times 1}$ with $ b_1'> b_2'>\cdots > b_p'>0$. 
Consider the $\mathbf{x}=\left(\frac{1}{b_1'},\ldots,\frac{1}{b_p'}\right)\in \Rb^{1\times p}$. Then, for $\mathbf{C}=\mathbf{b}'\mathbf{x}$
$$
\mathbf{C}_{ij}=\frac{b_i'}{b_j'}  \text{ and } \mathbf{C}_{ij}=\begin{cases}  1 & \text{if $i=j$}\\
>1 & \text{if $i<j$}\\
< 1 & \text{if $i>j$}.
\end{cases}
$$
Let $m$ be the greatest value  in $\mathbf{C}$ smaller than $1$.
% G: I think the m instantiated here is not correct
%, i.e., $m=\frac{b_s}{b_1}$.
Consider $\mathbf{E}=\mathbf{C}- m\mathbf{J}$.
Then,
$$
\mathbf{E}_{ij}=\frac{b_i'}{b_j'}- m \text{ and } \mathbf{E}_{ij}=\begin{cases}  1-m & \text{if $i=j$} \\
> 0 & \text{if $i<j$}\\
\leq 0  & \text{if $i>j$}.
\end{cases}
$$
As a consequence,
$$
\text{ReLU}(\mathbf{E})_{ij}=\begin{cases}  1-m & \text{if $i=j$}\\
>0 & \text{if $i<j$}\\
0  & \text{if $i>j$}.
\end{cases}
$$
This is an upper triangular matrix with (nonzero) value $1-m$ on its diagonal. It is therefore non-singular. 
We observe that $\mathbf{Q}\text{ReLU}(\mathbf{E})=\text{ReLU}(\mathbf{Q}\mathbf{E})$ for any row permutation $Q$. Furthermore, non-singularity is preserved under row permutations and $\mathbf{Q}\mathbf{J}=\mathbf{J}$. Hence, if we define $\mathbf{X}=\mathbf{z}\mathbf{x}$ and use the permutation matrix $\mathbf{P}$:
\begin{align*}
\mathbf{P}\text{ReLU}(\mathbf{B}\mathbf{X}-m\mathbf{J})&=
\text{ReLU}(\mathbf{P}\mathbf{B}\mathbf{z}\mathbf{x}-m\mathbf{P}\mathbf{J})=\text{ReLU}(\mathbf{E}-m\mathbf{J}),
\end{align*}
we have that $\text{ReLU}(\mathbf{B}\mathbf{X}-m\mathbf{J})$ is non-singular, as desired.
%So, the lemma is satisfied by taking $m$ as above and
%%$m=b_s/b_1$ and % G: this still looks wrong
%$\mathbf{X}=\mathbf{z}\mathbf{x}$.
\end{proof}

As a consequence of the argument presented above, we know that Lemma~\ref{lem:relulemma9} holds for all $m$ such that $m<1$ and such that
$m$ is an upper bound on the elements smaller than $1$ in $$(\mathbf{B}\mathbf{z})^{\textsc{t}}\mathbf{x},$$
with $\mathbf{z}=[1,M,M^2,\ldots,M^{q-1}]^{\textsc{t}}$ and $M$ an upper bound on the elements in $\mathbf{B}$, and where $\mathbf{x}$ consist of the reciprocals of the entries in $\mathbf{B}\mathbf{z}$. We will apply Lemma~\ref{lem:relulemma9} in each layer of our GNN architecture, i.e., to $\mathbf{B}^{(t)}$ for every $t\geq 0$. We next argue that we can fix $m$ uniformly across all these layers.

We start by observing that elements in $\mathbf{B}^{(t)}$ can be upper bounded.
\begin{lemma}\label{lemma:bound-B-unlabbeled}
For all $t\geq 0$ and all $v,c$, $\mathbf{B}^{(t)}_{vc}\leq n$ where $n$ is the dimension of the adjacency matrix $\mathbf{A}$ of $G$.
\end{lemma}
\begin{proof}
It suffices to note that $\mathbf{B}^{(t)}_{vc}$ is equal to the number of neighbors of $v$ of colours $c$. Clearly, there are at most $n$ neighbours.
\todo{G: true, but at this point we have not defined/shown that $\mathbf{B}$ is really encoding that information}
\end{proof}

It follows from
Lemma~\ref{lemma:bound-B-unlabbeled} and the choice of $m$ in the proof that, if we know
a bound on the number of layers of the architecture 
\todo{F. Do we need to know the number of layers?}
\todo{G. I guess not, the bound on their dimensions should suffice}
in advance and if we know
the size of the parameter matrices too, then Proposition~\ref{pro:gen+bias} can
be stated for a single constant $m$.
\begin{proposition}\label{pro:fixed-m}
    Let $(\mathbf{W}^{(i)})^t_{i=0}$ be a matrix sequence such that the dimensions
    of all $\mathbf{W}^{(i)}$ are at most $q$ and set
    \[
        m := \frac{qn^{q+1} - 1}{qn^{q+1}}.
    \]
    Then, there exist matrices $(\mathbf{Z}^{(i)})_{i=0}^t$ such that $\mathrm{ReLU}(\mathbf{B}^{(i)}\mathbf{Z}^{(i)} - m\mathbf{J})$ consists of linearly independent rows for all $0 \leq i \leq t$.
\end{proposition}
\todo{F. Why use $\mathbf{W}$ instead $\mathbf{B}$. The upper bound $t$ (number of layers) does not pop up in the expression for $m$..}
\todo{G. But the maximal dimension of $\mathbf{W}$ does pop up, so we need to make them explicit to mention the bound. I guess, B must also be made explicit for the proposition to make sense.}
\begin{proof}
\todo{F: Add argument outlined in Guillermo's email.}
Let b/c be such that b,c <= U and 0 < b < c. Furthermore, b and c are integers. We want to prove that b/c <= (U-1)/U which holds iff bU <= cU – c. Note that, since b and c are integers and b < c it suffices to prove that (c-1)U <= c(U-1). The latter holds iff cU – U <= cU – c iff c <= U which holds by assumption.
\end{proof}

\subsection{Good-for-left-multiplication matrices.}\label{subsec:left}
\begin{definition}\label{def:gfl}
Let  $\mathbf{Y}^{(t)}$ be a positive diagonal $n\times n$-matrix. We say that $\mathbf{Y}^{(t)}$ is \emph{good-for-left-multiplication}, GFL for short, if for any $i,j\in[1,n]$
\begin{equation}
\mathbf{F}_{i\bullet}^{(t)}=\mathbf{F}_{j\bullet}^{(t)} \Longrightarrow \mathbf{Y}^{(t)}_{ii}=\mathbf{Y}^{(t)}_{jj}. \label{eq:cond1}
\end{equation}
\end{definition}

In this section we prove that we can strengthen the results of~\eqref{eqn:grohegnn}\todo{G. The results of an equation?} as follows. For any sequence of GFL matrices $\mathbf{Y}^{(t)}$ and $\mathbf{F}^{(0)}$ satisfying Definition~\ref{def:label} there exist sequences of $\mathbf{W}^{(t)}$ and $m^{(t)}$ such that
\begin{equation}
\mathbf{F}^{(t+1)}:=\sigma(\mathbf{A}\mathbf{Y}^{(t)}\mathbf{F}^{(t)}\mathbf{W}^{(t)} - m^{(t)}\mathbf{J})
\end{equation}
is equivalent to 1-WL.

Let $U\subseteq V$ denote  a set of, say $p$,  vertices corresponding to the unique rows in $\mathbf{F}^{(t)}$ and define
 $\widetilde{\mathbf{F}^{(t)}}$ as the $p\times q$-matrix consisting of  the row vectors $\mathbf{F}^{(t)}_{v\bullet}$, for $v\in U$.
\begin{lemma}\label{lem:gfl}
  Let $\mathbf{F}^{(t)} \in \mathbb{R}^{n \times q}$ satisfy Definition~\ref{def:label} and let $\mathbf{Y}^{(t)}$
  be a $n\times n$ GFL matrix. Then, $\mathbf{Y}^{(t)}\mathbf{F}^{(t)}$ also satisfies Definition~\ref{def:label}, and furthermore the labelings induced by $\mathbf{Y}^{(t)}\mathbf{F}^{(t)}$ and $\mathbf{F}^{(t)}$ are equivalent.
\end{lemma}
\begin{proof}
	Let $U$ be a set of $p$ vertices identifying unique rows in $\mathbf{F}^{(t)}$, as described above.
We claim  that $\widetilde{\mathbf{Y}^{(t)}\mathbf{F}^{(t)}}$, the matrix consisting of the unique
rows in $\mathbf{Y}^{(t)}\mathbf{F}^{(t)}$, is the $p\times q$-matrix consisting of vectors $\mathbf{Y}^{(t)}_{vv}\mathbf{F}^{(t)}_{v\bullet}$, for $v\in U$.  Indeed, we first observe that 
$$\mathbf{Y}^{(t)}_{vv}\mathbf{F}^{(t)}_{v\bullet}\neq \mathbf{Y}^{(t)}_{ww}\mathbf{F}^{(t)}_{w\bullet}$$
 for any $v, w\in U$ such that $v\neq w$. Indeed, otherwise $\mathbf{F}^{(t)}_{v\bullet}$ and $\mathbf{F}^{(t)}_{w\bullet}$ would
 be distinct rows in $\mathbf{F}^{(t)}$ which are linearly dependent. This contradicts our assumption that $\mathbf{F}^{(t)}$ satisfies condition (a).
Hence, $\widetilde{\mathbf{Y}^{(t)}\mathbf{F}^{(t)}}$ surely contains the row vectors $\mathbf{Y}^{(t)}_{vv}\mathbf{F}^{(t)}_{v\bullet}$ for all $v\in U$. Since~(\ref{eq:cond1}) implies that the same rows in $\mathbf{F}^{(t)}$ get scaled in the same way, no other unique rows can exist in $\mathbf{Y}^{(t)}\mathbf{F}^{(t)}$. We also note that the rows in $\widetilde{\mathbf{Y}^{(t)}\mathbf{F}^{(t)}}$ are linearly
independent as well. In other words, condition (a) continues to hold for $\mathbf{Y}^{(t)}\mathbf{F}^{(t)}$. In fact, we have just shown that 
$$\mathbf{F}_{v\bullet}^{(t)}=\mathbf{F}_{w\bullet}^{(t)} \Longleftrightarrow \mathbf{Y}^{(t)}_{vv}\mathbf{F}_{v\bullet}^{(t)}=\mathbf{Y}^{(t)}_{ww}\mathbf{F}_{w\bullet}^{(t)}.$$
In other words, the vertex labelling induced by $\mathbf{Y}^{(t)}\mathbf{F}^{(t)}$
is equivalent to the vertex labelling induced by $\mathbf{F}^{(t)}$. Thus, because of condition (b), it is also equivalent to the labelling induced by $\mathbf{c}^{(t)}$ obtained by applying $1$-WL on $\mathbf{F}^{(t-1)}$. So condition (b) holds as well for $\mathbf{Y}^{(t)}\mathbf{F}^{(t)}$.
\end{proof}

\paragraph{Constructing $\mathbf{F}^{(t+1)}$.}
The independence of the vectors in $\widetilde{\mathbf{Y}^{(t)}\mathbf{F}^{(t)}}$ guarantees the existence of 
a $q\times p$-matrix $\mathbf{M}^{(t)}$ satisfying
$$
\widetilde{\mathbf{Y}^{(t)}\mathbf{F}^{(t)}}\mathbf{M}^{(t)}=\mathbf{I}_{p\times p},
$$
where $\mathbf{I}_{p\times p}$ denotes the $p\times p$ identity matrix. Indeed, we can let 
$$
\mathbf{M}^{(t)}=(\widetilde{\mathbf{Y}^{(t)}\mathbf{F}^{(t)}})^{\textsc{t}}\bigl(\widetilde{\mathbf{Y}^{(t)}\mathbf{F}^{(t)}}(\widetilde{\mathbf{Y}^{(t)}\mathbf{F}^{(t)}})^{\textsc{t}}\bigr)^{-1},
$$
where the invertibility of $\widetilde{\mathbf{Y}^{(t)}\mathbf{F}^{(t)}}(\widetilde{\mathbf{Y}^{(t)}\mathbf{F}^{(t)}})^{\textsc{t}}$ is guaranteed because of row-independence of $\widetilde{\mathbf{Y}^{(t)}\mathbf{F}^{(t)}}$. We next consider
$\mathbf{B}^{(t)}=\mathbf{A}\mathbf{Y}^{(t)}\mathbf{F}^{(t)}\mathbf{M}^{(t)}$ and observe that
\begin{equation}\label{eqn:counting-neighbors}
(\mathbf{B}^{(t)})_{vc}=(\mathbf{A}\mathbf{Y}^{(t)}\mathbf{F}^{(t)}\mathbf{M}^{(t)})_{vc}=\sum_{w} \mathbf{A}_{vw} \delta_{w,c}
\end{equation}
where $\delta_{w,c}=1$ if $(\mathbf{Y}^{(t)}\mathbf{F}^{(t)})_{w\bullet}=(\mathbf{Y}^{(t)}\mathbf{F}^{(t)})_{c\bullet}$
and $\delta_{w,c}=0$ otherwise. In other words, $\mathbf{B}^{(t)}_{vc}$ is the number of vertices adjacent to $v$ which are assigned, in the vertex labelling induced by 
$ \mathbf{Y}^{(t)}\mathbf{F}^{(t)}$, the label $(\mathbf{Y}^{(t)}\mathbf{F}^{(t)})_{c\bullet}$.
If we consider one-step of 1-WL, starting from the vertex labelling induced by $\mathbf{F}^{(t)}$, then
the obtained vertex labelling is equivalent to the vertex labelling induced by $\mathbf{B}^{(t)}$.

It will be useful later for us to have an upper bound on the entries of $\mathbf{B}^{(t)}$. The following bound follows directly from Equation~\eqref{eqn:counting-neighbors}.
\begin{lemma}\label{lem:bound-B}
    For all $t \in \mathbb{N}$ and all $v,c$, we have that $(\mathbf{B}^{(t)})_{vc} \leq n$ where $n$ is the size of $G$
    so that $\mathbf{A}$ is an $n \times n$ matrix.
\end{lemma}

Let $C'$ be an index set containing the, say $r$, unique rows of $\mathbf{B}^{(t)}$.
As before, let $\widetilde{\mathbf{B}^{(t)}}$ consist of the rows $\mathbf{B}^{(t)}_{c'\bullet}$, $c'\in C'$.

It is shown in~\cite{grohewl} that:
\begin{lemma}[Lemma 9 in~\cite{grohewl}]
There exists a matrix $\mathbf{Z}^{(t)}$ such that $\text{sign}(\widetilde{\mathbf{B}^{(t)}}\mathbf{Z}^{(t)}-\mathbf{J})$ consists of linearly independent rows.\qed
\end{lemma}
We complement this (below) by 
\begin{lemma}
If $\mathbf{B}$ does not contain a row consisting of zeroes, then
there exists a matrix $\mathbf{Z}^{(t)}$ and constant $m^{(t)}$ such that $\text{\normalfont ReLU}(\widetilde{\mathbf{B}^{(t)}}\mathbf{Z}^{(t)}-m^{(t)}\mathbf{J})$ consists of linearly independent rows.\qed
\end{lemma}
\todo{G: we should argue that some initial condition on F guarantees we never get 0 rows. }
\todo{F: I propose to assume that \textbf{no isolated vertices are present}. Then, by construction of the
feature vectors neither sign nor relu will create all zero rows. This should be checked inductively.}
These lemmas imply that 
\begin{equation}
\sigma(\mathbf{B}^{(t)}\mathbf{Z}^{(t)}- m^{(t)}\mathbf{J}) \label{eq:upd}
\end{equation}
is row-independent modulo equality when $\sigma$ is either sign or ReLU. Furthermore, since the vertex labelling induced by $\mathbf{B}^{(t)}$ was shown to be equivalent to the vertex labelling induced by 1-WL on $\mathbf{F}^{(t)}$, we have that the vertex labelling induced by~(\ref{eq:upd}) also has this property. Hence, looking back at the construction of $\mathbf{B}^{(t)}$ (which involved the matrix $\mathbf{M}^{(t)}$), we define $\mathbf{W}^{(t)}$ to be 
$\mathbf{M}^{(t)}\mathbf{Z}^{(t)}$. In other words,
\begin{equation}
\mathbf{F}^{(t+1)}:=\sigma(\mathbf{A}\mathbf{Y}^{(t)}\mathbf{F}^{(t)}\mathbf{W}^{(t)} - m^{(t)}\mathbf{J}) \label{eq:finalupd}
\end{equation}
is again a feature matrix satisfying conditions (a) and (b).
 
% \subsection{Multiplying from the right} 
\subsection{Multiplying from both sides.}\label{subsec:right}
\begin{definition}\label{def:rightmult}
Consider another diagonal non-negative $n\times n$ matrix $\mathbf{X}^{(t)}$  satisfying,
for any $i,j\in[1,n]$:
\begin{equation}
\mathbf{F}_{i\bullet}^{(t+1)}=\mathbf{F}_{j\bullet}^{(t+1)} \Longrightarrow \mathbf{X}^{(t)}_{ii}=\mathbf{X}^{(t)}_{jj}. \label{eq:cond2}
\end{equation}
\end{definition}

Lemma~\ref{lem:gfl} then implies that $\mathbf{X}^{(t)}\mathbf{F}^{(t+1)}$ also satisfies conditions (a) and (b). Indeed, we have shown that the vertex labelling induced by $\mathbf{F}^{(t)}$ and $\mathbf{F}^{(t+1)}$ correspond to the 1-WL labelings $\ell^{(t)}$ and $\ell^{(t+1)}$, respectively.
Since $\ell^{(t+1)}$ is a refinement of $\ell^{(t)}$, this implies that $\mathbf{F}_{i\bullet}^{(t+1)}=\mathbf{F}_{j\bullet}^{(t+1)} \implies \mathbf{F}_{i\bullet}^{(t)}=\mathbf{F}_{j\bullet}^{(t)}$. Hence, we can indeed use Lemma~\ref{lem:gfl}.
We note that
$$
\mathbf{X}^{(t)}\mathbf{F}^{(t+1)}=\mathbf{X}^{(t)}\sigma(\mathbf{A}\mathbf{Y}^{(t)}\mathbf{F}^{(t)}\mathbf{W}^{(t)} - m^{(t)}\mathbf{J})=\sigma(\mathbf{X}^{(t)}\mathbf{A}\mathbf{Y}^{(t)}\mathbf{F}^{(t)}\mathbf{W}^{(t)} - m^{(t)}\mathbf{X}^{(t)}\mathbf{J}),
$$
when $\sigma$ is either sign or ReLU. In other words, if we consider the update rule
\begin{equation}
\mathbf{F}^{(t+1)}:=\sigma(\mathbf{X}^{(t)}\mathbf{A}\mathbf{Y}^{(t)}\mathbf{F}^{(t)}\mathbf{W}^{(t)} - m^{(t)}\mathbf{X}^{(t)}\mathbf{J}) \label{eq:realfinalupd}
\end{equation}
then this is again a feature matrix satisfying Definition~\ref{def:label}.

We thus have shown how to, starting from $\mathbf{F}^{(t)}$, generate $\mathbf{F}^{(t+1)}$ whilst preserving Definition~\ref{def:label}. We kick-start by using $\mathbf{F}^{(0)}$, a $n\times q$-matrix which is row-independent modulo equality and
such that its induced vertex labelling is equivalent to the vertex labelling after $k$ iterations of 1-WL on a uniform labelling of vertices. Hence,

\begin{proposition}\label{pro:gen+bias}
Every architecture that allows for updates of the form
$$\mathbf{F}^{(t+1)}:=\sigma(\mathbf{X}^{(t)}\mathbf{A}\mathbf{Y}^{(t)}\mathbf{F}^{(t)}\mathbf{W}^{(t)} - m^{(t)}\mathbf{X}^{(t)}\mathbf{J}) $$
satisfies Definition~\ref{def:gen+bias}.
Here, $\mathbf{X}^{(t)}$ and $\mathbf{Y}^{(t)}$ are positive diagonal matrices satisfying Definitions~\ref{def:rightmult} and~\ref{def:gfl}, respectively. 
\end{proposition}


\paragraph{Applications.}
In particular, when $\mathbf{X}^{(t)}=\mathbf{Y}^{(t)}=\mathbf{I}_{n\times n}$, then this proposition reduces to the statement in~\cite{grohewl}. Another choice is $\mathbf{X}^{(t)}=\mathbf{Y}^{(t)}=\mathbf{D}_{n\times n}^{-1/2}$, where
$\mathbf{D}$ is the degree matrix of $\mathbf{A}$. Assuming that no isolated vertices are present, this indeed results in positive diagonal matrices. Furthermore, condition~(\ref{eq:cond1}) is satisfied provided that the vertex labelling of  $\mathbf{F}^{(0)}$ is equivalent to the vertex labelling after one iteration of 1-WL on a uniform labelling of vertices.
Indeed, after one such iteration of 1-WL, the corresponding vertex labelling has incorporated degree information, and hence vertices with same label cannot have distinct degrees. Consequently, vertices with the same rows in $\mathbf{F}^{(0)}$ cannot have distinct degrees. Hence,  condition~(\ref{eq:cond1}) is satisfied.

% \begin{proposition}\label{pro:kipf}
% Let $\mathbf{F}^{(0)}$ be a feature matrix which is row-independent modulo equality and
% such that its induced vertex labelling is equivalent to the vertex labelling after $1$ iteration of 1-WL on a uniform labelling of vertices. Then, for  every $t\geq 0$ there exists a weight matrix $\mathbf{W}^{(t)}$ and constant $m^{(t)}$ such that the vertex labelling induced by 
% $$\mathbf{F}^{(t+1)}:=\sigma(\mathbf{D}^{-1/2}\mathbf{A}\mathbf{D}^{-1/2}\mathbf{F}^{(t)}\mathbf{W}^{(t)} - m^{(t)}\mathbf{D}^{-1/2}\mathbf{J}) $$
% is equivalent to the vertex labelling after $t+1$ iterations of 1-WL, starting from a uniform vertex labelling. 
% \end{proposition}

\begin{corollary}\label{cor:normalised}
The normalised architecture satisfies Definition~\ref{def:gen+bias}
\end{corollary}

\subsection{Labelled graphs}\label{sec:labelled-graphs}
We next consider the case when $G$ is a labeled graphs in which the initial vertex labelling is not necessarily uniform.
To accommodate for such initial labelings, we introduce a new (learnable) weight vector $\mathbf{w}^{(t)}$ and put it on the diagonal of a matrix, i.e., $\text{diag}(\mathbf{w}^{(t)})$, and
consider GNNs of the form:
$$\mathbf{F}^{(t+1)}:=\sigma(\mathbf{X}^{(t)}(\mathbf{A}+\text{diag}(\mathbf{w}^{(t)}))\mathbf{Y}^{(t)}\mathbf{F}^{(t)}\mathbf{W}^{(t)} - m^{(t)}\mathbf{X}^{(t)}\mathbf{J}).$$

If we inductively assume that vertex labelling induced by $\mathbf{F}^{(t)}$ is equivalent to the one induced by 1-WL, starting from the initial (not necessarily uniform) labelling $\ell$ of $G$, then if we inspect the proof for the unlabelled case, we only need to ensure that the vertex labelling induced by
\begin{equation}
\mathbf{A}\mathbf{Y}^{(t)}\mathbf{F}^{(t)}\mathbf{M}^{(t)} + \text{diag}(\mathbf{w}^{(t)})\mathbf{Y}^{(t)}\mathbf{F}^{(t)}\mathbf{M}^{(t)} \label{eq:labeled}
\end{equation}
corresponds again to one step of 1-WL (on labeled graphs) starting from $\mathbf{F}^{(t)}$. We note, however, that 
$$
(\mathbf{Y}^{(t)}\mathbf{F}^{(t)}\mathbf{M}^{(t)})_{v,c}=\begin{cases} 1 & \text{if $v$ has colour $c$}\\
0 &\text{otherwise}
\end{cases}
$$
and recall  that $\mathbf{B}^{(t)}=\mathbf{A}\mathbf{Y}^{(t)}\mathbf{F}^{(t)}\mathbf{M}^{(t)}$ and
$$
(\mathbf{B}^{(t)})_{v,c}=\text{number of neighbours with colour $c$}.
$$
The 1-WL update rule, however, does not only take the colours of neighbours (and the number of neighbours of the same colour) into account. It also requires
to incorporate the initial labels (or, equivalently, the current colour).  This information is not necessarily reflected in $\mathbf{B}^{(t)}$. That is, there may be two equal rows in $\mathbf{B}^{(t)}$ that correspond to vertices with a different
initial label. 
\todo{F: Counter example two disjoint edges with nodes colours red-red, red-green.}
Define
$$
\delta=\min_{v,w,c}\{ | \mathbf{B}^{(t)}_{v,c}-\mathbf{B}^{(t)}_{w,c}\mid \mathbf{B}^{(t)}_{v,c}\neq \mathbf{B}^{(t)}_{w,c}\},
$$
i.e., the smallest non-zero difference between entries in $\mathbf{B}^{(t)}$.
Let $\epsilon<\delta$. Note $\delta\geq 1$. So, choosing $\epsilon$ close to $1$ will be fine.
%Let the current colours be $c_1,\ldots,c_K$ and define
%$\epsilon_{c_i}=i\times \epsilon$. 
We define
$$
\mathbf{w}^{(t)}_v=\epsilon.
$$
%if $v$ its current  colour is $c_i$.
%
%We deal with this, as follows. Let $[n]=I_1\uplus I_2\uplus\cdot\uplus I_k$ be a partition of the rows $\mathbf{B}^{(t)}$ according
%to row-equality. That is, for any $v,w\in I_i$, $(\mathbf{B}^{(t)})_{v,\bullet}=(\mathbf{B}^{(t)})_{w,\bullet}$. We distinguish between the following cases:
%\begin{itemize}
%\item If all vertices in $I_i$ have the same initial labelling, then we set $\mathbf{w}^{(t)}_v=\epsilon_i=0$
%for all $v\in I_i$, i.e., no correction to $(\mathbf{B}^{(t)})_{v\bullet}$ is needed. 
%\item Otherwise, we further partition $I_i$ into
%$I_i=I_{i1}\uplus\cdots\uplus I_{ik_i}$ induced by the initial labelling. More precisely, $v,w\in I_{ij}$ if $v$ and $w$ have the same initial label and $(\mathbf{B}^{(t)})_{v,\bullet}=(\mathbf{B}^{(t)})_{w,\bullet}$.
%We set $\mathbf{w}^{(t)}_v=\epsilon_j$
%of all $v\in I_{ij}$, $j\in[1,k_i]$, where all $\epsilon_j$ are different.
%\end{itemize}
Hence, $(\mathbf{A}\mathbf{Y}^{(t)}\mathbf{F}^{(t)}\mathbf{M}^{(t)})_{v,c} + (\text{diag}(\mathbf{w}^{(t)})\mathbf{Y}^{(t)}\mathbf{F}^{(t)}\mathbf{M}^{(t)})_{v,c}$
is equal to
\begin{equation}\label{eqn:count+epsilon}
\text{number of neighbours with colour $c$} +  \begin{cases} \epsilon & \text{if $v$ has colour $c$.}\\0 &\text{otherwise}
\end{cases}
\end{equation}
We note that the choice of $\epsilon$'s guarantee that adding the $\epsilon$
to some elements in $\mathbf{B}^{(t)}$ will never cause two distinct rows in
$\mathbf{B}^{(t)}$ to become equal. Indeed, let $\mathbf{B}_{v,\bullet}$ and
$\mathbf{B}_{w,\bullet}$ be two distinct rows. Suppose that $\epsilon$ is
added to the $i$th entry of  $\mathbf{B}_{v,\bullet}$  and $\epsilon$ to the
$j$the entry of $\mathbf{B}_{w,\bullet}$ (note that only one such column is
incremented since every vertex is labelled by exactly one colour).
Suppose that after these
additions the rows become equal. Suppose that $i\neq j$. Then this implies
that $j$th entry of $\mathbf{B}_{v,\bullet}$ was equal to $j$th entry of
$\mathbf{B}_{w,\bullet}$ minus $\epsilon$. This, however, is impossible since
$\epsilon<\delta$. If $i=j$, and $v$ and $w$ have the colour and the same
epsilon values is added to the $i$th entry of $\mathbf{B}_{v,\bullet}$ and
$\mathbf{B}_{w,\bullet}$. If these rows would be the same, then $i$the entry
of $\mathbf{B}_{v,\bullet}$ and $\mathbf{B}_{w,\bullet}$  were equal already.
Since we assumed  $\mathbf{B}_{v,\bullet}$ and $\mathbf{B}_{w,\bullet}$ to be
distinct and no other entries (than the $i$th entry) in these vectors is
incremented, they will remain distinct. So, distinct rows remain distinct.
% Finally, if $i=j$ but $v$ and $w$ have different initial labels, then $i$the entry of 
%$\mathbf{B}_{v,\bullet}$ and $\mathbf{B}_{w,\bullet}$ would be $|\epsilon_1-\epsilon_2|$ apart (if they would have become equal after being incremented). Since $|\epsilon_1-\epsilon_2|$ is a multiple $L\times \epsilon$ with $L\leq K$ we have that $|\epsilon_1-\epsilon_2|<\delta$. This contradicts that $\delta$ is the smallest distance between elements in $\mathbf{B}^{(t)}$.

To see what happens when  $\mathbf{B}_{v,\bullet}$ and $\mathbf{B}_{w,\bullet}$ are equal. If $v$ and $w$ have different colours, then, $\epsilon$ is added to these rows in different entries   in $\mathbf{B}_{v,\bullet}$ and $\mathbf{B}_{w,\bullet}$. Hence, the
increments make these two rows distinct. The argument above shows that these new rows will not coincide with any other distinct row. 
Finally, if $v$ and $w$ have the same colour, then $\epsilon$ is added to the same column in $\mathbf{B}_{v,\bullet}$ and $\mathbf{B}_{w,\bullet}$, so these
rows remain the same. So, rows that that were equal remain equal when the corresponding vertices have the same colour, and are made different when they have a different colour.

In summary, by considering
$(\mathbf{A}\mathbf{Y}^{(t)}\mathbf{F}^{(t)}\mathbf{M}^{(t)}+ \epsilon
\mathbf{I}\mathbf{Y}^{(t)}\mathbf{F}^{(t)}\mathbf{M}^{(t)})$ we ensure that each
unique row corresponds to 
the vertex labelling induced by 1-WL. We then proceed as before, i.e., by constructing the weight matrix and bias. We thus have that there exists for each $t>0$, there exists a weight matrix $\mathbf{W}^{(t)}$
and constants $\epsilon^{(t)}$ and $m^{(t)}$ such that 
$$\mathbf{F}^{(t+1)}:=\sigma(\mathbf{X}^{(t)}(\mathbf{A}+\epsilon^{(t)}
\mathbf{I})\mathbf{Y}^{(t)}\mathbf{F}^{(t)}\mathbf{W}^{(t)} - m^{(t)}\mathbf{X}^{(t)}\mathbf{J}).$$
corresponds to the vertex labelling induced by 1-WL and where the feature vectors are independent modulo row-equality.

\subsubsection{Modifications to obtain a fixed m}
We would now like to claim that Proposition~\ref{pro:fixed-m} still holds for
the present context. Although Lemma~\ref{lem:bound-B} is no longer valid, since
we have modified the architecture, the following analogue does hold and
follows from Equation~\eqref{eqn:count+epsilon}.
\begin{lemma}\label{lem:bound-B}
    For all $t \in \mathbb{N}$ and all $v,c$, we have that
    $(\mathbf{B}^{(t)})_{vc} \leq n + \epsilon$ where $n$ is the size of $G$ so that
    $\mathbf{A}$ is an $n \times n$ matrix.
\end{lemma}

Using the above lemma and following the suggestion of fixing $\epsilon < 1$ we
obtain our new version of Proposition~\ref{pro:fixed-m} for labelled graphs.
\begin{proposition}
    Let $(\mathbf{W}^{(i)})^t_{i=0}$ be a matrix sequence such that the dimensions
    of all $\mathbf{W}^{(i)}$ are at most $q$, $1 > \epsilon = \frac{a}{b}$, and set
    \[
      m := \frac{q(n+1)^{q+1} + \frac{b-1}{b} - 1}{q(n+1)^{q+1}}.
    \]
    Then, there exist matrices $(\mathbf{Z}^{(i)})_{i=0}^t$ such that $\mathrm{ReLU}(\mathbf{B}^{(i)}\mathbf{Z}^{(i)} - m\mathbf{J})$ consists of linearly independent rows for all $0 \leq i \leq t$.
\end{proposition}
\todo{F. Same comment as before: why not use $\mathbf{B}$? Why do you need the upper bound $t$?}

\section{An upper bound for the Kipf+bias architecture}
Recall the architecture from Proposition~\ref{pro:gen+bias}
and observe the following.
\begin{align}
    \mathbf{F}^{(t+1)}_{i\bullet} &= \sigma(
        \mathbf{X}^{(t)}_{ii} \mathbf{A}_{i\bullet}
        \mathbf{Y}^{(t)}\mathbf{F}^{(t)}\mathbf{W}^{(t)}
        -m^{(t)}\mathbf{X}_{ii}^{(t)}\mathbf{J})\\
    &= \sigma \left(
        \mathbf{X}^{(t)}_{ii}
        \begin{bmatrix}
        \mathbf{A}^{(t)}_{i1} \mathbf{Y}^{(t)}_{11} &
        \cdots & \mathbf{A}^{(t)}_{ij} \mathbf{Y}^{(t)}_{jj} & \cdots
        \end{bmatrix}
        \mathbf{F}^{(t)}\mathbf{W}^{(t)}
        -m^{(t)}\mathbf{X}_{ii}^{(t)}\mathbf{J})
        \right)\\
    &= \sigma \left(
        \mathbf{X}^{(t)}_{ii}
        \begin{bmatrix}
        \cdots &
        \sum_{k}\mathbf{A}^{(t)}_{ik} \mathbf{Y}^{(t)}_{kk} \mathbf{F}^{(t)}_{k j}  & \cdots
        \end{bmatrix}
        \mathbf{W}^{(t)}
        -m^{(t)}\mathbf{X}_{ii}^{(t)}\mathbf{J})
        \right)\\
    &= \sigma \left(
        \mathbf{X}^{(t)}_{ii}
        \begin{bmatrix}
        \cdots & \sum_{k \in N_G(i)} \mathbf{Y}^{(t)}_{kk} \mathbf{F}^{(t)}_{k j} & \cdots
        \end{bmatrix}
        \mathbf{W}^{(t)}
        -m^{(t)}\mathbf{X}_{ii}^{(t)}\mathbf{J})
        \right)\label{eq:fkn-indices}
\end{align}

Let us denote by $\mathbf{B}^{(t,i)}$ the row vector $[
\cdots \sum_{k \in N_G(i)} \mathbf{Y}^{(t)}_{kk} \mathbf{F}^{(t)}_{k j} \cdots]$ and by $M_i$ the feature-vector multiset $\ldbl \mathbf{F}^{(t)}_{k\bullet} \st k \in N_G(i) \rdbl$. The following
observation will be useful.
\begin{lemma}\label{lem:dumb-obs}
    If $M_i = M_{j}$ and
    condition~\eqref{eq:cond1} holds
    then $\mathbf{B}^{(t,i)} = \mathbf{B}^{(t,j)}$.
\end{lemma}

%A labelling $\ell'$ is said to be coarser than the
%labelling $\ell$ if and only if $\ell(u) = \ell(v)
%\implies \ell'(u) = \ell'(v)$. 
We will now argue that 
our architecture always yields coarser updates than the
one used in the $1$-WL algorithm.
\begin{proposition}\label{pro:upper-bound}
Let $\mathbf{F}^{(0)}$ be a feature matrix which is row-independent modulo equality and
such that its induced vertex labelling is equivalent to the vertex labelling after $k\geq 1$ iterations of 1-WL on a uniform labelling of vertices. Then, for every $t\geq 0$, for all weight matrices $\mathbf{W}^{(t)}$ and all constants $m^{(t)}$, the vertex labelling induced by 
$$\mathbf{F}^{(t+1)}:=\sigma(\mathbf{X}^{(t)}\mathbf{A}\mathbf{Y}^{(t)}\mathbf{F}^{(t)}\mathbf{W}^{(t)} - m^{(t)}\mathbf{X}^{(t)}\mathbf{J}) $$
is coarser to the vertex labelling after $k+t$ iterations of 1-WL. Here, $\mathbf{X}^{(t)}$ and $\mathbf{Y}^{(t)}$ are positive diagonal matrices satisfying the conditions~\eqref{eq:cond1} and~\eqref{eq:cond2}, respectively. 
\end{proposition}
\begin{proof}
    We argue that for all $t \geq 0$ it holds that $\ell^{(k+t)}(i) = \ell^{(k+t)}(j)
    \implies \mathbf{F}^{(k+t)}_{i\bullet}
    = \mathbf{F}^{(k+t)}_{j\bullet}$. The
    claim holds trivially for $t=0$ so we have a base case.
    
    For the inductive step, let $i,j$ be 
    arbitrary vertex indices such that
    $\ell^{(t+k+1)}(i)=\ell^{(t+k+1)}(j)$.
    From the definition of the $1$-WL
    update, we know that
    \[
        \ldbl \ell^{(k+t)}(i') \st i' \in N_G(i) \rdbl
        =
        \ldbl \ell^{(k+t)}(j') \st j' \in N_G(j) \rdbl,
    \]
    hence $M_i = M_j$ by induction hypothesis.
    Since we have assumed $\mathbf{Y}^{(t)}$
    satisfies condition~\eqref{eq:cond1}, it follows from Lemma~\ref{lem:dumb-obs} that $\mathbf{B}^{(k+t+1,i)} = \mathbf{B}^{(k+t+1,j)}$. To conclude, we
    note that since we have further assumed condition~\eqref{eq:cond2} holds, Equation~\eqref{eq:fkn-indices} gives us
    the desired result.
\end{proof}

\subsection{Modifications for the labelled case}
We now focus on the following architecture suggested
in Section~\ref{sec:labelled-graphs}. We observe the following.
\begin{align}
    \mathbf{F}^{(t+1)}_{i\bullet} &= \sigma(
        \mathbf{X}^{(t)}_{ii} (\mathbf{A} + \epsilon^{(t)}\mathbf{I})_{i\bullet}
        \mathbf{Y}^{(t)}\mathbf{F}^{(t)}\mathbf{W}^{(t)}
        -m^{(t)}\mathbf{X}_{ii}^{(t)}\mathbf{J})\\
    &= \sigma \left(
        \mathbf{X}^{(t)}_{ii}
        \begin{bmatrix}
        \cdots & \left(\sum_{k \in N_G(i)} \mathbf{Y}^{(t)}_{kk} \mathbf{F}^{(t)}_{k j}\right) + \epsilon^{(t)}\mathbf{Y}^{(t)}_{ii}\mathbf{F}^{(t)}_{ij} & \cdots
        \end{bmatrix}
        \mathbf{W}^{(t)}
        -m^{(t)}\mathbf{X}_{ii}^{(t)}\mathbf{J})
        \right)
\end{align}

Similarly to our previous argument, we now
denote by $\mathbf{C}^{(t,i)}$ the row vector 
\[
    \begin{bmatrix}
        \cdots & \left(\sum_{k \in N_G(i)} \mathbf{Y}^{(t)}_{kk} \mathbf{F}^{(t)}_{k j}\right) + \epsilon^{(t)}\mathbf{Y}^{(t)}_{ii}\mathbf{F}^{(t)}_{ij} & \cdots
    \end{bmatrix}
\]
and by $M_i$ the feature-vector multiset $\ldbl \mathbf{F}^{(t)}_{k\bullet} \st k \in N_G(i) \rdbl$. We 
will need the following analogue of Lemma~\ref{lem:dumb-obs}.
\begin{lemma}
    If $M_i = M_{j}$,
    condition~\eqref{eq:cond1} holds, and $\mathbf{F}^{(t)}_{i\bullet} = \mathbf{F}^{(t)}_{j\bullet}$,
    then $\mathbf{C}^{(t,i)} = \mathbf{C}^{(t,j)}$.
\end{lemma}

For all $t,k$ the $1$-WL algorithm not only
gives us that
\[
    \ldbl \ell^{(k+t)}(i') \st i' \in N_G(i) \rdbl
    =
    \ldbl \ell^{(k+t)}(j') \st j' \in N_G(j) \rdbl,
\]
if $\ell^{(t+k+1)}(i)=\ell^{(t+k+1)}(j)$, but also that
$\ell^{(k+t)}(i) = \ell^{(k+t)}(j)$. Hence, using the above lemma, we can repeat the argument used to prove Proposition~\ref{pro:upper-bound} to obtain the following.


\begin{proposition}
Let $\mathbf{F}^{(0)}$ be a feature matrix which is row-independent modulo equality and
such that its induced vertex labelling is equivalent to the vertex labelling after $k\geq 1$ iterations of 1-WL on a uniform labelling of vertices. Then, for every $t\geq 0$, for all weight matrices $\mathbf{W}^{(t)}$ and all constants $m^{(t)}$, $\epsilon^{(t)}$, the vertex labelling induced by 
$$\mathbf{F}^{(t+1)}:=\sigma(\mathbf{X}^{(t)}(\mathbf{A} + \epsilon^{(t)}\mathbf{I})\mathbf{Y}^{(t)}\mathbf{F}^{(t)}\mathbf{W}^{(t)} - m^{(t)}\mathbf{X}^{(t)}\mathbf{J}) $$
is coarser to the vertex labelling after $k+t$ iterations of 1-WL. Here, $\mathbf{X}^{(t)}$ and $\mathbf{Y}^{(t)}$ are positive diagonal matrices satisfying the conditions~\eqref{eq:cond1} and~\eqref{eq:cond2}, respectively.
\end{proposition}


\section{Architecture}
It is often reported that the expressive power of GNN architectures is bounded by the one-dimensional Weisfeiler-Lehman algorithm (or 1-WL for short). That is, when two vertices
are assigned the same label by 1-WL, then also the feature vectors computed by GNNs of these vertices will be the same. Intuitively, this means that the distinguishing power of vertices of GNNs is weaker than that of 1-WL. The connection between the GNN architectures mentioned above and the 1-WL algorithm has, to our knowledge, not been made precise. We next formally describe this connection by providing both lower and upper bounds of the existing architectures and the architecture of the form~(\ref{eq:architecture}).

\section{Upper bounding the expressive power}
We start by investigateing the limit of the expressive power of the GNN architecture~(\ref{eq:architecture}).
%\begin{equation}
%\mathbf{F}^{(t+1)}:=\sigma\left(\mathbf{L}(\mathbf{A}+p\mathbf{I})\mathbf{R}\mathbf{F}^{(t)}\mathbf{W}^{(t)} + q\mathbf{B}\right), \label{eq:architecture}
%\end{equation}
%where $\mathbf{L}$ and $\mathbf{R}$ are non-negative diagonal matrices,  $\mathbf{B}$ is a bias matrix, $\mathbf{I}$ is the identity matrix,  $p$ and $q$ are learnable parameters in $[0,1]$, 
%$\mathbf{W}^{(t)}$ is a learnable weight matrix, and $\sigma$ is a non-linear activation function such as sgn or ReLU. 


% We first assume that the following conditions are satisfied for all $t\geq 0$:
% \begin{equation}
% \mathbf{F}^{(t)}\sqsubseteq\mathbf{L},\quad
% \mathbf{F}^{(t)}\sqsubseteq\mathbf{R},\text{ and }
% \mathbf{F}^{(t)}\sqsubseteq\mathbf{B}. \label{eq:conditions}
% \end{equation}
% In other words, the vertex labellings induced by the matrices $\mathbf{L}$, $\mathbf{R}$ and
% $\mathbf{B}$ are coarser than the vertex labelling induced by $\mathbf{F}^{(t)}$.





\end{document}
